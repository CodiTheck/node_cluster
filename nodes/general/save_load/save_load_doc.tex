% Document type and package imports.
\documentclass[a4paper, 11pt]{article}
\usepackage[utf8]{inputenc}
\usepackage[T1]{fontenc}
\usepackage[french]{babel}
\usepackage{charter}
\usepackage[top = 2cm, bottom = 2cm, left = 1cm, right = 1cm]{geometry}
\usepackage{setspace}
\usepackage{color}
\usepackage{xcolor}
\usepackage{hyperref}
\usepackage{tocloft}

% Preanblue.
\onehalfspacing
\definecolor{gray}{rgb}{0.4, 0.4, 0.4}
\definecolor{silver}{rgb}{0.95, 0.95, 0.95}
\renewcommand{\thesection}{\Roman{section} --}
\definecolor{darkgreen}{HTML}{1E8C15}
\cftsetindents{section}{1em}{2.5em}
\hypersetup {colorlinks=true, linkcolor=blue, urlcolor=blue, pdftitle={SaveLoad module doc}}

% The start of the article.
\begin{document}
	% Change document background to silver color.
	\pagecolor{silver}
	% SaveLoad module description.
	\huge{\hspace{14.5cm}\textit{\textbf{\textcolor{darkgreen}{SaveLoadFx}}}}\large{} \tableofcontents 
	\newpage
	% SaveLoad module definition.
	\section{Définition}
	\textcolor{darkgreen}{\textbf{SaveLoadFx}} est un module conçut pour la sauvegarde et le chargement de 
	données dans un jeu. Ce dernier permettra aux développeurs de sauvegarder et de charger facilement leurs 
	données. Cependant, les données du jeu sont déposées dans un fichier configuré par le développeur lui-
	même.\\
	\textcolor{red}{\textbf{NB}:} La sauvegarde et le chargement des données peuvent être sécurisées. Notez
	également que ce module est de nature indestructible, est compatible à un jeu 2D, 3D et n'est pas 
	sauvegardable.

	% SaveLoad properties definition.
	\section{Les propriétés disponibles}
	% TargetPath property.
	\textbf{+ \textcolor{red}{int} \hypertarget{idx}{TargetPath} = \textcolor{blue}{0}:} Contient les 
	différents chemins que prend en charge ce module. Ces chemins \\représentent les endroits possibles où
	l'on peut déposé le fichier de sauvegarde du jeu. Les valeurs \\possibles sont celles définient au sein
	de la méthode \textit{\textcolor{blue}{get\_os\_dir ()}} de la classe \textbf{\textcolor{darkgreen}
	{MegaAssets}}.\\\\
	% Source property.
	\textbf{+ \textcolor{darkgreen}{String} Source:} Contient le chemin pointant vers le fichier de
	sauvegarde définit ou à créé par le module au cours des sauvegardes.\\\\
	% ActiveCheckpoint property.
	\textbf{+ \textcolor{darkgreen}{String} ActiveCheckpoint:} Contient le nom du point de sauvegarde à
	ciblé pour la gestion des \\données du jeu sur le fichier de sauvegarde. Elle représente également le
	point de sauvegarde actif. Cette option met le focus sur le point de sauvegarde donné parmi plusieurs 
	autres points de sauvegarde dans lequel les sauvegardes et les chargements de données seront éffectuées.
	Notez que le fichier de sauvegarde peut avoir un ou plusieurs point(s) de sauvegarde. Pour pouvoir 
	sauvegarder et/ou charger les données du jeu, il faudra préciser le point de sauvegarde à ciblé pour 
	effectuer l'opération demandée. Pour tous traitement à faire sur les données du jeu, on solicitera par 
	défaut, le point de sauvegarde actif au niveau du fichier de sauvegarde.\\\\
	% Security property.
	\textbf{+ \textcolor{red}{int} \hypertarget{idx}{Security} = \textcolor{blue}{7}:} Contient les
	différents modes de sécurité possibles que l'on peut utilisé au cours de la sauvegarde des données du
	jeu. Les valeurs possibles sont celles définient au sein de la méthode \textit{\textcolor{blue}
	{serialize ()}} de la classe \textbf{\textcolor{darkgreen}{MegaAssets}}.\\\\
	% Level property.
	\textbf{+ \textcolor{red}{int} \hypertarget{idx}{Level} = \textcolor{blue}{1}:} Contient les différents
	niveaux de sécurité possibles que l'on peut utilisé au cours de la sauvegarde des données du jeu. Les
	valeurs possibles sont celles définient au sein de la méthode \textit{\textcolor{blue}{serialize ()}} de 
	la classe \textbf{\textcolor{darkgreen}{MegaAssets}}.\\\\
	% FileChecksum property.
	\textbf{+ \textcolor{red}{int} FileChecksum = \textcolor{blue}{1}:} Quelle méthode de chiffrement
	voulez-vous utiliser pour générer le \\checksum du fichier de sauvegarde ? Les valeurs possibles sont
	celles définient au sein de la méthode \textit{\textcolor{blue}{serialize ()}} de la classe 
	\textbf{\textcolor{darkgreen}{MegaAssets}}.\\\\
	% Key property.
	\textbf{+ \textcolor{darkgreen}{String} \hypertarget{pass}{Key}:} Contient le mot de passe à utilisé
	pour sécuriser les données du jeu. Le développeur a la possibilité de générer automatiquement un mot de
	passe via le booléen \textit{\hyperlink{genpass}{GeneratePassword}}.\\\\
	% ScreenCapture property.
	\textbf{+ \textcolor{red}{bool} ScreenCapture = \textcolor{red}{false}:} Voulez-vous éffectuer une
	capture d'écran à chaque sauvegarde \\éffectuée ?\\\\
	% Size property.
	\textbf{+ \textcolor{darkgreen}{Vector2} Size = \textcolor{darkgreen}{Vector2} (\textcolor{blue}{64}, 
	\textcolor{blue}{64}):} Contrôle la résolution des captures d'écran à générées lors des sauvegardes.\\\\
	% Quality property.
	\textbf{+ \textcolor{red}{int} Quality = \textcolor{blue}{2}:} Contrôle la qualité des captures d'écran
	à générées lors des sauvegardes. Les \\valeurs possibles sont celles définient au sein de la classe
	\href{https://docs.godotengine.org/en/stable/classes/class_image.html#class-image-constant-interpolate-bilinear}{\textit{\textcolor{darkgreen}{Image}}} de Godot.\\\\
	% CompressMode property.
	\textbf{+ \textcolor{red}{int} CompressMode = \textcolor{blue}{2}:} Quel mode de compression voulez-vous
	adopter pour compresser les \\captures d'écran générées lors des sauvegardes ? Les valeurs possibles
	sont celles définient au sein de la méthode \textit{\textcolor{blue}{get\_screen\_shot ()}} de la classe 
	\textbf{\textcolor{darkgreen}{MegaAssets}}.\\\\
	% CompressSource property.
	\textbf{+ \textcolor{red}{int} CompressSource = \textcolor{blue}{2}:} A partir de quelle source de
	compression les compressions des captures d'écran générées lors des sauvegardes seront éffectuées ? Les
	valeurs possibles sont celles définient au sein de la classe
	\href{https://docs.godotengine.org/en/stable/classes/class_image.html#enum-image-compresssource}
	{\textit{\textcolor{darkgreen}{Image}}} de Godot.\\\\
	% CompressRatio property.
	\textbf{+ \textcolor{red}{float} CompressRatio = \textcolor{blue}{1000.0}:} Quel taux de compression
	appliqué aux captures d'écran \\générées lors des sauvegardes ?\\\\
	% Format property.
	\textbf{+ \textcolor{red}{int} Format = \textcolor{blue}{6}:} Quel sera le format des captures d'écran à 
	générées lors des sauvegardes ? Les valeurs possibles sont celles définient au sein de la méthode
	\textit{\textcolor{blue}{get\_screen\_shot ()}} de la classe \textbf{\textcolor{darkgreen}{MegaAssets}}.
	\\\\
	% LoadAllData property.
	\textbf{+ \textcolor{red}{bool} LoadAllData = \textcolor{red}{true}:} Voulez-vous charger
	automatiquement le fichier de sauvegarde à \\l'initialisation du jeu ?\\\\
	% TargetScenes property.
	\textbf{+ \textcolor{darkgreen}{PoolStringArray} TargetScenes:} Contient tous les noms en chaîne de
	caractères des scènes à \\ciblées. Cette option, permettra au module de charger ou de recharger le 
	fichier de sauvegarde ciblé par le développeur lorsque l'une des scènes spécifiées par ce dernier sera 
	charger et active.\\
	\textcolor{red}{\textbf{NB}:} La répétition du nom d'une ou de plusieurs scène(s) n'est pas tolérée.\\\\
	% GeneratePassword property.
	\textbf{+ \textcolor{red}{bool} \hypertarget{genpass}{GeneratePassword} = \textcolor{red}{false}:}
	Souhaitez-vous générer automatiquement un mot de passe ? A ce niveau, la valeur du champ
	\textit{\hyperlink{pass}{Key}} sera mise à jour à chaque génération.

	% SaveLoad methods definition.
	\section{Les méthodes disponibles}
	% Void generate_password () method description.
	\begin{description}
		\item [+ \textcolor{red}{void} \textcolor{blue}{generate\_password} (delay = 0.0):] Génère au 
		hazard, un mot de passe en fonction des configurations présentent à son niveau. La valeur du champ
		\textit{\hyperlink{pass}{Key}} sera mise à jour à chaque \\génération. \textbf{Notez que vous n'avez
		pas la possibilité de générer un mot de passe lorsque le jeu est en cours d'exécution}.
		\begin{itemize}
			\item [>> \textbf{\textcolor{red}{float} delay}:] Quel est le temps mort avant la génération ?\\
		\end{itemize}
	\end{description}
	% Void save_game_data () method description.
	\newpage \begin{description}
		\item [+ \textcolor{red}{void} \textcolor{blue}{save\_game\_data} (checkpoints = null, ignore =
		null, active\_camera = null, delay = 0.0):] Enregistre le dictionnaire des données du jeu dans le
		fichier de sauvegarde.
		\begin{itemize}
			\item [>> \textbf{\textcolor{darkgreen}{String | PoolStringArray} checkpoints}:] Quel(s) est/
			sont le(s) point(s) de sauvegarde à \\modifié(s) avant la sauvegarde complète des données du jeu
			? La mise à jour que subira les \\différents points de sauvegarde ne sera rien d'autre que
			l'insertion des clés spéciales suivantes: \textit{\textcolor{gray}{\_\_last\_save\_date\_\_,
			\_\_last\_save\_time\_\_, \_\_last\_load\_date\_\_, \_\_last\_load\_time\_\_,
			\_\_game\_time\_\_}} et \textit{\textcolor{gray}{\_\_game\_screenshot\_\_}} dont se base
			certaines fonctionnalités du module pour éffectuer leurs traitements. \textbf{Notez que si aucun
			point de sauvegarde n'est précisé, celui actif ne sera pas solicité}.
			\item [>> \textbf{\textcolor{darkgreen}{Variant} ignore}:] Contient la liste des clés à ignorées
			au cours de la sauvegarde des données. Notez que vous avez la possibilité d'ignorer un point de
			sauvegarde tout entier. Dans ce cas, toutes ces clés subiront le même sort.
			\item [>> \textbf{\textcolor{darkgreen}{Camera} active\_camera}:] Voulez-vous faire une capture
			d'écran à partir d'une autre caméra avant la sauvegarde générale des données du jeu ?
			\item [>> \textbf{\textcolor{red}{float} delay}:] Quel est le temps mort avant la sauvegarde des 
			données du jeu ?\\
		\end{itemize}
	\end{description}
	% Void load_game_data () method description.
	\begin{description}
		\item [+ \textcolor{red}{void} \textcolor{blue}{load\_game\_data} (checkpoints = null, ignore =
		null, delay = 0.0):] Charge les données du jeu à partir du fichier de sauvegarde créé par la
		méthode \textcolor{blue}{\textit{save\_game\_data ()}}.
 		\begin{itemize}
 			\item [>> \textbf{\textcolor{darkgreen}{String | PoolStringArray} checkpoints}:] Quel(s) est/
			sont le(s) point(s) de sauvegarde à \\modifié(s) après le chargement complet des données du jeu
			? Notez que si vous précisez des points de sauvegarde dans ce paramètre, Seule ces derniers 
			seront redéfinit dans le \\gestionnaire de données. Les autres conserveront leurs données. 
			\textbf{Par défaut, toutes les \\données du gestionnaire sont écrasées par celles chargées à
			partir du fichier de \\sauvegarde}.
			\item [>> \textbf{\textcolor{darkgreen}{String | PoolStringArray} ignore}:] Contient la liste
			des clés à ignorées au cours du chargement des données. Notez que vous avez la possibilité
			d'ignorer un point de sauvegarde tout entier. Dans ce cas, toutes ces clés subiront le même
			sort.
			\item [>> \textbf{\textcolor{red}{float} delay}:] Quel est le temps mort avant le chargement des 
			données du jeu ?\\
		\end{itemize}
	\end{description}
	% Bool is_progress () method description.
	\begin{description}
		\item [+ \textcolor{red}{bool} \textcolor{blue}{is\_progress} ():] Détermine s'il y a une quelconque
		progression éffectuée dans le jeu.\\
	\end{description}
	% String get_root_folders () method description.
	\begin{description}
		\item [+ \textcolor{darkgreen}{String} \textcolor{blue}{get\_root\_folders} ():] Renvoie les
		dossiers parents du fichier de sauvegarde.\\
	\end{description}
	% String get_full_path () method description.
	\begin{description}
		\item [+ \textcolor{darkgreen}{String} \textcolor{blue}{get\_full\_path} ():] Renvoie le chemin
		complet pointant vers le fichier de sauvegarde.\\
	\end{description}
	% Dictionary get_last_save_date () method description.
	\begin{description}
		\item [+ \textcolor{darkgreen}{Dictionary} \textcolor{blue}{get\_last\_save\_date} (checkpoint =
		''):] Renvoie la date de la dernière sauvegarde \\éffectuée pour un point de sauvegarde donné. Cette
		méthode exploite la clé \textit{\textcolor {gray}{\_\_last\_save\_date\_\_}} pour accomplir sa
		tâche. Le dictionaire renvoyé par cette méthode contient les clés suivantes:
		\textit{\textcolor{gray}{year, day, month, weekday}} et \textit{\textcolor{gray}{dst}} (heure
		d'été).
		\begin{itemize}
			\item [>> \textbf{\textcolor{darkgreen}{String} checkpoint}:] Contient le nom du point de 
			sauvegarde à ciblé. Par défaut, le point de \\sauvegarde actif est utilisé pour faire le
			traitement demandé.\\
		\end{itemize}
	\end{description}
	% Dictionary get_last_save_time () method description.
	\begin{description}
		\item [+ \textcolor{darkgreen}{Dictionary} \textcolor{blue}{get\_last\_save\_time} (checkpoint =
		'')]: Renvoie le temps de la dernière sauvegarde éffectuée pour un point de sauvegarde donné. Cette
		méthode exploite la clé \textit{\textcolor {gray}{\_\_last\_save\_time\_\_}} pour accomplir sa
		tâche. Le dictionaire renvoyé par cette méthode contient les clés suivantes:
		\textit{\textcolor{gray}{hour, minute}} et \textit{\textcolor{gray}{second}}.
		\begin{itemize}
			\item [>> \textbf{\textcolor{darkgreen}{String} checkpoint}:] Contient le nom du point de 
			sauvegarde à ciblé. Par défaut, le point de \\sauvegarde actif est utilisé pour faire le
			traitement demandé.\\
		\end{itemize}
	\end{description}
	% Int get_game_time () method description.
	\begin{description}
		\item [+ \textcolor{red}{int} \textcolor{blue}{get\_game\_time} (checkpoint = ''):] Renvoie le temps
		total écoulé toutes les fois où le jeu a été ouvert pour un point de sauvegarde donné. \textbf{Notez 
		qu'en cas d'échec, le temps écoulé depuis le démarrage du jeu sera renvoyé}. Cette méthode exploite
		la clé \textit{\textcolor{gray}{\_\_game\_time\_\_}} pour accomplir sa tâche.
		\begin{itemize}
			\item [>> \textbf{\textcolor{darkgreen}{String} checkpoint}:] Contient le nom du point de 
			sauvegarde à ciblé. Par défaut, le point de \\sauvegarde actif est utilisé pour faire le
			traitement demandé.\\
		\end{itemize}
	\end{description}
	% Int get_global_game_time () method description.
	\begin{description}
		\item [+ \textcolor{red}{int} \textcolor{blue}{get\_global\_game\_time} (checkpoint = ''):] Renvoie
		le temps global écoulé toutes les fois où le jeu a été ouvert pour un point de sauvegarde donné.
		\textbf{Notez qu'en cas d'échec, le temps écoulé depuis le démarrage du jeu sera renvoyé}. Cette
		méthode exploite la clé \textit{\textcolor{gray}{\_\_game\_time\_\_}} pour accomplir sa tâche.
		\begin{itemize}
			\item [>> \textbf{\textcolor{darkgreen}{String} checkpoint}:] Contient le nom du point de 
			sauvegarde à ciblé. Par défaut, le point de \\sauvegarde actif est utilisé pour faire le
			traitement demandé.\\
		\end{itemize}
	\end{description}
	% Dictionary get_last_load_date () method description.
	\begin{description}
		\item [+ \textcolor{darkgreen}{Dictionary} \textcolor{blue}{get\_last\_load\_date} (checkpoint =
		''):] Renvoie la date du dernier chargement \\éffectué pour un point de sauvegarde donné. Cette
		méthode exploite la clé \textit{\textcolor {gray}{\_\_last\_load\_date\_\_}} pour accomplir sa
		tâche. \textbf{Notez qu'en cas d'échec, la date du dernier chargement éffectué au niveau du fichier
		de sauvegarde sera renvoyé}. Le dictionaire renvoyé par cette méthode contient les clés suivantes:
		\textit{\textcolor{gray}{year, day, month, weekday}} et \textit{\textcolor{gray}{dst}} (heure
		d'été).
		\begin{itemize}
			\item [>> \textbf{\textcolor{darkgreen}{String} checkpoint}:] Contient le nom du point de 
			sauvegarde à ciblé. Par défaut, le point de \\sauvegarde actif est utilisé pour faire le
			traitement demandé.\\
		\end{itemize}
	\end{description}
	% Dictionary get_last_load_time () method description.
	\begin{description}
		\item [+ \textcolor{darkgreen}{Dictionary} \textcolor{blue}{get\_last\_load\_time} (checkpoint =
		'')]: Renvoie le temps du dernier chargement \\éffectué pour un point de sauvegarde donné. Cette
		méthode exploite la clé \textit{\textcolor {gray}{\_\_last\_load\_time\_\_}} pour accomplir sa
		tâche. \textbf{Notez qu'en cas d'échec, le temps du dernier chargement éffectué au niveau du fichier
		de sauvegarde sera renvoyé}. Le dictionaire renvoyé par cette méthode contient les clés suivantes:
		\textit{\textcolor{gray}{hour, minute}} et \textit{\textcolor{gray}{second}}.
		\begin{itemize}
			\item [>> \textbf{\textcolor{darkgreen}{String} checkpoint}:] Contient le nom du point de 
			sauvegarde à ciblé. Par défaut, le point de \\sauvegarde actif est utilisé pour faire le
			traitement demandé.\\
		\end{itemize}
	\end{description}
	% ImageTexture get_screen_capture () method description.
	\begin{description}
		\item [+ \textcolor{darkgreen}{ImageTexture} \textcolor{blue}{get\_screen\_capture} (checkpoint =
		''):] Renvoie la capture d'écran générée pour un point de sauvegarde donné lors d'une sauvegarde.
		Cette méthode exploite la clé \textit{\textcolor {gray}{\\\_\_game\_screenshot\_\_}} pour accomplir
		sa tâche.
		\begin{itemize}
			\item [>> \textbf{\textcolor{darkgreen}{String} checkpoint}:] Contient le nom du point de 
			sauvegarde à ciblé. Par défaut, le point de \\sauvegarde actif est utilisé pour faire le
			traitement demandé.\\
		\end{itemize}
	\end{description}
	% Void set_data () method description.
	\begin{description}
		\item [+ \textcolor{red}{void} \textcolor{blue}{set\_data} (key, value, checkpoint = '', delay = 
		0.0):] Mets à jour le gestionnaire de \\données.
		\begin{itemize}
			\item [>> \textbf{\textcolor{darkgreen}{String} key}:] Contient le nom de la clé qui va servir
			d'identification à la valeur à insérée. \textbf{Evitez de mettre des espaces, si vous tenez à
			récupéré valeur contenu dans votre clé}. Si votre clé n'existe pas, elle sera automatique créé.
			\item [>> \textbf{\textcolor{darkgreen}{Variant} value}:] Contient la valeur de la clé à
			ajoutée ou à modifiée.
			\item [>> \textbf{\textcolor{darkgreen}{String} checkpoint}:] Contient le nom du point de
			sauvegarde à ciblé. Si le point de sauvegarde n'est pas définit dans le gestionnaire de données, 
			il sera automatiquement créé. Par défaut, le point de sauvegarde actif est pris pour cible.
			\item [>> \textbf{\textcolor{red}{float} delay}:] Quel est le temps mort avant la mise à jour du
			gestionnaire de données ?\\
		\end{itemize}
	\end{description}
	% Variant get_data () method description.
	\begin{description}
		\item [+ \textcolor{darkgreen}{Variant} \textcolor{blue}{get\_data} (key, default = null, checkpoint
		= ''):] Renvoie la valeur correspondante à \textit{\textcolor{gray}{key}}.
		\begin{itemize}
			\item [>> \textbf{\textcolor{darkgreen}{String} key}:] Contient le nom de la clé qui va servir
			d'identification à la valeur à récupérée.
			\item [>> \textbf{\textcolor{darkgreen}{Variant} default}:] Quelle valeur va t-on retournée,
			lorsque le nom de la clé ou celui du point de sauvegarde n'est pas définit dans le gestionnaire 
			de données.
			\item [>> \textbf{\textcolor{darkgreen}{String} checkpoint}:] Contient le nom du point de
			sauvegarde à ciblé. Par défaut, le point de sauvegarde actif est pris pour cible.\\
		\end{itemize}
	\end{description}
	% Array get_checkpoints_list () method description.
	\begin{description}
		\item [+ \textcolor{darkgreen}{Array} \textcolor{blue}{get\_checkpoints\_list} ():] Renvoie la liste
		de tous le(s) point(s) de sauvegarde défini(en)t au sein du gestionnaire de données.\\
	\end{description}
	% Bool has_checkpoints () method description.
	\begin{description}
		\item [+ \textcolor{red}{bool} \textcolor{blue}{has\_checkpoints} (checkpoints):] Le(s) point(s) de
		sauvegarde référé(s) est/sont-il(s) \\défini(en)t au sein du gestionaire de données ?
		\begin{itemize}
			\item [>> \textbf{\textcolor{darkgreen}{String} checkpoint}:] Contient le(s) nom(s) de(s)
			point(s) de sauvegarde à ciblé(s).\\
		\end{itemize}
	\end{description}
	% Void destroy_keys () method description.
	\begin{description}
		\item [+ \textcolor{red}{void} \textcolor{blue}{destroy\_keys} (keys = null, checkpoints = null,
		delay = 0.0):] Supprime un ou plusieurs \\identifiant(s) dans un ou plusieurs point(s) de 
		sauvegarde. Notez que la valeur de(s) identifiant(s) sera également supprimée.
		\begin{itemize}
			\item [>> \textbf{\textcolor{darkgreen}{String | PoolStringArray} keys}:] Contient le(s)
			identifiant(s) à supprimé(s). \textbf{Si aucune clé n'a été référée, on assistera à un nétoyage 
			complet de toutes les données au sein du point de sauvegarde en question}.
			\item [>> \textbf{\textcolor{darkgreen}{String | PoolStringArray} checkpoints}:] Contient le(s) 
			nom(s) de(s) point(s) de sauvegarde dans lesquel(s) le(s) identifiant(s) précisé(s) sera(ont)
			supprimé(s). Par défaut, on ciblera le point de sauvegarde actif pour faire le traitement.
			\item [>> \textbf{\textcolor{red}{float} delay}:] Quel est le temps mort avant la mise à jour du
			gestionnaire de données ?\\
		\end{itemize}
	\end{description}
	% Bool has_keys () method description.
	\newpage \begin{description}
		\item [+ \textcolor{red}{bool} \textcolor{blue}{has\_keys} (keys, checkpoints = null):] Détermine si
		un ou plusieurs identifiant(s) sont bel et bien définient dans un ou plusieurs point(s) de
		sauvegarde.
		\begin{itemize}
			\item [>> \textbf{\textcolor{darkgreen}{String | PoolStringArray} keys}:] Contient tous le(s)
			identifiant(s) à cherché(s).
			\item [>> \textbf{\textcolor{darkgreen}{String | PoolStringArray} checkpoints}:] Quel(s) est/
			sont le(s) point(s) de sauvegarde qui \\sera(ont) ciblé(s) ? Par défaut, le point de sauvegarde
			actif sera pris pour cible.\\
		\end{itemize}
	\end{description}
	% Dictionary | String get_game_data () method description.
	\begin{description}
		\item [+ \textcolor{darkgreen}{Dictionary | String} \textcolor{blue}{get\_game\_data} (json =
		true):] Renvoie toutes les informations sur le \\gestionnaire des données du jeu.
		\begin{itemize}
			\item [>> \textbf{\textcolor{red}{bool} json}:] Voulez-vous renvoyer le résultat sous le format 
			json ?\\
		\end{itemize}
	\end{description}
	% Dictionary | String get_checkpoint_data () method description.
	\begin{description}
		\item [+ \textcolor{darkgreen}{Dictionary | String} \textcolor{blue}{get\_checkpoint\_data}
		(checkpoint = '', json = true):] Renvoie toutes les \\données présentes au sein d'un point de
		sauvegarde donné.
		\begin{itemize}
			\item [>> \textbf{\textcolor{darkgreen}{String} checkpoint}:] Quel point de sauvegarde voulez-
			vous ciblé ? Par défaut, le point de \\sauvegarde actif sera utiliser pour faire le traitement.
			\item [>> \textbf{\textcolor{red}{bool} json}:] Voulez-vous renvoyer le résultat sous le format 
			json ?\\
		\end{itemize}
	\end{description}
	% Void destroy_game_data () method description.
	\begin{description}
		\item [+ \textcolor{red}{void} \textcolor{blue}{destroy\_game\_data} (on\_disc = false, delay =
		0.0):] Supprime toutes les données du jeu au sein du gestionnaire. Faites très attention lorsque
		vous utilisez cette fonction, car il n'y a pas de retour arrière après une telle opération.
		\begin{itemize}
			\item [>> \textbf{\textcolor{red}{bool} on\_disc}:] Voulez-vous supprimer physiquement le
			fichier de sauvegarde du disque dure ?
			\item [>> \textbf{\textcolor{red}{float} delay}:] Quel est le temps mort avant le nétoyage ?\\
		\end{itemize}
	\end{description}
	% Void destroy_checkpoints () method description.
	\begin{description}
		\item [+ \textcolor{red}{void} \textcolor{blue}{destroy\_checkpoints} (checkpoints = null, delay =
		0.0):] Supprime un ou plusieurs point(s) de sauvegarde.
		\begin{itemize}
			\item [>> \textbf{\textcolor{darkgreen}{String | PoolStringArray} checkpoints}:] Quel(s) est/
			sont le(s) nom(s) de(s) point(s) de \\sauvegarde à détruire ? Par défaut, le point de sauvegarde
			actif est pris pour cible.
			\item [>> \textbf{\textcolor{red}{float} delay}:] Quel est le temps mort avant le nétoyage ?\\
		\end{itemize}
	\end{description}
	% Void set_checkpoint_data () method description.
	\begin{description}
		\item [+ \textcolor{red}{void} \textcolor{blue}{set\_checkpoint\_data} (data, checkpoint = '', delay
		= 0.0):] Mets à jour les données au sein d'un point de sauvegarde donné.
		\begin{itemize}
			\item [>> \textbf{\textcolor{darkgreen}{Dictionary} data}:] Quelles sont différentes données qui
			constituent le point de sauvegarde en question. \textbf{Evitez de mettre des espaces dans vos
			clés si vous tenez à les récupérées}.
			\item [>> \textbf{\textcolor{darkgreen}{String} checkpoint}:] Quel est le nom du point de
			sauvegarde à modifié ? Notez que si ce dernier n'est pas définit dans le gestionnaire, il sera
			automatiquement créé. Par défaut, le point de sauvegarde actif sera pris pour cible.
			\item [>> \textbf{\textcolor{red}{float} delay}:] Quel est le temps mort avant la mise à jour du
			gestionnaire des données ?\\
		\end{itemize}
	\end{description}
	% Int get_total_data_count () method description.
	\begin{description}
		\item [+ \textcolor{red}{int} \textcolor{blue}{get\_total\_data\_count} ():] Renvoie le nombre total
		de données définient au sein du \\gestionnaire des données du jeu.
	\end{description}

	% SaveLoad signals definition.
	\newpage \section{Les événements disponibles}
	% before_save () signal description.
	\begin{description}
		\item [+ \textcolor{blue}{before\_save}:] Signal appelé avant la sauvegarde des données du jeu.
	\end{description}
	% before_load () signal description.
	\begin{description}
		\item [+ \textcolor{blue}{before\_load}:] Signal appelé avant le chargement des données du jeu.
	\end{description}
	% game_data_changed () signal description.
	\begin{description}
		\item [+ \textcolor{blue}{game\_data\_changed}:] Signal appelé lorsque la structure des données
		définient au sein du \\gestionnaire a changé.\\
	\end{description}
	% before_destroy () signal description.
	\begin{description}
		\item [+ \textcolor{blue}{before\_destroy} (data):] Signal appelé avant la destruction des données 
		du jeu.
		\begin{itemize}
			\item [>> \textbf{\textcolor{darkgreen}{Dictionary | Array} data}:] Contient des informations
			sur les différent(s) élément(s) à détruire.\\
		\end{itemize}
	\end{description}
	% after_save () signal description.
	\begin{description}
		\item [+ \textcolor{blue}{after\_save} (datum\_count):] Signal appelé après la sauvegarde des
		données du jeu.
		\begin{itemize}
			\item [>> \textbf{\textcolor{red}{int} datum\_count}:] Contient le nombre total de donnée(s)
			sauvegardée(s).\\
		\end{itemize}
	\end{description}
	% after_load () signal description.
	\begin{description}
		\item [+ \textcolor{blue}{after\_load} (datum\_count):] Signal appelé après le chargement des
		données du jeu.
		\begin{itemize}
			\item [>> \textbf{\textcolor{red}{int} datum\_count}:] Contient le nombre total de donnée(s)
			chargée(s).\\
		\end{itemize}
	\end{description}
	% after_destroy () signal description.
	\begin{description}
		\item [+ \textcolor{blue}{after\_destroy} (data):] Signal appelé après la destruction des données du
		jeu. Cet événement renvoie un dictionaire contenant les clés suivantes:
		\begin{itemize}
			\item [>> \textbf{\textcolor{darkgreen}{Array} checkpoints}:] Contient la liste de tous les
			points de sauvegarde supprimés.
			\item [>> \textbf{\textcolor{darkgreen}{Array} keys}:] Contient la liste de toutes les clés
			supprimées.\\
		\end{itemize}
	\end{description}
	% before_update () signal description.
	\begin{description}
		\item [+ \textcolor{blue}{before\_update} (data):] Signal appelé avant la mise à jour du
		gestionnaire des données (à chaque fois qu'on ajoute ou modifie la valeur d'une clé). Cet événement
		renvoie un dictionaire contenant les clés suivantes:
		\begin{itemize}
			\item [>> \textbf{\textcolor{darkgreen}{String} key}:] Contient le nom de la clé à modifiée ou
			insérée dans le gestionnaire de données.
			\item [>> \textbf{\textcolor{darkgreen}{String} value}:] Contient la valeur de la clé à modifiée
			insérée dans le gestionnaire de données.
			\item [>> \textbf{\textcolor{darkgreen}{String} checkpoint}:] Contient le nom du point de
			sauvegarde pris pour cible.\\
		\end{itemize}
	\end{description}
	% after_update () signal description.
	\begin{description}
		\item [+ \textcolor{blue}{after\_update} (data):] Signal appelé après la mise à jour du gestionnaire 
		des données (à chaque fois qu'on ajoute ou modifie la valeur d'une clé). Cet événement renvoie un
		dictionaire contenant les clés suivantes:
		\begin{itemize}
			\item [>> \textbf{\textcolor{darkgreen}{String} key}:] Contient le nom de la clé modifiée ou
			insérée dans le gestionnaire de données.
			\item [>> \textbf{\textcolor{darkgreen}{String} value}:] Contient la valeur de la clé modifiée
			ou insérée dans le gestionnaire de données.
			\item [>> \textbf{\textcolor{darkgreen}{String} checkpoint}:] Contient le nom du point de
			sauvegarde pris pour cible.\\
		\end{itemize}
	\end{description}
	% file_cant_open () signal description.
	\begin{description}
		\item [+ \textcolor{blue}{file\_cant\_open} (data):] Signal déclenché lorsqu'on a du mal à ouvrir
		le fichier de sauvegarde ou que son accès à été refusé. Cet événement renvoie un dictionaire
		contenant les clés suivantes:
		\begin{itemize}
			\item [>> \textbf{\textcolor{darkgreen}{String} path}:] Contient le chemin pointant vers le
			fichier de sauvegarde.
			\item [>> \textbf{\textcolor{red}{int} type}:] Contient le type de l'erreur déclenché.
			\item [>> \textbf{\textcolor{darkgreen}{String} message}:] Contient le méssage renvoyé par 
			l'erreur déclenché.\\
		\end{itemize}
	\end{description}
	% file_not_found () signal description.
	\begin{description}
		\item [+ \textcolor{blue}{file\_not\_found} (data):] Signal déclenché lorsque le fichier de 
		sauvegarde n'est pas définit. Cet événement renvoie un dictionaire contenant les clés suivantes:
		\begin{itemize}
			\item [>> \textbf{\textcolor{darkgreen}{String} path}:] Contient le chemin pointant vers le
			fichier de sauvegarde.
			\item [>> \textbf{\textcolor{red}{int} type}:] Contient le type de l'erreur déclenché.
			\item [>> \textbf{\textcolor{darkgreen}{String} message}:] Contient le méssage renvoyé par 
			l'erreur déclenché.\\
		\end{itemize}
	\end{description}
	% file_corrupted () signal description.
	\begin{description}
		\item [+ \textcolor{blue}{file\_corrupted} (data):] Signal déclenché lorsque le fichier de
		sauvegarde a été corromput de \\l'extérieur. Cet événement renvoie un dictionaire contenant les 
		clés suivantes:
		\begin{itemize}
			\item [>> \textbf{\textcolor{darkgreen}{String} path}:] Contient le chemin pointant vers le
			fichier de sauvegarde.
			\item [>> \textbf{\textcolor{red}{int} type}:] Contient le type de l'erreur déclenché.
			\item [>> \textbf{\textcolor{darkgreen}{String} message}:] Contient le méssage renvoyé par 
			l'erreur déclenché.\\
		\end{itemize}
	\end{description}
	% file_saving event description.
	\begin{description}
		\item [+ \textcolor{blue}{file\_saving} (data):] Signal déclenché pendant qu'on sauvegarde les
		données du jeu dans le fichier de sauvegarde. Cet événement renvoie un dictionaire contenant les
		clés suivantes:
		\begin{itemize}
			\item [>> \textbf{\textcolor{darkgreen}{String} path}:] Contient le chemin pointant vers le
			fichier de sauvegarde.
			\item [>> \textbf{\textcolor{red}{int} progress}:] Contient la progression actuelle en 
			pourcentage de la sauvegarde.\\
		\end{itemize}
	\end{description}
	% file_loading event description.
	\begin{description}
		\item [+ \textcolor{blue}{file\_loading} (data):] Signal déclenché pendant que le fichier de
		sauvegarde est en cours de \\chargement. Cet événement renvoie un dictionaire contenant les clés 
		suivantes:
		\begin{itemize}
			\item [>> \textbf{\textcolor{darkgreen}{String} path}:] Contient le chemin pointant vers le
			fichier de sauvegarde.
			\item [>> \textbf{\textcolor{red}{bool} is\_over}:] Est-ce que toutes les données du fichier de
			sauvegarde ont-elles été \\complètement chargées ?
			\item [>> \textbf{\textcolor{red}{int} progress}:] Contient le nombre total de donnée(s) déjà
			chargée(s) en mémoire.\\
		\end{itemize}
	\end{description}
	% game_time_changed event description.
	\begin{description}
		\item [+ \textcolor{blue}{game\_time\_changed} (time):] Signal déclenché à chaque fois que le temps 
		écoulé depuis le \\démarrage du jeu, évolue.
		\begin{itemize}
			\item [>> \textbf{\textcolor{red}{int} time}:] Contient le temps actuellement écoulé en seconde.
			\\
		\end{itemize}
	\end{description}
\end{document}