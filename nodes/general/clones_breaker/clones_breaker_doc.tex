% Document type and package imports.
\documentclass[a4paper, 11pt]{article}
\usepackage[utf8]{inputenc}
\usepackage[T1]{fontenc}
\usepackage[french]{babel}
\usepackage{charter}
\usepackage[top = 2cm, bottom = 2cm, left = 1cm, right = 1cm]{geometry}
\usepackage{setspace}
\usepackage{color}
\usepackage{xcolor}
\usepackage{hyperref}
\usepackage{tocloft}

% Preanblue.
\onehalfspacing
\definecolor{gray}{rgb}{0.4, 0.4, 0.4}
\definecolor{silver}{rgb}{0.95, 0.95, 0.95}
\renewcommand{\thesection}{\Roman{section} --}
\definecolor{darkgreen}{HTML}{1E8C15}
\cftsetindents{section}{1em}{2.5em}
\hypersetup {colorlinks=true, linkcolor=blue, urlcolor=blue, pdftitle={ClonesBreaker module doc}}

% The start of the article.
\begin{document}
	% Change document background to silver color.
	\pagecolor{silver}
	% ClonesBreaker module description.
	\huge{\hspace{13cm}\textit{\textbf{\textcolor{darkgreen}{ClonesBreakerFx}}}}\large{} \tableofcontents
	\newpage
	% ClonesBreaker module definition.
	\section{Définition}
	\textcolor{darkgreen}{\textbf{ClonesBreakerFx}} est un module qui supprime les doublons d'un objet dans
	une scène grâce à son identifiant.\\
	\textcolor{red}{\textbf{NB}:} Notez que ce module est compatible à un jeu 2D, 3D et n'est pas
	sauvegardable.

	% ClonesBreaker properties definition.
	\section{Les propriétés disponibles}
	% Sources property.
	\textbf{+ \textcolor{darkgreen}{Array} Sources:} Tableau de dictionnaires contenant toutes les 
	différentes configurations sur chaque doublon. Les dictionnaires issus de ce tableau supportent les clés
	suivantes:\\
	\begin{itemize}
		\item[>> \textbf{\textcolor{darkgreen}{String | NodePath} reference}:] Quelle est la porté des 
		destructions ? Ce champs contient le noeud à partir duquel les clones seront détruits.\\
		\item[>> \textbf{\textcolor{darkgreen}{String} id}:] Contient l'identifiant à recherché. 
		L'utilisation de cette clé est obligatoire.\\
		\item[>> \textbf{\textcolor{red}{bool} partial = \textcolor{red}{false}}:] Définit si les doublons 
		du noeud en question seront détruit soit par la \\méthode \textcolor{blue}{\textit{queue\_free ()}}, 
		soit par la propriété \textcolor{gray}{\textit{visible}}.\\
		\item[>> \textbf{\textcolor{red}{float} timeout = \textcolor{blue}{0.0}}:] Quel est le délai avant 
		le démarrage du processus de destruction ?\\
		\item[>> \textbf{\textcolor{red}{float} delay = \textcolor{blue}{0.0}}:] Quel est le délai avant la 
		destruction de chaque clone dans l'arbre de la scène en question ?\\
		\item[>> \textbf{\textcolor{red}{int} search = \textcolor{blue}{3}}:] Contient le moyen à utilisé 
		pour chercher les clones à détruire. Les valeurs \\possibles sont:
		\begin{itemize}
			\item[-> \textbf{\textcolor{gray}{MegaAseets.NodeProperty.NAME} ou \textcolor{blue}{0}}:] 
			Ciblage par nom.
			\item[-> \textbf{\textcolor{gray}{MegaAseets.NodeProperty.GROUP} ou \textcolor{blue}{1}}:] 
			Ciblage par groupe.
			\item[-> \textbf{\textcolor{gray}{MegaAseets.NodeProperty.TYPE} ou \textcolor{blue}{2}}:] 
			Ciblage par type.
			\item[-> \textbf{\textcolor{gray}{MegaAseets.NodeProperty.ANY} ou \textcolor{blue}{3}}:] Ciblage 
			sur n'importe quel type.\\
		\end{itemize}
	\end{itemize}
	\textcolor{red}{\textbf{NB}:} Les répétitions au niveau des identifiants ne sont pas tolérées.

	% ClonesBreaker methods definition.
	\section{Les méthodes disponibles}
	% Void destroy_clones () method description.
	\begin{description}
		\item [+ \textcolor{red}{void} \textcolor{blue}{destroy\_clones} (delay = 0.0):] Lance le procéssus 
		de destruction des clones.
		\begin{itemize}
			\item [>> \textbf{\textcolor{red}{float} delay}:] Quel est le temps mort avant la destruction 
			des doublons ?\\
		\end{itemize}
	\end{description}

	% ClonesBreaker signals definition.
	\newpage \section{Les événements disponibles}
	% before_destroy_clone () signal description.
	\begin{description}
		\item [+ \textcolor{blue}{before\_destroy\_clone} (data):] Signal déclenché avant la destruction 
		d'un clone. Cet événement renvoie un dictionnaire contenant les clé suivantes:
		\begin{itemize}
			\item [>> \textbf{\textcolor{darkgreen}{Node} node}:] Contient le noeud ayant déclenché
			l'événement.
			\item [>> \textbf{\textcolor{darkgreen}{Node} clone}:] Contient le noeud (clone) à détruire.\\
		\end{itemize}
	\end{description}
	% after_destroy_clone () signal description.
	\begin{description}
		\item [+ \textcolor{blue}{after\_destroy\_clone} (data):] Signal déclenché après la destruction d'un 
		clone. Cet événement renvoie un dictionnaire contenant les clés suivantes:
		\begin{itemize}
			\item [>> \textbf{\textcolor{darkgreen}{Node} node}:] Contient le noeud ayant déclenché 
			l'événement.
			\item [>> \textbf{\textcolor{darkgreen}{Node} clone}:] Contient le noeud (clone) après 
			désactivation de sa visibilité. Cette valeur est \\renvoyée uniquement si la destruction est 
			partielle.\\
		\end{itemize}
	\end{description}
	% before_destroy_clones () signal description.
	\begin{description}
		\item [+ \textcolor{blue}{before\_destroy\_clones} (node):] Signal déclenché avant la destruction de 
		tous les clones.
		\begin{itemize}
			\item [>> \textbf{\textcolor{darkgreen}{Node} node}:] Contient le noeud ayant déclenché 
			l'événement.\\
		\end{itemize}
	\end{description}
	% after_destroy_clones () signal description.
	\begin{description}
		\item [+ \textcolor{blue}{after\_destroy\_clones} (node):] Signal déclenché après la destruction 
		immédiate de tous les clones. 
		\begin{itemize}
			\item [>> \textbf{\textcolor{darkgreen}{Node} node}:] Contient le noeud ayant déclenché 
			l'événement.\\
		\end{itemize}
	\end{description}
\end{document}