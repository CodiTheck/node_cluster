% Document type and package imports.
\documentclass[a4paper, 11pt]{article}
\usepackage[utf8]{inputenc}
\usepackage[T1]{fontenc}
\usepackage[french]{babel}
\usepackage{charter}
\usepackage{hyperref}
\usepackage{tocloft}
\usepackage[top = 2cm, bottom = 2cm, left = 1cm, right = 1cm]{geometry}
\usepackage{setspace}
\usepackage{color}
\usepackage{xcolor}

% Preanblue.
\onehalfspacing
\definecolor{gray}{rgb}{0.4, 0.4, 0.4}
\definecolor{silver}{rgb}{0.95, 0.95, 0.95}
\renewcommand{\thesection}{\Roman{section} --}
\definecolor{darkgreen}{HTML}{1E8C15}
\cftsetindents{section}{1em}{2.5em}
\hypersetup {colorlinks=true, linkcolor=blue, urlcolor=blue, pdftitle={Animator module doc}}

% The start of the article.
\begin{document}
	% Change document background to silver color.
	\pagecolor{silver}
	% Animator module description.
	\huge{\hspace{14.5cm}\textit{\textbf{\textcolor{darkgreen}{AnimatorFx}}}}\large{} \tableofcontents
	\newpage
	% Animator module definition.
	\section{Définition}
	\textcolor{darkgreen}{\textbf{AnimatorFx}} est un module conçut pour éffectuer des animations sur la 
	valeurs valeurs des propriétés de n'importe quel noeud du moteur. Cependant, on note des limites par 
	rapport à certains types de données. Ce module se base sur le noeud
	\href{https://docs.godotengine.org/en/stable/classes/class_tween.html}{\textit{\textcolor{darkgreen}
	{Tween}}} pour exécuter ses animations.\\
	\textcolor{red}{\textbf{NB}:} Ce module est compatible à un jeu 2D, 3D et n'est pas sauvegardable.

	% Animator properties definition.
	\section{Les propriétés disponibles}
	% Direction property.
	\textbf{+ \textcolor{red}{int} \hypertarget{direction}{Direction} = \textcolor{blue}{0}:} Contient la 
	direction de l'animation globale des éléments du module. Les valeurs possibles sont:
	\begin{itemize}
		\item[-> \textbf{\textcolor{gray}{MegaAssets.Orientation.NORMAL} ou \textcolor{blue}{0}}:] Animation 
		en sens normale.
		\item[-> \textbf{\textcolor{gray}{MegaAssets.Orientation.REVERSED} ou \textcolor{blue}{1}}:]
		Animation en sens renversé.
		\item[-> \textbf{\textcolor{gray}{MegaAssets.Orientation.RANDOM} ou \textcolor{blue}{2}}:] Animation 
		aléatoire.
		\item[-> \textbf{\textcolor{gray}{MegaAssets.Orientation.ALTERNATE} ou \textcolor{blue}{3}}:] 
		Animation en sens alterné.
		\item[-> \textbf{\textcolor{gray}{MegaAssets.Orientation.ALTERNATE\_REVERSE} ou \textcolor{blue}
		{4}}:] Animation en sens alterné renversé.\\
	\end{itemize}
	% Sync property.
	\textbf{+ \textcolor{red}{bool} Sync = \textcolor{red}{false}:} Animation globale des éléments 
	s'éffectuera t-elle de façon synchrônée ?\\\\
	% Count property.
	\textbf{+ \textcolor{red}{int} Count = \textcolor{blue}{1}:} Combien de fois l'animation globale des 
	éléments s'éffectuera ?\\\\
	\textbf{+ \textcolor{red}{int} Action = \textcolor{blue}{0}:} Contient les différentes actions possibles
	qu'on peut éffectuées sur les animations. Les valeurs possibles sont:
	\begin{itemize}
		\item[-> \textbf{\textcolor{gray}{MegaAssets.MediaState.NONE} ou \textcolor{blue}{0}}:] Aucune 
		action ne sera éffectuée.
		\item[-> \textbf{\textcolor{gray}{MegaAssets.MediaState.PLAY} ou \textcolor{blue}{1}}:] Joue toutes
		les animations disponibles sur le module.
		\item[-> \textbf{\textcolor{gray}{MegaAssets.MediaState.PAUSE} ou \textcolor{blue}{2}}:] Suspend 
		toutes les animations en cours de lecture.
		\item[-> \textbf{\textcolor{gray}{MegaAssets.MediaState.STOP} ou \textcolor{blue}{3}}:] Stop toutes 
		les animations en cours de lecture.\\
	\end{itemize}
	% ListenModule property.
	\textbf{+ \textcolor{darkgreen}{NodePath} ListenModule:} Contient l'instance d'un module du même trampe
	de ce module. \\Exemple: \textit{\textcolor{darkgreen}{AnimatorFx, MenuFx, AudioTrackFx}} etc... En
	d'autres termes, contient la référence d'un module possèdant également un curseur. Le but de cette
	fonctionnalité est de synchrôniser la valeur du curseur du module référé à celle de ce module.\\\\
	% Animations property.
	\textbf{+ \textcolor{darkgreen}{Array} Animations:} Tableau de dictionnaires contenant toutes les 
	différentes configurations sur chaque animation prise en charge par le développeur. Les dictionnaires 
	issus de ce tableau supportent les clés suivantes:\\
	\begin{itemize}
		\item[>> \textbf{\textcolor{red}{float} duration = \textcolor{blue}{1.0}}:] Quelle est la durée de 
		l'animation à éffectuée ?\\
		\item[>> \textbf{\textcolor{red}{float} speed = \textcolor{blue}{1.0}}:] Contrôle la vitesse de 
		l'animation.\\
		\item[>> \textbf{\textcolor{red}{float} seek = \textcolor{blue}{0.0}}:] Contrôle le temps de départ
		de l'animation.\\
		\item[>> \textbf{\textcolor{red}{float} timeout = \textcolor{blue}{1.0}}:] Quel est le délai avant 
		le déclenchement de l'animation en question ?\\
		\item[>> \textbf{\textcolor{red}{int} repeat = \textcolor{blue}{1}}:] Combien de fois, l'animation
		sera répéter ? Cette option ne s'active que lorsque sa valeur est supérieur à \textcolor{blue}
		{\textbf{0}}. Dans le cas contraire, l'animation n'est pas exécutée. Une valeur négative, entraînera
		une animation infinie.\\
		\item[>> \textbf{\textcolor{red}{int} direction = \textcolor{blue}{0}}:] Contrôle le sens 
		d'exécution de l'animation. Les valeurs possibles sont:
		\begin{itemize}
			\item[-> \textbf{\textcolor{gray}{MegaAssets.Orientation.NORMAL} ou \textcolor{blue}{0}}:]
			Animation en sens normale.
			\item[-> \textbf{\textcolor{gray}{MegaAssets.Orientation.REVERSED} ou \textcolor{blue}{1}}:]
			Animation en sens renversé.
			\item[-> \textbf{\textcolor{gray}{MegaAssets.Orientation.ALTERNATE} ou \textcolor{blue}{3}}:]
			Animation en sens alterné.
			\item[-> \textbf{\textcolor{gray}{MegaAssets.Orientation.ALTERNATE\_REVERSE} ou \textcolor{blue}
			{4}}:] Animation en sens alterné renversé.
			\item[-> \textbf{\textcolor{gray}{MegaAssets.Orientation.PING\_PONG} ou \textcolor{blue}{5}}:]
			Animation booléenne.\\
		\end{itemize}
		\item[>> \textbf{\textcolor{red}{int} easing = \textcolor{blue}{2}}:] Contrôle l'assouplissement de
		l'animation. Les valeurs possibles sont celles de \\Godot.\\
		\item[>> \textbf{\textcolor{red}{int} type = \textcolor{blue}{0}}:] Contrôle le type de transition à 
		éffectuée. Les valeurs possibles sont celles de Godot.\\
		\item[>> \textbf{\textcolor{darkgreen}{String | NodePath} target}:] Quel est le chemin d'accès du
		noeud sera qui victime de l'animation en question ?\\
		\item[>> \textbf{\textcolor{darkgreen}{String | NodePath} action}:] Quel est le nom de la propriété
		ou méthode qui sera victime de l'animation tout en sachant que cette dernière est belle et bien
		définit au sein du noeud référé. Notez que si votre animation touche une méthode alors, les
		caractères \textit{\textcolor{gray}{()}} seront misent à la fin du nom donné à la méthode en
		question. Dans ce cas les valeurs possibles au sein de l'intervalle [\textit{\textcolor{gray}
		{ivalue}}; \textit{\textcolor{gray}{fvalue}}] \\seront passées en paramètre à la méthode ciblée.\\
		\item[>> \textbf{\textcolor{darkgreen}{Variant} ivalue}:] Quelle est la valeur initiale de la 
		propriété ciblée ? Si vous donnez le chemin \\d'accès d'un noeud, la valeur initiale sera égale à
		celle définit dans le noeud référé tout en sachant que la propriété ciblée est également définie 
		dans ce dernier avec le même type de donnée. \\Notez que vous avez également la possibilité de 
		donner un dictionaire supportant les clés: \textit{\textcolor{gray}{source, action, value et
		params}} dont la description se trouve dans les bases du framework au niveau de la clé
		\textit{\textcolor{gray}{actions}} du champ \textit{\textcolor{gray}{EventBindings}}.\\
		\item[>> \textbf{\textcolor{darkgreen}{Variant} fvalue}:] Quelle est la valeur finale de la 
		propriété ciblée ? L'utilisation de cette clé est la même que la précédente.\\
		\item[>> \textbf{\textcolor{red}{bool} forwards = \textcolor{red}{false}}:] Devons-nous rénitialiser 
		les valeurs de la ou des propriété(s) victime de l'animation après son exécution ?\\
		\newpage \item[>> \textbf{\textcolor{darkgreen}{Array | Dictionary} started}:] Signal déclenché à 
		chaque fois qu'on démarre immédiatement cette animation. Cette clé exécute les différentes actions 
		données à son déclenchement. Pour soumettre les actions à exécutées référez vous à la méthode
		utilisée au niveau de la clé \textit{\textcolor{gray}{actions}} de la propriété
		\textit{\textcolor{gray}{EventsBindings}} dans les bases du framework.\\
		\item[>> \textbf{\textcolor{darkgreen}{Array | Dictionary} finished}:] Signal déclenché à chaque 
		fois que l'animation a terminée son \\exécution. Cette clé exécute les différentes actions données à 
		son déclenchement. Pour soumettre les actions à exécutées référez vous à la méthode utilisée au
		niveau de la clé \textit{\textcolor{gray}{actions}} de la propriété \textit{\textcolor{gray}
		{EventsBindings}} dans les bases du framework.\\
		\item[>> \textbf{\textcolor{darkgreen}{Array | Dictionary} playing}:] Signal déclenché pendant que
		l'animation est en cours d'exécution. Cette clé exécute les différentes actions données
		à son déclenchement. Pour soumettre les actions à exécutées référez vous à la méthode utilisée au
		niveau de la clé \textit{\textcolor{gray}{actions}} de la propriété \textit{\textcolor{gray}
		{\\EventsBindings}} dans les bases du framework.\\
	\end{itemize}
	\textcolor{red}{\textbf{NB}:} Gardez à l'esprit que ce module se sert d'un curseur pour sélectionner les 
	animations définient par le développeur. Ce curseur n'est rien d'autre que l'index de position de chaque 
	configuration \\d'animation. Par défaut, sa valeur est \textbf{\textcolor{blue}{0}} lorsqu'il y a une ou 
	plusieurs animation(s) configurée(s) sur le module et \textbf{\textcolor{blue}{-1}} si aucune animation 
	n'est disponible sur le module. Avant de continuer, sachez que nous ne tolérons pas les répétitions sur 
	le nom des propriétés ou méthodes que vous utiliserez.

	% Animator methods definition.
	\section{Les méthodes disponibles}
	% Void play () method description.
	\begin{description}
		\item [+ \textcolor{red}{void} \textcolor{blue}{play} (id = null, config = \{\}, interval = 0.0):]
		Joue une ou plusieurs animation(s). Si \\aucun identifiant n'a été référé, la valeur du curseur sera
		utiliser pour éffectuer le traitement demandé.
		\begin{itemize}
			\item [>> \textbf{\textcolor{red}{int} | \textcolor{darkgreen}{PoolIntArray} id}:] Quel(s) est/
			sont le(s) identifiant(s) de(s) animation(s) à exécutée(s) ?
			\item [>> \textbf{\textcolor{darkgreen}{Dictionary | Array} config}:] Voulez-vous changer la 
			valeur de certaines clés de(s) animation(s) avant son/leur exécution ? Si vous donnez un 
			tableau, alors il ne devra que contenir des \\dictionaires.
			\item [>> \textbf{\textcolor{red}{float} interval}:] Quel est le temps mort avant l'exécution de 
			chaque animation ?\\
		\end{itemize}
	\end{description}
	% Void pause () method description.
	\begin{description}
		\item [+ \textcolor{red}{void} \textcolor{blue}{pause} (id = null, interval = 0.0):] Suspend
		l'exécution d'une ou de plusieurs animation(s). Si aucun identifiant n'a été référé, la valeur du
		curseur sera utiliser pour éffectuer le traitement demandé.
		\begin{itemize}
			\item [>> \textbf{\textcolor{red}{int} | \textcolor{darkgreen}{PoolIntArray} id}:] Quel(s) est/
			sont le(s) identifiant(s) de(s) animation(s) à suspendre ?
			\item [>> \textbf{\textcolor{red}{float} interval}:] Quel est le temps mort avant la suspension 
			de chaque animation ?\\
		\end{itemize}
	\end{description}
	% Void stop () method description.
	\begin{description}
		\item [+ \textcolor{red}{void} \textcolor{blue}{stop} (id = null, interval = 0.0):] Arrète 
		l'exécution d'une ou de plusieurs animation(s) à partir de sa/leur position. Si aucun identifiant
		n'a été référé, la valeur du curseur sera utiliser pour éffectuer le traitement demandé.
		\begin{itemize}
			\item [>> \textbf{\textcolor{red}{int} | \textcolor{darkgreen}{PoolIntArray} id}:] Quel(s) est/
			sont le(s) identifiant(s) de(s) animation(s) à arrètée(s) ?
			\item [>> \textbf{\textcolor{red}{float} interval}:] Quel est le temps mort avant l'arrèt de 
			chaque animation ?\\
		\end{itemize}
	\end{description}
	% Void preview () method description.
	\begin{description}
		\item [+ \textcolor{red}{void} \textcolor{blue}{preview} (config = \{\}, delay = 0.0):] Exécute
		l'animation immédiatement précédente de celle indexée par le curseur du module.
		\begin{itemize}
			\item [>> \textbf{\textcolor{darkgreen}{Dictionary} config}:] Voulez-vous changer la valeur de
			certaines clés de l'animation avant son exécution ?
			\item [>> \textbf{\textcolor{red}{float} delay}:] Quel est le temps mort avant l'exécution de 
			l'animation en question ?\\
		\end{itemize}
	\end{description}
	% Void next () method description.
	\begin{description}
		\item [+ \textcolor{red}{void} \textcolor{blue}{next} (config = \{\}, delay = 0.0):] Exécute 
		l'animation suivant celle indexée par le curseur du module.
		\begin{itemize}
			\item [>> \textbf{\textcolor{darkgreen}{Dictionary} config}:] Voulez-vous changer la valeur de
			certaines clés de l'animation avant son exécution ?
			\item [>> \textbf{\textcolor{red}{float} delay}:] Quel est le temps mort avant l'exécution de 
			l'animation en question ?\\
		\end{itemize}
	\end{description}
	% Int get_progress () method description.
	\begin{description}
		\item [+ \textcolor{red}{int} \textcolor{blue}{get\_progress} (id = null):] Retourne la progression
		actuelle d'une animation en cours \\d'exécution. Si aucun identifiant n'a été précisé, celle de 
		l'ensemble des animations sera renvoyer. La valeur de la progression est dans l'intervalle 
		[\textcolor{blue}{0}; \textcolor{blue}{100}].
		\begin{itemize}
			\item [>> \textbf{\textcolor{red}{int} id}:] Quel est l'identifiant de l'animation dont-on 
			souhaite récupérée la progression ?\\
		\end{itemize}
	\end{description}
	% Float get_normalized_progress () method description.
	\begin{description}
		\item [+ \textcolor{red}{float} \textcolor{blue}{get\_normalized\_progress} (id = null):] Retourne 
		la valeur normalisée de la progression \\actuelle d'une animation en cours d'exécution. Si aucun 
		identifiant n'a été précisé, celle de \\l'ensemble des animations sera renvoyer. La valeur de la 
		progression est dans l'intervalle [\textcolor{blue}{0.0}; \textcolor{blue}{1.0}].
		\begin{itemize}
			\item [>> \textbf{\textcolor{red}{int} id}:] Quel est l'identifiant de l'animation dont-on 
			souhaite récupérée la progression ?\\
		\end{itemize}
	\end{description}
	% Float get_eleapsed_time () method description.
	\begin{description}
		\item [+ \textcolor{red}{float} \textcolor{blue}{get\_eleapsed\_time} (id = null):] Retourne le 
		temps écoulé depuis l'exécution d'une \\animation. Si aucun identifiant n'a été précisé, celui de 
		l'ensemble des animations sera renvoyer.
		\begin{itemize}
			\item [>> \textbf{\textcolor{red}{int} id}:] Quel est l'identifiant de l'animation à ciblée ?\\
		\end{itemize}
	\end{description}
	% Int get_state () method description.
	\begin{description}
		\item [+ \textcolor{red}{int} \textcolor{blue}{get\_state} (id = null):] Renvoie l'état 
		d'utilisation d'une animation donnée. Si aucun \\identifiant n'a été précisé, le traitement sera 
		éffectué sur l'ensemble des animations du module. Les valeurs possibles de retour sont:
		\begin{itemize}
			\item[-> \textbf{\textcolor{gray}{MegaAssets.MediaState.NONE} ou \textcolor{blue}{0}}:] Aucun
			traitement ou animation arrètée.
			\item[-> \textbf{\textcolor{gray}{MegaAssets.MediaState.PLAY} ou \textcolor{blue}{1}}:]
			Animation en cours d'exécution.
			\item[-> \textbf{\textcolor{gray}{MegaAssets.MediaState.PAUSE} ou \textcolor{blue}{2}}:]
			Animation suspendu.
			\item[-> \textbf{\textcolor{gray}{MegaAssets.MediaState.LOOP} ou \textcolor{blue}{4}}:]
			Animation en exécution infinie.
		\end{itemize}
		\begin{itemize}
			\item [>> \textbf{\textcolor{red}{int} id}:] Quel est l'identifiant de l'animation à ciblée ?\\
		\end{itemize}
	\end{description}
	% Int get_current_repeatition () method description.
	\newpage \begin{description}
		\item [+ \textcolor{red}{int} \textcolor{blue}{get\_current\_repeatition} (id = null):] Renvoie le 
		nombre actuelle de répétitions éffectuées \\auprès d'une animation donnée. Si aucun identifiant n'a 
		été renseigné, celui de l'ensemble des animations sera renvoyer.
		\begin{itemize}
			\item [>> \textbf{\textcolor{red}{int} id}:] Quel est l'identifiant de l'animation à ciblée ?\\
		\end{itemize}
	\end{description}
	% Int get_cursor () method description.
	\begin{description}
		\item [+ \textcolor{red}{int} \textcolor{blue}{get\_cursor} ():] Renvoie la valeur actuelle du 
		curseur du module.\\
	\end{description}
	% Void set_cursor () method description.
	\begin{description}
		\item [+ \textcolor{red}{void} \textcolor{blue}{set\_cursor} (new\_value):] Change la valeur 
		actuelle du curseur du module.
		\begin{itemize}
			\item [>> \textbf{\textcolor{red}{int} new\_value}:] Quel est la nouvelle valeur du curseur ?\\
		\end{itemize}
	\end{description}
	% Int | PoolIntArray get_active_anim_index () method description.
	\begin{description}
		\item [+ \textcolor{red}{int} | \textcolor{darkgreen}{PoolIntArray} \textcolor{blue}
		{get\_active\_anim\_index} ():] Renvoie le(s) position(s) de la ou des animation(s) en cours 
		d'exécution.\\
	\end{description}
	% Int get_prev_anim_index () method description.
	\begin{description}
		\item [+ \textcolor{red}{int} \textcolor{blue}{get\_prev\_anim\_index} ():] Renvoie la position de 
		l'animation précédement exécutée ou \\générée.\\
	\end{description}
	% Int get_next_anim_index () method description.
	\begin{description}
		\item [+ \textcolor{red}{int} \textcolor{blue}{get\_next\_anim\_index} ():] Renvoie la position du 
		future animation à exécutée. N'appelée cette méthode que si la valeur du champ
		\textit{\hyperlink{direction}{Direction}} est sur \textit{\textcolor{gray}{RANDOM}}.\\
	\end{description}
	% Int get_big_anim_index () method description.
	\begin{description}
		\item [+ \textcolor{red}{int} \textcolor{blue}{get\_big\_anim\_index} ():] Renvoie la position de
		l'animation ayant la plus grande durée \\parmit celles définient par le développeur.\\
	\end{description}
	% Int get_small_anim_index () method description.
	\begin{description}
		\item [+ \textcolor{red}{int} \textcolor{blue}{get\_small\_anim\_index} ():] Renvoie la position de 
		l'animation ayant la plus petite durée \\parmit celles définient par le développeur.\\
	\end{description}
	% Float get_total_duration () method description.
	\begin{description}
		\item [+ \textcolor{red}{float} \textcolor{blue}{get\_total\_duration} ():] Renvoie le temps total
		des animations disponibles sur le module (sans les délai).\\
	\end{description}
	% Float get_total_timeout () method description.
	\begin{description}
		\item [+ \textcolor{red}{float} \textcolor{blue}{get\_total\_timeout} ():] Renvoie la somme des 
		délai définient sur les animations du module.\\
	\end{description}
	% Float get_total_time () method description.
	\begin{description}
		\item [+ \textcolor{red}{float} \textcolor{blue}{get\_total\_time} ():] Renvoie le temps total
		des animations disponibles sur le module (avec les délai).\\
	\end{description}
	% Dictionary get_animations_data () method description.
	\begin{description}
		\item [+ \textcolor{darkgreen}{Dictionary} \textcolor{blue}{get\_animations\_data} (json = false):] 
		Renvoie toutes les données sur les animations disponibles du module.
		\begin{itemize}
			\item [>> \textbf{\textcolor{red}{bool} json}:] Voulez-vous renvoyer les données au format json 
			?\\
		\end{itemize}
	\end{description}

	% Animator signals definition.
	\section{Les événements disponibles}
	% animation_changed event description.
	\begin{description}
		\item [+ \textcolor{blue}{animation\_changed} (data):] Signal déclenché lorsque l'animation en cours 
		d'exécution a \\changée. Notez que cet événement ne s'appel que lorsque le mode de lecture des 
		animations du \\module est synchrône. Il renvoie un dictionaire contenant les clés suivantes:
		\begin{itemize}
			\item [>> \textbf{\textcolor{darkgreen}{Node} node}:] Contient le noeud où cet signal a été 
			émit.
			\item [>> \textbf{\textcolor{red}{int} preview}:] Contient la position de l'animation 
			précédement exécutée.
			\item [>> \textbf{\textcolor{red}{int} current}:] Contient l'index de position de l'animation
			actuellement en cours d'exécution.\\
		\end{itemize}
	\end{description}
	% animation_started event description.
	\begin{description}
		\item [+ \textcolor{blue}{animation\_started} (data):] Signal déclenché à chaque fois qu'on démarre 
		immédiatement une animation. Cet événement renvoie un dictionaire contenant les clés suivantes:
		\begin{itemize}
			\item [>> \textbf{\textcolor{darkgreen}{Node} node}:] Contient le noeud où cet signal a été 
			émit.
			\item [>> \textbf{\textcolor{red}{int} index}:] Contient la position de l'animation en question.
			\\
		\end{itemize}
	\end{description}
	% animation_finished event description.
	\begin{description}
		\item [+ \textcolor{blue}{animation\_finished} (data):] Signal déclenché à chaque fois qu'une 
		animation a terminée son exécution. Cet événement renvoie un dictionaire contenant les clés 
		suivantes:
		\begin{itemize}
			\item [>> \textbf{\textcolor{darkgreen}{Node} node}:] Contient le noeud où cet signal a été 
			émit.
			\item [>> \textbf{\textcolor{red}{int} index}:] Contient la position de l'animation en question.
			\\
		\end{itemize}
	\end{description}
	% animation_playing event description.
	\begin{description}
		\item [+ \textcolor{blue}{animation\_playing} (data):] Signal déclenché pendant qu'une animation est 
		en cours d'exécution. Cet événement renvoie un dictionaire contenant les clés suivantes:
		\begin{itemize}
			\item [>> \textbf{\textcolor{darkgreen}{Node} node}:] Contient le noeud où cet signal a été 
			émit.
			\item [>> \textbf{\textcolor{red}{float} time}:] Contient le temps écoulé depuis l'exécution de
			l'animation.
			\item [>> \textbf{\textcolor{red}{int} progress}:] Contient la progression actuelle de 
			l'animation.
			\item [>> \textbf{\textcolor{red}{float} normalized}:] Contient la progression normalisée de 
			l'animation.
			\item [>> \textbf{\textcolor{red}{int} count}:] Contient le nombre total de répétitions 
			éffectuées depuis la première exécution. Cependant, si l'animation est exécutée à l'infinie, 
			vous aurez une valeur négative.
			\item [>> \textbf{\textcolor{red}{int} index}:] Contient la position de l'animation en question.
			\\
		\end{itemize}
	\end{description}
	% list_started event description.
	\begin{description}
		\item [+ \textcolor{blue}{list\_started} (node):] Signal déclenché à chaque fois qu'une animation 
		globale démarre son \\exécution.
		\begin{itemize}
			\item [>> \textbf{\textcolor{darkgreen}{Node} node}:] Contient le noeud où cet signal a été 
			émit.\\
		\end{itemize}
	\end{description}
	% list_finished event description.
	\begin{description}
		\item [+ \textcolor{blue}{list\_finished} (node):] Signal déclenché à chaque fois qu'une animation 
		globale termine son \\exécution.
		\begin{itemize}
			\item [>> \textbf{\textcolor{darkgreen}{Node} node}:] Contient le noeud où cet signal a été 
			émit.\\
		\end{itemize}
	\end{description}
	% list_playing event description.
	\begin{description}
		\item [+ \textcolor{blue}{list\_playing} (data):] Signal déclenché pendant qu'une animation globale 
		est en cours \\d'exécution. Cet événement renvoie un dictionaire contenant les clés suivantes:
		\begin{itemize}
			\item [>> \textbf{\textcolor{darkgreen}{Node} node}:] Contient le noeud où cet signal a été 
			émit.
			\item [>> \textbf{\textcolor{red}{float} time}:] Contient le temps écoulé depuis l'exécution de
			l'ensemble des animations du \\module.
			\item [>> \textbf{\textcolor{red}{int} progress}:] Contient la progression actuelle de 
			l'ensemble des animations du module.
			\item [>> \textbf{\textcolor{red}{float} normalized}:] Contient la progression normalisée de 
			l'ensemble des animations du module.
			\item [>> \textbf{\textcolor{red}{int} count}:] Contient le nombre total de répétitions 
			éffectuées depuis la première exécution. \\Cependant, si l'animation générale est exécutée à 
			l'infinie, vous aurez une valeur négative.\\
		\end{itemize}
	\end{description}
	% timeout event description.
	\begin{description}
		\item [+ \textcolor{blue}{timeout} (data):] Signal déclenché à chaque fois qu'un temps mort est 
		requis avant exécuter une animation. Cet événement renvoie un dictionaire contenant les clés 
		suivantes:
		\begin{itemize}
			\item [>> \textbf{\textcolor{darkgreen}{Node} node}:] Contient le noeud où cet signal a été 
			émit.
			\item [>> \textbf{\textcolor{red}{int} preview}:] Contient la position de l'animation 
			précédente.
			\item [>> \textbf{\textcolor{red}{int} current}:] Contient la position de l'animation future.\\
		\end{itemize}
	\end{description}
\end{document}