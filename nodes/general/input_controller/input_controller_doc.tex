% Document type and package imports.
\documentclass[a4paper, 11pt]{article}
\usepackage[utf8]{inputenc}
\usepackage[T1]{fontenc}
\usepackage[french]{babel}
\usepackage{charter}
\usepackage[top = 2cm, bottom = 2cm, left = 1cm, right = 1cm]{geometry}
\usepackage{setspace}
\usepackage{color}
\usepackage{xcolor}
\usepackage{hyperref}
\usepackage{tocloft}

% Preanblue.
\onehalfspacing
\definecolor{gray}{rgb}{0.4, 0.4, 0.4}
\definecolor{silver}{rgb}{0.95, 0.95, 0.95}
\renewcommand{\thesection}{\Roman{section} --}
\definecolor{darkgreen}{HTML}{1E8C15}
\cftsetindents{section}{1em}{2.5em}
\hypersetup {colorlinks=true, linkcolor=blue, urlcolor=blue, pdftitle={InputController module doc}}

% The start of the article.
\begin{document}
	% Change document background to silver color.
	\pagecolor{silver}
	% InputController module description.
	\huge{\hspace{12.5cm}\textit{\textbf{\textcolor{darkgreen}{InputControllerFx}}}}\large{}
	\tableofcontents \newpage
	% InputController module definition.
	\section{Définition}
	\textcolor{darkgreen}{\textbf{InputControllerFx}} est un module conçut pour la gestion des différentes 
	entrées de l'utilisateur \\(Clavier, Souris, et Manette). L'objectif de ce module est de permettre aux
	développeurs de pouvoir \\facilement géré les entrées de ses clients sans trop se gèner. Concernant 
	l'utilisation de ce dernier, le \\développeur doit être précis sur les touches dont-il souhaite prendre 
	en charge dans son jeu. \\Une mauvaise utilisation peut conduire à des bugs. Notez que ce module prend 
	également en charge les changements de touches dynamiques (\textit{\textcolor{gray}{keybindings}}).\\
	\textcolor{red}{\textbf{NB}:} Ce module est compatible à un jeu 2D, 3D et est sauvegardable. Avant de 
	continuer, le module ne supporte pas encore les écrans tactiles.

	% InputController properties definition.
	\section{Les propriétés disponibles}
	% PlayerIndex property.
	\textbf{+ \textcolor{red}{int} PlayerIndex:} Quel joueur voulez-vous écouter ? Notez que cette option
	n'est destiné qu'aux manettes.\\\\
	% DisableMouseMotion property.
	\textbf{+ \textcolor{red}{bool} DisableMouseMotion = \textcolor{red}{true}:} Contrôle l'activation ainsi 
	que la désactivation de la prise en charge des mouvements de la souris lors d'une opération de 
	changement de touche (\textit{\textcolor{gray}{keybindings}}).\\\\
	% Inputs property.
	\textbf{+ \textcolor{darkgreen}{Array} \hypertarget{inputs}{Inputs}:} Tableau de dictionnaires contenant 
	toutes les différentes configurations sur chaque touche prise en charge par le développeur. Les 
	dictionnaires issus de ce tableau supportent les clés suivantes:
	\begin{itemize}
		\item[>> \textbf{\textcolor{red}{int} controller = \textcolor{blue}{0}}:] Quel contrôleur voulez-
		vous écouter ? Les valeurs possibles sont:
		\begin{itemize}
			\item[-> \textbf{\textcolor{gray}{InputControllerFx.KEYBOARD} ou \textcolor{blue}{0}}:] Ecoute 
			le clavier de l'ordinateur.
			\item[-> \textbf{\textcolor{gray}{InputControllerFx.MOUSE} ou \textcolor{blue}{1}}:] Ecoute la 
			souris de l'ordinateur.
			\item[-> \textbf{\textcolor{gray}{InputControllerFx.JOYSTICK} ou \textcolor{blue}{2}}:] Ecoute 
			la manette connectée à l'ordinateur.
			\item[-> \textbf{\textcolor{gray}{InputControllerFx.INPUTMAP} ou \textcolor{blue}{3}}:] Ecoute 
			la carte des entrées définit au sein de Godot.
			\item[-> \textbf{\textcolor{gray}{InputControllerFx.ALL} ou \textcolor{blue}{4}}:] Ecoute 
			l'entrée de n'importe quel contrôleur.\\
		\end{itemize}
		\item[>> \textbf{\textcolor{darkgreen}{String} \hypertarget{key}{key}}:] Quel est le nom de la 
		touche du contrôleur à mettre en écoute ? A ce niveau, il y a certaines restrictions que vous devez 
		respecter. Si le contrôleur est le clavier, vous êtes plié de mettre le nom correspondant à la 
		touche recherchée tout en respectant ce que renvoie \textcolor{gray}{\\Godot} lorsqu'on appuie sur 
		une telle touche. Si le contrôleur est la souris, en cas de bouton, il doit suivre la nomenclature 
		\textcolor{gray}{\textbf{Mouse+IndexDeLaToucheVoulue}}. \\Exemple: \textcolor{gray}{\textit{Mouse2, 
		Mouse1, ...MouseN}}. \\En cas d'axe, le ciblage des défilements doit se faire avec les mots clés 
		suivants: \textcolor{gray}{\textit{-ScrollX, +ScrollX, -ScrollY, +ScrollY}} et celui des mouvements
		avec \textcolor{gray}{-MouseX, +MouseX, -MouseY, +MouseY}. Si le contrôleur est la manette, en cas 
		de bouton, il doit suivre la nomenclature \textcolor{gray}{\textbf{\\Joy+IndexDeLaToucheVoulue}}.\\
		Exemple: \textcolor{gray}{\textit{Joy0, Joy1, ...JoyN}}. \\En cas d'axe, suivre la nomenclature 
		\textcolor{gray}{\textbf{Signe+Axis+IndexDeLaxeVoulu}}. \\Exemple: \textcolor{gray}{\textit{-Axis2, 
		+Axis0, ...-/+AxisN}}. \newpage Si vous écoutez la carte d'une ou de plusieurs entrée(s) prédéfinit 
		dans les configurations de votre projet, vous devez renseigner l'identificateur de la carte à 
		ciblée.\\
		\item[>> \textbf{\textcolor{red}{bool} any = \textcolor{red}{false}}:] Voulez-vous écouter toutes
		les touches du contrôleur actuel ?\\
		\item[>> \textbf{\textcolor{red}{bool} translate = \textcolor{red}{false}}:] Voulez-vous traduire 
		la touche actuelle en terme plus claire pour les \\futures utilisateurs de votre application ?
		L'activation de cette option essaye de faire une \\traduction objective de la touche détectée tout 
		en s'adaptant au contrôleur détecté. Cela ne marche pas à tous les fois. Cette option n'intervient 
		qu'en mode \textcolor{gray}{\textit{keybindings}}. Notez qu'elle n'est pas activée lorsque le 
		contrôleur utilise la carte des entrées définit au sein de Godot.\\
		\item[>> \textbf{\textcolor{red}{bool} changed = \textcolor{red}{false}}:] Voulez-vous changer la 
		touche actuelle ? Cette option est très utilisé dans l'implémentation d'un système de 
		\textcolor{gray}{\textit{keybindings}}.\\
		\item[>> \textbf{\textcolor{red}{float} strength = \textcolor{blue}{0.0}}:] Contient la valeur d'axe
		maximale que renvoye un bouton. Cette option \\devient utile lorsqu'au cours de l'exécution du jeu,
		une touche d'axe a été remplacée par un \\bouton au cours d'un processus de changement de touche. 
		Les valeurs possibles de cette clé sont dans l'intervalle [\textcolor{blue}{-1.0}; \textcolor{blue}
		{1.0}]. Notez que cette valeur n'est renvoyée que si et seulement si la touche renseignée au niveau 
		de la clé \textit{\hyperlink{key}{key}} est un bouton.\\
		\item[>> \textbf{\textcolor{red}{bool} negative = \textcolor{red}{false}}:] Quel sera le signe de la 
		valeur de la touche ou de l'axe détecté(s) ? Par défaut, la valeur retournée est strictement 
		positive ou égale à 0.0.\\
		\item[>> \textbf{\textcolor{red}{bool} loop = \textcolor{red}{false}}:] Souhaitez-vous mettre en 
		écoute continue la touche à détectée ? A ce niveau, tant que celle-ci sera maintenue enfoncée, elle 
		sera toujours détectée.\\
		\item[>> \textbf{\textcolor{red}{bool} timeout = \textcolor{blue}{0.0}}:] Quel est le délai avant
		chaque détection. Lorsque l'option \textcolor{gray}{\textit{loop}} est activée, on obtient des 
		intervalles réguliers de détection. Dans le cas contraire, l'utilisateur est plié d'effectuer un 
		appuie long pour déclencher l'action prévue à cet effet.\\
		\item[>> \textbf{\textcolor{red}{int} \hypertarget{type}{type} = \textcolor{blue}{1}}:] Quel type de 
		donnée désirez-vous renvoyée. Les valeurs possibles sont celles \\définient au sein de Godot. 
		Cependant les types prises en charge sont: \textit{\textcolor{gray}{TYPE\_FLOAT}} et
		\textit{\textcolor{gray}{\\TYPE\_VECTOR2}}.\\
		\item[>> \textbf{\textcolor{red}{int} axis = \textcolor{blue}{1}}:] Quel axe sera considéré pour le 
		retour des valeurs issues de la touche détectée ? Cette option prend son utilitée lorsque la valeur
		de la clé \textcolor{gray}{\textit{type}} est sur \textcolor{blue}{1}. Les valeurs possibles sont:
		\begin{itemize}
			\item[-> \textbf{\textcolor{gray}{MegaAssets.Axis.NONE} ou \textcolor{blue}{0}}:] Renvoie une 
			valeur nulle ou un vecteur nulle quelque soit la détection.
			\item[-> \textbf{\textcolor{gray}{MegaAssets.Axis.X} ou \textcolor{blue}{1}}:] Renvoie un 
			vecteur d'abscisse différent de zéro ou nulle.
			\item[-> \textbf{\textcolor{gray}{MegaAssets.Axis.Y} ou \textcolor{blue}{2}}:] Renvoie un 
			vecteur d'ordonné différent de zéro ou nulle.\\
		\end{itemize}
		\item[>> \textbf{\textcolor{red}{int} count = \textcolor{blue}{1}}:] Combien de fois la touche sera 
		détectée avant le déclenchement de(s) action(s) prévu à son égard ?\\
		\item[>> \textbf{\textcolor{red}{int} step = \textcolor{blue}{1}}:] Quel est le pas à ajouté à 
		chaque fois que la touche en question est détectée ? Cette option ne s'active que lorsque sa valeur 
		et celle de \textit{\textcolor{gray}{count}} sont supérieur à 0.\\
		\item[>> \textbf{\textcolor{red}{bool} resetlevel = \textcolor{red}{true}}:] Voulez-vous 
		rénitialiser le niveau d'appuie de la touche à chaque fois qu'il serait égale à sa valeur maximale ?
		En d'autres termes, doit-on remettre à zéro le niveau d'appuie de la touche lorsqu'elle déclenche
		l'action qui lui est due ?\\
		\item[>> \textbf{\textcolor{red}{int} decrease = \textcolor{blue}{0}}:] Quel est le pas à enlevé 
		lorsque la touche détectée est relâchée ? Cette option permet de mettre en place un système de combo 
		contre système. Elle devient utile lorsque l'on souhaite que le joueur appuie sur une certaine 
		touche donnée de façon répété pour accomplir une action. Cette fonctionnalité fera office de 
		décrémenteur pour ramener les éfforts de l'utilisateur à zéro, si ce dernier décide d'abandonner. 
		Cette option ne s'active que lorsque sa valeur et celle de la clé \textcolor{gray}
		{\textit{frequence}} sont supérieur à zéro.\\
		\item[>> \textbf{\textcolor{red}{float} frequence = \textcolor{blue}{0.03}}:] Quel est le temps mort 
		avant chaque décrémentation. Cette option ne s'active que lorsque sa valeur et celle de
		\textcolor{gray}{\textit{decrease}} sont supérieur à zéro.\\
		\item[>> \textbf{\textcolor{darkgreen}{Array} keydown}]: Que se passera t-il lorsqu'on appuyera sur 
		la touche en question ? Cette clé contient tous les configurations relatives à un ou plusieurs flux 
		d'exécution(s). L'utilisation de cette clé est déjà décrite au niveau des bases du framework. 
		Précisement le sujet portant sur \\l'utilisation de la propriété \textcolor{gray}{EventsBindings} 
		(la section des actions d'un événement).\\
		\item[>> \textbf{\textcolor{darkgreen}{Array} keyup}]: Que se passera t-il lorsqu'on relâchera la
		touche en question ? Notez que cette \\dernière possède les mêmes propriétés que la clé 
		\textit{\textcolor{gray}{keydown}} et s'utilise de la même manière que lui.\\
	\end{itemize}
	% InputsMap property.
	\textbf{+ \textcolor{darkgreen}{PoolStringArray} InputsMap:} Contient la liste complète des cartes des
	entrées pré-configurées dans les configurations du projet en cours de réalisation. Cette liste est 
	également solicitée lorsque le module passe en mode \textit{\textcolor{gray}{keybindings}}.\\\\
	% ExternalInputs property.
	\textbf{+ \textcolor{darkgreen}{Array} \hypertarget{extinp}{ExternalInputs}:} Tableau de
	\textcolor{darkgreen}{NodePath} ou de \textcolor{darkgreen}{String} pointant tous vers les différents 
	instances de ce module en vue d'établir une écoute générale.\\
	\textcolor{red}{\textbf{NB}:} Les répétitions au niveau des touches ne sont pas tolérées.

	% InputController methods definition.
	\section{Les méthodes disponibles}
	% Void keyboard_activation () method description.
	\begin{description}
		\item [+ \textcolor{red}{void} \textcolor{blue}{keyboard\_activation} (activation, delay = 0.0):] 
		Contrôle l'état du clavier en terme de \\permition.
		\begin{itemize}
			\item [>> \textbf{\textcolor{red}{bool} activation}:] Contient l'état du clavier.
			\item [>> \textbf{\textcolor{red}{float} delay}:] Quel est le temps mort avant le changement 
			d'état ?\\
		\end{itemize}
	\end{description}
	% Void mouse_activation () method description.
	\newpage \begin{description}
		\item [+ \textcolor{red}{void} \textcolor{blue}{mouse\_activation} (activation, delay = 0.0):] 
		Contrôle l'état de la souris en terme de \\permition.
		\begin{itemize}
			\item [>> \textbf{\textcolor{red}{bool} activation}:] Contient l'état de la souris.
			\item [>> \textbf{\textcolor{red}{float} delay}:] Quel est le temps mort avant le changement 
			d'état ?\\
		\end{itemize}
	\end{description}
	% Void joystick_activation () method description.
	\begin{description}
		\item [+ \textcolor{red}{void} \textcolor{blue}{joystick\_activation} (activation, delay = 0.0):] 
		Contrôle l'état de la manette en terme de permition.
		\begin{itemize}
			\item [>> \textbf{\textcolor{red}{bool} activation}:] Contient l'état de la manette.
			\item [>> \textbf{\textcolor{red}{float} delay}:] Quel est le temps mort avant le changement 
			d'état ?\\
		\end{itemize}
	\end{description}
	% Bool is_keyboad_enabled () method description.
	\begin{description}
		\item [+ \textcolor{red}{bool} \textcolor{blue}{is\_keyboard\_enabled} ():] Le clavier est-il en 
		état d'envoyer des signaux pour déclencher des actions ?\\
	\end{description}
	% Bool is_mouse_enabled () method description.
	\begin{description}
		\item [+ \textcolor{red}{bool} \textcolor{blue}{is\_mouse\_enabled} ():] La souris est-elle en état 
		d'envoyer des signaux pour déclencher des actions ?\\
	\end{description}
	% Bool is_joystick_enabled () method description.
	\begin{description}
		\item [+ \textcolor{red}{bool} \textcolor{blue}{is\_joystick\_enabled} ():] La manette est-elle en 
		état d'envoyer des signaux pour déclencher des actions ?\\
	\end{description}
	% Int get_detected_key () method description.
	\begin{description}
		\item [+ \textcolor{red}{int} \textcolor{blue}{get\_detected\_key} ():] Renvoie l'index de position
		de la touche qui a été détectée suite à un appuie ou un relâchement.\\
	\end{description}
	% PoolStringArray | String get_translation_of () method description.
	\begin{description}
		\item [+ \textcolor{darkgreen}{PoolStringArray | String} \textcolor{blue}{get\_translation\_of} 
		(id = null):] Renvoie le(s) nom(s) réel(s) de(s) \\touche(s) au sein du paramètre \textcolor{gray}
		{\textit{id}} en fonction du contrôleur détecté. Par défaut, la traduction se fera sur la touche 
		actuellement détectée.
		\begin{itemize}
			\item [>> \textbf{\textcolor{darkgreen}{String | PoolIntArray | Array | PoolStringArray} |
			\textcolor{red}{int} id}:] Contient le(s) identifiant(s) de(s) touche(s) à ciblée(s). La/les 
			chaîne(s) de caractères fait/font ici référence aux valeurs que doit prendre la clé 
			\textit{\hyperlink{key}{key}}. Le(s) entier(s) ici fait/font référence aux index de position de 
			la ou des touche(s) pris pour cible dans le champ \textit{\hyperlink{inputs}{Inputs}}.\\
		\end{itemize}
	\end{description}
	% String get_detected_controller () method description.
	\begin{description}
		\item [+ \textcolor{darkgreen}{String} \textcolor{blue}{get\_detected\_controller} ():] Renvoie le 
		nom du contrôleur actuellement détecté. La \\valeur \textcolor{gray}{Unknown Controller} est 
		renvoyée lorsqu'un contrôleur n'a pas pu être identifié.\\
	\end{description}
	% Int | PoolIntArray get_keycode () method description.
	\begin{description}
		\item [+ \textcolor{red}{int} | \textcolor{darkgreen}{PoolIntArray} \textcolor{blue}{get\_keycode} 
		(id = null):] Renvoie le code de la/des touche(s) référencée(s) dans le paramètre \textcolor{gray}
		{\textit{id}}. Par défaut, le code de la touche actuellement détectée est renvoyé.
		\begin{itemize}
			\item [>> \textbf{\textcolor{darkgreen}{String | PoolIntArray | Array | PoolStringArray} |
			\textcolor{red}{int} id}:] Contient le(s) identifiant(s) de(s) touche(s) à ciblée(s). La/les 
			chaîne(s) de caractères fait/font ici référence aux valeurs que doit prendre la clé 
			\textit{\hyperlink{key}{key}}. Le(s) entier(s) ici fait/font référence aux index de position de 
			la ou des touche(s) pris pour cible dans le champ \textit{\hyperlink{inputs}{Inputs}}.\\
		\end{itemize}
	\end{description}
	% float | Vector2 get_normalized_value () method description.
	\begin{description}
		\item [+ \textcolor{red}{float} | \textcolor{darkgreen}{Vector2} \textcolor{blue}
		{get\_normalized\_value} (id = null):] Renvoie une valeur d'axe comprise entre \textcolor{blue}{0.0} 
		et \textcolor{blue}{1.0} de la touche référencée dans le paramètre \textcolor{gray}{\textit{id}}. 
		Par défaut, la valeur d'axe de la touche actuellement détectée est renvoyée. Notez que le changement 
		de type de valeur au niveau de cette fonction se fait par rapport à la valeur contenue dans la clé 
		\textit{\hyperlink{type}{type}} du champ \textit{\hyperlink{inputs}{Inputs}}.
		\begin{itemize}
			\item [>> \textbf{\textcolor{darkgreen}{String} | \textcolor{red}{int} id}:] Contient 
			l'identifiant de la touche à ciblée. La chaîne de caractères fait ici \\référence aux valeurs 
			que doit prendre la clé \textit{\hyperlink{key}{key}}. L'entier ici fait alusion à l'index de 
			position de la touche pris pour cible dans le champ \textit{\hyperlink{inputs}{Inputs}}.\\
		\end{itemize}
	\end{description}
	% Void key_activation () method description.
	\begin{description}
		\item [+ \textcolor{red}{void} \textcolor{blue}{key\_activation} (id, activation, delay = 0.0):] 
		Contrôle l'état d'une touche donnée en terme de permition.
		\begin{itemize}
			\item [>> \textbf{\textcolor{darkgreen}{String | PoolIntArray | Array | PoolStringArray} |
			\textcolor{red}{int} id}:] Contient le(s) identifiant(s) de(s) touche(s) à ciblée(s). La/les 
			chaîne(s) de caractères fait/font ici référence aux valeurs que doit prendre la clé 
			\textit{\hyperlink{key}{key}}. Le(s) entier(s) ici fait/font référence aux index de position de 
			la ou des touche(s) pris pour cible dans le champ \textit{\hyperlink{input}{Inputs}}.
			\item [>> \textbf{\textcolor{red}{bool} activation}:] Contient l'état de la touche en question.
			\item [>> \textbf{\textcolor{red}{float} delay}:] Quel est le temps mort avant le changement 
			d'état ?\\
		\end{itemize}
	\end{description}
	% Bool is_enabled () method description.
	\begin{description}
		\item [+ \textcolor{red}{bool} \textcolor{blue}{is\_enabled} (id):] Détermine si la ou les touche(s) 
		donnée(s) est/sont active(s) pour \\d'éventuelle appuies ou relâchements.
		\begin{itemize}
			\item [>> \textbf{\textcolor{darkgreen}{String | PoolIntArray | Array | PoolStringArray} |
			\textcolor{red}{int} id}:] Contient le(s) identifiant(s) de(s) touche(s) à ciblée(s). La/les 
			chaîne(s) de caractères fait/font ici référence aux valeurs que doit prendre la clé 
			\textit{\hyperlink{key}{key}}. Le(s) entier(s) ici fait/font référence aux index de position de 
			la ou des touche(s) pris pour cible dans le champ \textit{\hyperlink{inputs}{Inputs}}.\\
		\end{itemize}
	\end{description}
	% Void enable_keys_without () method description.
	\begin{description}
		\item [+ \textcolor{red}{void} \textcolor{blue}{enable\_keys\_without} (ids, delay = 0.0):] Active 
		l'utilisation de toutes les touches prises en charge par le développeur exceptées celles renseignées 
		dans le paramètre \textcolor{gray}{\textit{ids}}.
		\begin{itemize}
			\item [>> \textbf{\textcolor{darkgreen}{Array} ids}:] Contient les identifiants des touches à 
			ciblées. Chaque élément de ce tableau accepte soit un \textcolor{darkgreen}{String} (la valeur 
			contenue dans clé \textit{\hyperlink{key}{key}}), soit un \textcolor{red}{int} (l'index de 
			position d'une touche au sein du champ \textit{\hyperlink{inputs}{Inputs}}).
			\item [>> \textbf{\textcolor{red}{float} delay}:] Quel est le temps mort avant les changements
			d'états ?\\
		\end{itemize}
	\end{description}
	% Void disable_keys_without () method description.
	\begin{description}
		\item [+ \textcolor{red}{void} \textcolor{blue}{disable\_keys\_without} (ids, delay = 0.0):] 
		Désactive l'utilisation de toutes les touches prises en charge par le développeur exceptées celles 
		renseignées dans le paramètre \textcolor{gray}{\textit{ids}}.
		\begin{itemize}
			\item [>> \textbf{\textcolor{darkgreen}{Array} ids}:] Contient les identifiants des touches à 
			ciblées. Chaque élément de ce tableau accepte soit un \textcolor{darkgreen}{String} (la valeur 
			contenue dans clé \textit{\hyperlink{key}{key}}), soit un \textcolor{red}{int} (l'index de 
			position d'une touche au sein du champ \textit{\hyperlink{inputs}{Inputs}}).
			\item [>> \textbf{\textcolor{red}{float} delay}:] Quel est le temps mort avant les changements
			d'états ?\\
		\end{itemize}
	\end{description}
	% Int | PoolIntArray get_enabled_keys () method description.
	\begin{description}
		\item [+ \textcolor{red}{int} | \textcolor{darkgreen}{PoolIntArray} \textcolor{blue}
		{get\_enabled\_keys} ():] Renvoie les identifants de toutes les touches n'ayant pas été mise hors 
		d'usage.\\
	\end{description}
	% Int | PoolIntArray get_disabled_keys () method description.
	\begin{description}
		\item [+ \textcolor{red}{int} | \textcolor{darkgreen}{PoolIntArray} \textcolor{blue}
		{get\_disabled\_keys} ():] Renvoie les identifiants de toutes les touches ayant été mise hors 
		d'usage.\\
	\end{description}
	% Dictionary get_enabled_keys_data () method description.
	\begin{description}
		\item [+ \textcolor{darkgreen}{Dictionary} \textcolor{blue}{get\_enabled\_keys\_data} (json = 
		true):] Renvoie les données de toutes les touches n'ayant pas été mise hors d'usage.
		\begin{itemize}
			\item [>> \textbf{\textcolor{red}{bool} json}:] Voulez-vous renvoyer le résultat sous le format 
			json ?\\
		\end{itemize}
	\end{description}
	% Dictionary get_disabled_keys_data () method description.
	\begin{description}
		\item [+ \textcolor{darkgreen}{Dictionary} \textcolor{blue}{get\_disabled\_keys\_data} (json = 
		true):] Renvoie les données de toutes les touches ayant été mise hors d'usage.
		\begin{itemize}
			\item [>> \textbf{\textcolor{red}{bool} json}:] Voulez-vous renvoyer le résultat sous le format 
			json ?\\
		\end{itemize}
	\end{description}
	% Void set_key_level () method description.
	\begin{description}
		\item [+ \textcolor{red}{void} \textcolor{blue}{set\_key\_level} (id = null, new\_value, delay = 
		0.0):] Redéfinie la valeur du niveau d'appuie d'une touche donnée. Par défaut, cette méthode cible
		le niveau d'appuie de la touche actuellement détectée.
		\begin{itemize}
			\item [>> \textbf{\textcolor{darkgreen}{String} | \textcolor{red}{int} id}:] Contient 
			l'identifiant de la touche à ciblée. La chaîne de caractères fait ici \\référence aux valeurs 
			que doit prendre la clé \textit{\hyperlink{key}{key}}. L'entier ici fait alusion à l'index de 
			position de la touche pris pour cible dans le champ \textit{\hyperlink{inputs}{Inputs}}.
			\item [>> \textbf{\textcolor{red}{int} new\_value}:] Quelle est la nouvelle valeur du niveau 
			d'appuie ?
			\item [>> \textbf{\textcolor{red}{float} delay}:] Quel est le temps mort avant la redéfinition ? 
			\\
		\end{itemize}
	\end{description}
	% Int get_key_level () method description.
	\begin{description}
		\item [+ \textcolor{red}{int} \textcolor{blue}{get\_key\_level} (id = null):] Renvoie la valeur du 
		niveau d'appuie d'une touche donnée. Par défaut, cette méthode renvoie le valeur du niveau d'appuie 
		de la touche actuellement détectée.
		\begin{itemize}
			\item [>> \textbf{\textcolor{darkgreen}{String} | \textcolor{red}{int} id}:] Contient 
			l'identifiant de la touche à ciblée. La chaîne de caractères fait ici \\référence aux valeurs 
			que doit prendre la clé \textit{\hyperlink{key}{key}}. L'entier ici fait alusion à l'index de 
			position de la touche pris pour cible dans le champ \textit{\hyperlink{inputs}{Inputs}}.\\
		\end{itemize}
	\end{description}
	% Bool is_combo () method description.
	\begin{description}
		\item [+ \textcolor{red}{bool} \textcolor{blue}{is\_combo} (id = null):] Détermine si une ou 
		plusieurs touche(s) est/sont en traint d'être appuyée(s) de façon répétée. Par défaut, cette méthode 
		renvoie un résultat en se basant sur la touche actuellement détectée.
		\begin{itemize}
			\item [>> \textbf{\textcolor{darkgreen}{String | PoolIntArray | Array | PoolStringArray} |
			\textcolor{red}{int} id}:] Contient le(s) identifiant(s) de(s) touche(s) à ciblée(s). La/les 
			chaîne(s) de caractères fait/font ici référence aux valeurs que doit prendre la clé
			\textit{\hyperlink{key}{key}}. Le(s) entier(s) ici fait/font référence aux index de position de 
			la ou des touche(s) pris pour cible dans le champ \textit{\hyperlink{inputs}{Inputs}}.\\
		\end{itemize}
	\end{description}
	% float | Vector2 get_axis_value () method description.
	\begin{description}
		\item [+ \textcolor{red}{float} | \textcolor{darkgreen}{Vector2} \textcolor{blue}{get\_axis\_value} 
		(id = null):] Renvoie la valeur d'axe d'une touche grâce à son identifiant. Par défaut, la valeur 
		d'axe de la touche actuellement détectée est renvoyée. Notez que le changement de type de valeur au 
		niveau de cette fonction se fait par rapport à la valeur contenue dans la clé
		\textit{\hyperlink{type}{type}} du champ \textit{\hyperlink{inputs}{Inputs}}.
		\begin{itemize}
			\item [>> \textbf{\textcolor{darkgreen}{String} | \textcolor{red}{int} id}:] Contient 
			l'identifiant de la touche à ciblée. La chaîne de caractères fait ici \\référence aux valeurs 
			que doit prendre la clé \textit{\hyperlink{key}{key}}. L'entier ici fait alusion à l'index de 
			position de la touche pris pour cible dans le champ \textit{\hyperlink{inputs}{Inputs}}.\\
		\end{itemize}
	\end{description}
	% Int get_key_pourcent () method description.
	\begin{description}
		\item [+ \textcolor{red}{int} \textcolor{blue}{get\_key\_pourcent} (id = null):] Renvoie en 
		pourcentage la valeur du niveau d'appuie d'une touche grâce à son identifiant. Par défaut, le 
		pourcentage de la valeur d'appuie de la touche actuellement détectée est renvoyée.
		\begin{itemize}
			\item [>> \textbf{\textcolor{darkgreen}{String} | \textcolor{red}{int} id}:] Contient 
			l'identifiant de la touche à ciblée. La chaîne de caractères fait ici \\référence aux valeurs 
			que doit prendre la clé \textit{\hyperlink{key}{key}}. L'entier ici fait alusion à l'index de 
			position de la touche pris pour cible dans le champ \textit{\hyperlink{inputs}{Inputs}}.\\
		\end{itemize}
	\end{description}
	% Bool is_a_button () method description.
	\begin{description}
		\item [+ \textcolor{red}{bool} \textcolor{blue}{is\_a\_button} (id = null):] Détermine si la ou les 
		touche(s) donnée(s) est/sont un/des \\bouton(s) ou pas. Par défaut, la vérification se fait sur la 
		touche actuellement détectée.
		\begin{itemize}
			\item [>> \textbf{\textcolor{darkgreen}{String | PoolIntArray | Array | PoolStringArray} |
			\textcolor{red}{int} id}:] Contient le(s) identifiant(s) de(s) touche(s) à ciblée(s). La/les 
			chaîne(s) de caractères fait/font ici référence aux valeurs que doit prendre la clé
			\textit{\hyperlink{key}{key}}. Le(s) entier(s) ici fait/font référence aux index de position de 
			la ou des touche(s) pris pour cible dans le champ \textit{\hyperlink{inputs}{Inputs}}.\\
		\end{itemize}
	\end{description}
	% Bool is_an_axis () method description.
	\begin{description}
		\item [+ \textcolor{red}{bool} \textcolor{blue}{is\_an\_axis} (id = null):] Détermine si la ou les 
		touche(s) donnée(s) est/sont un/des \\bouton(s) d'axe ou pas. Par défaut, la vérification se fait 
		sur la touche actuellement détectée.
		\begin{itemize}
			\item [>> \textbf{\textcolor{darkgreen}{String | PoolIntArray | Array | PoolStringArray} |
			\textcolor{red}{int} id}:] Contient le(s) identifiant(s) de(s) touche(s) à ciblée(s). La/les 
			chaîne(s) de caractères fait/font ici référence aux valeurs que doit prendre la clé
			\textit{\hyperlink{key}{key}}. Le(s) entier(s) ici fait/font référence aux index de position de 
			la ou des touche(s) pris pour cible dans le champ \textit{\hyperlink{inputs}{Inputs}}.\\
		\end{itemize}
	\end{description}
	% Bool is_pressed () method description.
	\begin{description}
		\item [+ \textcolor{red}{bool} \textcolor{blue}{is\_pressed} (id = null):] Détermine si la ou les 
		touche(s) donnée(s) est/sont appuyée(s) ou pas. Par défaut, la vérification se fait sur la touche 
		actuellement détectée.
		\begin{itemize}
			\item [>> \textbf{\textcolor{darkgreen}{String | PoolIntArray | Array | PoolStringArray} |
			\textcolor{red}{int} id}:] Contient le(s) identifiant(s) de(s) touche(s) à ciblée(s). La/les 
			chaîne(s) de caractères fait/font ici référence aux valeurs que doit prendre la clé
			\textit{\hyperlink{key}{key}}. Le(s) entier(s) ici fait/font référence aux index de position de 
			la ou des touche(s) pris pour cible dans le champ \textit{\hyperlink{inputs}{Inputs}}.
		\end{itemize}
	\end{description}

	% InputController signals definition.
	\section{Les événements disponibles}
	% key_changed event description.
	\begin{description}
		\item [+ \textcolor{blue}{key\_changed} (data):] Signal déclenché lorsqu'on change de touche dans un 
		procèssus de \textcolor{gray}{\textit{\\Keybindings}}. Cet événement renvoie un dictionaire 
		contenant les clés suivantes:
		\begin{itemize}
			\item [>> \textbf{\textcolor{darkgreen}{Node} node}:] Contient le noeud où cet signal a été 
			émit.
			\item [>> \textbf{\textcolor{darkgreen}{String} key}:] Contient la nouvelle touche détectée.\\
		\end{itemize}
	\end{description}
	% key_action event description.
	\begin{description}
		\item [+ \textcolor{blue}{key\_action} (data):] Signal déclenché lorsque la touche détectée respecte 
		les critères exigées par le développeur conduisant ainsi à l'exécution des actions prévues à cet 
		éffet. Cet événement \\renvoie un dictionaire contenant les clés suivantes:
		\begin{itemize}
			\item [>> \textbf{\textcolor{darkgreen}{Node} node}:] Contient le noeud où cet signal a été 
			émit.
			\item [>> \textbf{\textcolor{darkgreen}{String} key}:] Contient la touche détectée.
			\item [>> \textbf{\textcolor{red}{float} delta}:] Contient le temps éffectué depuis la dernière 
			trame du jeu.\\
		\end{itemize}
	\end{description}
	% key_cloned event description.
	\begin{description}
		\item [+ \textcolor{blue}{key\_cloned} (data):] Signal déclenché lorsque la touche détectée dans un 
		procèssus de \textcolor{gray}{\textit{\\Keybindings}} existe déjà dans la liste des touches prise en 
		charge par le module. Cet événement renvoie un dictionaire contenant les clés suivantes:
		\begin{itemize}
			\item [>> \textbf{\textcolor{darkgreen}{Node} node}:] Contient le noeud où cet signal a été 
			émit.
			\item [>> \textbf{\textcolor{darkgreen}{PoolIntArray} | \textcolor{red}{int} index}:] Contient 
			le(s) position(s) de(s) clone(s).
			\item [>> \textbf{\textcolor{darkgreen}{Node | Array} other}:] Contient le(s) noeud(s) contenant 
			des clones. Cette clé n'est renvoyée que si des répétitions ont été trouvées dans les autres 
			instances de ce module par le biais du champ \textit{\hyperlink{extinp}{ExternalInputs}}.\\
		\end{itemize}
	\end{description}
	% key_level_changed event description.
	\begin{description}
		\item [+ \textcolor{blue}{key\_level\_changed} (data):] Signal déclenché lorsqu'on change le niveau 
		d'appuie d'une touche. Cet événement renvoie un dictionaire contenant les clés suivantes:
		\begin{itemize}
			\item [>> \textbf{\textcolor{darkgreen}{Node} node}:] Contient le noeud où cet signal a été 
			émit.
			\item [>> \textbf{\textcolor{red}{int} value}:] Contient la nouvelle valeur du niveau d'appuie.
			\item [>> \textbf{\textcolor{red}{int} index}:] Contient l'index de position de la touche qui a 
			été affectée par ce changement.\\
		\end{itemize}
	\end{description}
	% keydown event description.
	\begin{description}
		\item [+ \textcolor{blue}{keydown} (data):] Signal déclenché lorsqu'on appuie sur une touche prise 
		en charge par le \\module. Cet événement renvoie un dictionaire contenant les clés suivantes:
		\begin{itemize}
			\item [>> \textbf{\textcolor{darkgreen}{Node} node}:] Contient le noeud où cet signal a été 
			émit.
			\item [>> \textbf{\textcolor{red}{int} index}:] Contient l'index de position de la touche qui a 
			été appuyée.\\
		\end{itemize}
	\end{description}
	% keyup event description.
	\begin{description}
		\item [+ \textcolor{blue}{keyup} (data):] Signal déclenché lorsqu'on relâche une touche prise en 
		charge par le module. Cet événement renvoie un dictionaire contenant les clés suivantes:
		\begin{itemize}
			\item [>> \textbf{\textcolor{darkgreen}{Node} node}:] Contient le noeud où cet signal a été 
			émit.
			\item [>> \textbf{\textcolor{red}{int} index}:] Contient l'index de position de la touche qui a 
			été relâchée.\\
		\end{itemize}
	\end{description}
	% decreaser_started event description.
	\begin{description}
		\item [+ \textcolor{blue}{decreaser\_started} (data):] Signal déclenché lorsque le décrémenteur 
		automatique se démarre. Cet événement renvoie un dictionaire contenant la clé suivante:
		\begin{itemize}
			\item [>> \textbf{\textcolor{darkgreen}{Node} node}:] Contient le noeud où cet signal a été 
			émit.\\
		\end{itemize}
	\end{description}
	% decreaser_stoped event description.
	\begin{description}
		\item [+ \textcolor{blue}{decreaser\_stoped} (data):] Signal déclenché lorsque le décrémenteur 
		automatique s'arrète. Cet événement renvoie un dictionaire contenant la clé suivante:
		\begin{itemize}
			\item [>> \textbf{\textcolor{darkgreen}{Node} node}:] Contient le noeud où cet signal a été 
			émit.\\
		\end{itemize}
	\end{description}
	% key event description.
	\begin{description}
		\item [+ \textcolor{blue}{key} (data):] Signal déclenché lorsqu'on appuie ou relâche une touche. Cet 
		événement renvoie un dictionaire contenant les clés suivantes:
		\begin{itemize}
			\item [>> \textbf{\textcolor{darkgreen}{Node} node}:] Contient le noeud où cet signal a été 
			émit.
			\item [>> \textbf{\textcolor{red}{float} delta}:] Contient le temps éffectué depuis la dernière 
			trame du jeu.
			\item [>> \textbf{\textcolor{red}{int} index}:] Contient l'index de position de la touche qui a 
			été appuyée ou relâchée.\\
		\end{itemize}
	\end{description}
\end{document}