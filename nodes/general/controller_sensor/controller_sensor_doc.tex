% Document type and package imports.
\documentclass[a4paper, 11pt]{article}
\usepackage[utf8]{inputenc}
\usepackage[T1]{fontenc}
\usepackage[french]{babel}
\usepackage{charter}
\usepackage{hyperref}
\usepackage{tocloft}
\usepackage[top = 2cm, bottom = 2cm, left = 1cm, right = 1cm]{geometry}
\usepackage{setspace}
\usepackage{color}
\usepackage{xcolor}

% Preanblue.
\onehalfspacing
\definecolor{gray}{rgb}{0.4, 0.4, 0.4}
\definecolor{silver}{rgb}{0.95, 0.95, 0.95}
\renewcommand{\thesection}{\Roman{section} --}
\definecolor{darkgreen}{HTML}{1E8C15}
\cftsetindents{section}{1em}{2.5em}
\hypersetup {colorlinks=true, linkcolor=blue, urlcolor=blue, pdftitle={ControllerSensor module doc}}

% The start of the article.
\begin{document}
	% Change document background to silver color.
	\pagecolor{silver}
	% ControllerSensor module description.
	\huge{\hspace{12cm}\textit{\textbf{\textcolor{darkgreen}{ControllerSensorFx}}}}\large{}
	\tableofcontents \newpage
	% ControllerSensor module definition.
	\section{Définition}
	\textcolor{darkgreen}{\textbf{ControllerSensorFx}} est un module conçut pour la détection des 
	différentes commandes \\interagissants avec le jeu. Il adapte les touches d'aide du jeu en fonction du 
	contrôleur détecté. Bien évidemment, ces différentes touches ne sont rien d'autres que des
	\href{https://docs.godotengine.org/en/stable/classes/class_sprite.html}
	{\textit{\textcolor{darkgreen}{Sprite}}},
	\href{https://docs.godotengine.org/en/stable/classes/class_sprite3d.html}
	{\textit{\textcolor{darkgreen}{Sprite3D}}} ou
	\href{https://docs.godotengine.org/en/stable/classes/class_texturerect.html}
	{\textit{\textcolor{darkgreen}{TextureRect}}} réalisés par le développeur. \\Exemple: Le 
	\href{https://docs.godotengine.org/en/stable/classes/class_sprite.html}
	{\textit{\textcolor{darkgreen}{Sprite}}} de la touche "X" de la manette XBox; celui de la touche "Space" 
	du clavier etc... \\Le fonctionnement de ce module est assez banal et très simple à exploité.\\
	\textcolor{red}{\textbf{NB}:} Ce module est de nature indestructible, est compatible à un jeu 2D, 3D et 
	n'est pas sauvegardable.

	% ControllerSensor properties definition.
	\section{Les propriétés disponibles}
	% Controller property.
	\textbf{+ \textcolor{red}{int} Controller = \textcolor{blue}{0}:} Contient les différents contrôleurs 
	détectés par le module. La valeur de ce champ varie en fonction des commandes connectées par 
	l'ordinateur.
	\\\\
	% Count property.
	\textbf{+ \textcolor{red}{int} \hypertarget{count}{Count} = \textcolor{blue}{-1}:} Combien de commandes 
	voulez-vous écoutée ? La valeur nulle coupe les \\détections. Tandis que celle négative entraîne une 
	détection ilimitée des commandes.\\\\
	% BaseController property.
	\textbf{+ \textcolor{red}{bool} BaseController = \textcolor{red}{true}:} Doit-on utilisée la commande de 
	base à chaque déconnexion ?\\\\
	% Adapter property.
	\textbf{+ \textcolor{red}{bool} Adapter = \textcolor{red}{false}:} Doit-on mettre à jour les
	\href{https://docs.godotengine.org/en/stable/classes/class_sprite.html}
	{\textit{\textcolor{darkgreen}{Sprite}}} des boutons lorsqu'une nouvelle commande est détectée ?\\\\
	% Categories property.
	\textbf{+ \textcolor{darkgreen}{Array} Categories:} Tableau de dictionnaires contenant toutes les 
	différentes configurations sur chaque catégory de contrôleur prise en charge par le développeur. Les 
	dictionnaires issus de ce tableau \\supportent les clés suivantes:\\
	\begin{itemize}
		\item[>> \textbf{\textcolor{darkgreen}{String} \hypertarget{cat}{category}}:] Le contrôleur détecté 
		appartient à quelle catégory (xbox 360; xbox one; \\playstation; wii; nintendo switch; etc...).\\
		\item[>> \textbf{\textcolor{red}{int} search = \textcolor{blue}{3}}:] Contient le moyen à utilisé 
		pour chercher les noeuds (\href{https://docs.godotengine.org/en/stable/classes/class_sprite.html}
		{\textit{\textcolor{darkgreen}{Sprite}}},
		\href{https://docs.godotengine.org/en/stable/classes/class_sprite3d.html}
		{\textit{\textcolor{darkgreen}{Sprite3D}}} ou
		\href{https://docs.godotengine.org/en/stable/classes/class_texturerect.html}
		{\textit{\textcolor{darkgreen}{\\TextureRect}}}) qui seront victime de l'influence de ce module. Les 
		valeurs possibles sont:
		\begin{itemize}
			\item[-> \textbf{\textcolor{gray}{MegaAssets.NodeProperty.NAME} ou \textcolor{blue}{0}}:] 
			Ciblage par nom.
			\item[-> \textbf{\textcolor{gray}{MegaAssets.NodeProperty.GROUP} ou \textcolor{blue}{1}}:] 
			Ciblage par groupe.
			\item[-> \textbf{\textcolor{gray}{MegaAssets.NodeProperty.TYPE} ou \textcolor{blue}{2}}:] 
			Ciblage par type.
			\item[-> \textbf{\textcolor{gray}{MegaAssets.NodeProperty.ALL} ou \textcolor{blue}{3}}:] Ciblage 
			sur n'importe quel type.\\
			\item[>> \textbf{\textcolor{darkgreen}{Vector2} size = \textcolor{darkgreen}{Vector2}
			(\textcolor{blue}{50}, \textcolor{blue}{50})}:] Contrôle la résolution des textures.\\
			\item[>> \textbf{\textcolor{red}{int} quality = \textcolor{blue}{2}}:] Contrôle la qualité des
			textures. Les valeurs possibles sont celles définient au sein de classe
			\href{https://docs.godotengine.org/en/stable/classes/class_image.html}
			{\textit{\textcolor{darkgreen}{Image}}} de Godot.\\
		\end{itemize}
	\end{itemize}
	\newpage Si vous voulez que la propriété \textit{\textcolor{gray}{texture}} des noeuds ciblés change en 
	fonction du type de contrôleur \\détecté par l'ordinateur, vous devez suivre dans le(s) dictionaire(s) 
	défini(en)t au sein de la propriété \textit{\hyperlink{cat}{Categories}}, la nomenclature:
	\textit{\textcolor{gray}{IdentifiantDuNoeud: LienVersImageDuSpriteAssocier}}. Si vous désirez \\affecter 
	une même texture à plusieurs noeuds à la fois, séparez chaque identifiant de noeud par un pipe. IDEM
	pour la catégory.\\\\
	% TargetScenes property.
	\textbf{+ \textcolor{darkgreen}{PoolStringArray} TargetScenes:} Cette option permet au module de savoirs 
	quant est-ce qu'il doit ciblé des noeuds ou pas. N'ajoutez que les scènes possédant des noeuds ayant 
	pour objectif de servir de bouton d'aide à l'utilisateur de votre produit. Cela permettra ainsi 
	d'optimiser les recherches à faire avant la mise à jour de la texture des noeuds pris pour cible.\\
	\textcolor{red}{\textbf{NB}:} La présence de doublons au niveau des catégories de manettes et les noms 
	des scènes n'est pas tolérée.

	% ControllerSensor methods definition.
	\section{Les méthodes disponibles}
	% PoolStringArray get_detected_controller_names () method description.
	\begin{description}
		\item [+ \textcolor{darkgreen}{PoolStringArray} \textcolor{blue}{get\_detected\_controller\_names} 
		():] Renvoie les noms de tous les contrôleurs se trouvant dans l'intervalle de détection imposé par
		le développeur. Pour agir sur l'intervalle de détection, modifiez la valeur de la propriété
		\textit{\hyperlink{count}{Count}}.\\
	\end{description}
	% String get_active_controller_name () method description.
	\begin{description}
		\item [+ \textcolor{darkgreen}{String} \textcolor{blue}{get\_active\_controller\_name} ():] Renvoie 
		le nom du contrôleur en cours d'utilisation.\\
	\end{description}
	% Int get_active_controller_index () method description.
	\begin{description}
		\item [+ \textcolor{red}{int} \textcolor{blue}{get\_active\_controller\_index} ():] Renvoie la 
		position du joueur associé au contrôleur en cours d'utilisation.
	\end{description}

	% ControllerSensor signals definition.
	\section{Les événements disponibles}
	% controller_changed event description.
	\begin{description}
		\item [+ \textcolor{blue}{controller\_changed} (name):] Signal déclenché lorsqu'on change de 
		contrôleur.
		\begin{itemize}
			\item [>> \textbf{\textcolor{darkgreen}{String} name}:] Contient le nom du nouveau contrôleur
			actif.\\
		\end{itemize}
	\end{description}
	% controller_connected event description.
	\begin{description}
		\item [+ \textcolor{blue}{controller\_connected} (name):] Signal déclenché lorsqu'un nouveau 
		contrôleur a été connecté à l'ordinateur.
		\begin{itemize}
			\item [>> \textbf{\textcolor{darkgreen}{String} name}:] Contient le nom du nouveau contrôleur
			connecté.\\
		\end{itemize}
	\end{description}
	% controller_disconnected event description.
	\begin{description}
		\item [+ \textcolor{blue}{controller\_disconnected} (name):] Signal déclenché lorsqu'un nouveau 
		contrôleur a été \\déconnecté de l'ordinateur.
		\begin{itemize}
			\item [>> \textbf{\textcolor{darkgreen}{String} name}:] Contient le nom du contrôleur 
			déconnecté.\\
		\end{itemize}
	\end{description}
\end{document}