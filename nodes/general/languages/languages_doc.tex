% Document type and package imports.
\documentclass[a4paper, 11pt]{article}
\usepackage[utf8]{inputenc}
\usepackage[T1]{fontenc}
\usepackage[french]{babel}
\usepackage{charter}
\usepackage[top = 2cm, bottom = 2cm, left = 1cm, right = 1cm]{geometry}
\usepackage{setspace}
\usepackage{color}
\usepackage{xcolor}
\usepackage{hyperref}
\usepackage{tocloft}

% Preanblue.
\onehalfspacing
\definecolor{gray}{rgb}{0.4, 0.4, 0.4}
\definecolor{silver}{rgb}{0.95, 0.95, 0.95}
\renewcommand{\thesection}{\Roman{section} --}
\definecolor{darkgreen}{HTML}{1E8C15}
\cftsetindents{section}{1em}{2.5em}
\hypersetup {colorlinks=true, linkcolor=blue, urlcolor=blue, pdftitle={Languages module doc}}

% The start of the article.
\begin{document}
	% Change document background to silver color.
	\pagecolor{silver}
	% LanguagesFx module description.
	\huge{\hspace{14cm}\textit{\textbf{\textcolor{darkgreen}{LanguagesFx}}}}\large{} \tableofcontents 
	\newpage
	% Languages module definition.
	\section{Définition}
	\textcolor{darkgreen}{\textbf{LanguagesFx}} est un module conçut pour la gestion des langues d'un jeu. 
	Elle assure le chargement des différents langages du jeu. Ces langues sont décidées par le développeur. 
	En d'autres termes, c'est le développeur lui-même qui intègre les différentes langues dont-il pourra 
	charger grâce à ce module. Cela lui permettra ainsi de prendre en charge autant de langues que possible.
	\\
	\textcolor{red}{\textbf{NB}:} Notez que ce module est de nature indestructible, est compatible à un jeu 
	2D, 3D et est \\sauvegardable.

	% Languages properties definition.
	\section{Les propriétés disponibles}
	% TargetPath property.
	\textbf{+ \textcolor{red}{int} \hypertarget{idx}{TargetPath} = \textcolor{blue}{0}:} Contient les 
	différents chemins que prend en charge ce module. Ces chemins \\représentent les endroits possibles où
	l'on peut accédé au fichier (.csv). Les valeurs possibles sont celles définient au sein de la méthode
	\textit{\textcolor{blue}{get\_os\_dir ()}} de la classe \textbf{\textcolor{darkgreen}{MegaAssets}}. Le
	champ d'activité de ce champ est uniquement sur le moteur Godot.\\\\
	% Input property.
	\textbf{+ \textcolor{darkgreen}{String} Input:} Contient le chemin pointant vers le fichier de langue à 
	prendre charge. L'extention de ce fichier doit être en (.csv). Le champ d'activité de ce champ est
	uniquement sur le moteur Godot.\\\\
	% Separator property.
	\textbf{+ \textcolor{darkgreen}{String} Separator = \textcolor{blue}{0}:} Contient les différents
	séparateurs que supporte ce module. Le champ \\d'activité de ce champ est uniquement sur le moteur
	Godot. Les valeurs possibles sont:
	\begin{itemize}
		\item [-> \textbf{\textcolor{gray}{LanguagesFx.Divider.COMMA} ou \textcolor{blue}{0}}:] Le fichier
		sera traité avec les \textbf{virgules}.
		\item [-> \textbf{\textcolor{gray}{LanguagesFx.Divider.SEMICOLON} ou \textcolor{blue}{1}}:] Le
		fichier sera traité avec les \textbf{points virgule}.
		\item [-> \textbf{\textcolor{gray}{LanguagesFx.Divider.TAB} ou \textcolor{blue}{2}}:] Le fichier
		sera traité avec les \textbf{tabulations}.\\
	\end{itemize}
	% Output property.
	\textbf{+ \textcolor{darkgreen}{String} Output:} Contient le chemin pointant vers le(s) fichier(s) de
	langue au format (.translation).\\\\
	% Security property.
	\textbf{+ \textcolor{red}{bool} \hypertarget{idx}{Security} = \textcolor{red}{true}:} Voulez-vous mettre
	une sécurité sur le(s) fichier(s) de langue ?\\\\
	% Level property.
	\textbf{+ \textcolor{red}{int} \hypertarget{idx}{Level} = \textcolor{blue}{1}:} Contient les différents
	niveaux de sécurité possibles pour la création de(s) fichier(s) au format (.translation). Les valeurs
	possibles sont celles définient au sein de la méthode \textit{\textcolor{blue}{serialize ()}} de la
	classe \textbf{\textcolor{darkgreen}{MegaAssets}}.\\\\
	% Key property.
	\textbf{+ \textcolor{darkgreen}{String} \hypertarget{pass}{Pass}:} Contient le mot de passe à utilisé
	pour sécuriser les données de langue. Vous avez la possibilité de générer automatiquement un mot de
	passe via le booléen \textit{\hyperlink{genpass}{GeneratePassword}}.\\\\
	% ActiveLanguage property.
	\textbf{+ \textcolor{red}{int} | \textcolor{darkgreen}{String} ActiveLanguage:} Contient la langue à
	chargée. Notez que les valeurs que vous \\enterez ici, devront prendre en charge les éléments qui seront
	générés dans le \textcolor{blue}{\textit{Output}}. Vous avez la possibilité de cibler une langue en
	donnant le nom correspondant à son fichier.\\\\
	% AutoGeneration property.
	\newpage \textbf{+ \textcolor{red}{bool} \hypertarget{autogen}{AutoGeneration} = \textcolor{red}
	{false}:} Souhaitez-vous générer de façon automatique les correspondances (.translation) du fichier
	(.csv) ? Notez que cette option, une fois activée, met en écoute l'édition du \\fichier source en vue de
	générer en temps réel, les équivalents (.translation). Vous ne serez donc plus tenu d'appuyer sur le
	bouton \textit{\hyperlink{genlanglfile}{GenerateLangFile(s)}} à chaque fois que vous progresserez dans
	l'édition de votre source. Toute fois, désactivé cette fonctionnalité avant chaque exécution du jeu pour
	éviter des écoutes inutiles. Le champ d'activité de cette propriété est uniquement sur le moteur Godot.
	\\\\
	% Interval property.
	\textbf{+ \textcolor{red}{float} Interval = \textcolor{blue}{3.0}:} Quel est le temps mort avant chaque 
	génération ? Notez que la valeur \\minimum de ce champ est de \textcolor{blue}{0.1} secondes et ne 
	s'active que lorsque l'option \textcolor{blue}{\textit{AutoGeneration}} est activé. Le champ d'activité 
	de cette propriété est uniquement sur le moteur Godot.\\\\
	% GeneratePassword property.
	\textbf{+ \textcolor{red}{bool} \hypertarget{genpass}{GeneratePassword} = \textcolor{red}{false}:}
	Souhaitez-vous générer automatiquement un mot de passe ? A ce niveau, la valeur du champ
	\textit{\hyperlink{pass}{Passs}} sera mise à jour à chaque génération. Le champ d'activité de cette
	propriété est uniquement sur le moteur Godot.\\\\
	% GenerateLangFile(s) property.
	\textbf{+ \textcolor{red}{bool} \hypertarget{genlangfile}{GenerateLangFile(s)} = \textcolor{red}
	{false}:} Bouton générant les équivalents (.translation) du fichier source (.csv). Le champ d'activité
	de cette propriété est uniquement sur le moteur Godot.

	% Languages méthods definition.
	\section{Les méthodes disponibles}
	% Void generate_passwords () method description.
	\begin{description}
		\item [+ \textcolor{red}{void} \textcolor{blue}{generate\_passwords} (delay = 0.0):] Génère au 
		hazard, un mot de passe en fonction des configurations présentes à son niveau. Le(s) élément(s) 
		contenant la clé \textit{\hyperlink{pass}{password}} seront affecté(s) par cette génération.
		Attention ! Cette méthode ne s'exécute qu'en mode édition.
		\begin{itemize}
			\item [>> \textbf{\textcolor{red}{float} delay}:] Quel est le temps mort avant la génération ?\\
		\end{itemize}
	\end{description}
	% Void generate_lang_files () method description.
	\begin{description}
		\item [+ \textcolor{red}{void} \textcolor{blue}{generate\_lang\_files} (delay = 0.0):] Génère les 
		équivalents (.translation) du fichier source (.csv) ayant été précisé. Attention ! Cette méthode ne
		s'exécute qu'en mode édition.
		\begin{itemize}
			\item [>> \textbf{\textcolor{red}{float} delay}:] Quel est le temps mort avant la génération ?\\
		\end{itemize}
	\end{description}
	% String get_value_at () method description.
	\begin{description}
		\item [+ \textcolor{darkgreen}{String} \textcolor{blue}{get\_value\_at} (key, default = "Null"):]
		Renvoie la valeur associée à un identifiant donné en fonction du langage actif.
		\begin{itemize}
			\item [>> \textbf{\textcolor{darkgreen}{String} key}:] Contient l'identificateur de la valeur à
			retournée.
			\item [>> \textbf{\textcolor{darkgreen}{String} default}:] Quelle valeur doit-on retournée en 
			cas de problème ?\\
		\end{itemize}
	\end{description}
	% Dictionary | String get_language_data () method description.
	\begin{description}
		\item [+ \textcolor{darkgreen}{Dictionary | String} \textcolor{blue}{get\_language\_data} (json =
		true):] Renvoie toutes les données associées au langage actif.
		\begin{itemize}
			\item [>> \textbf{\textcolor{red}{bool} json}:] Voulez-vous renvoyer le résultat sous le format 
			json ?\\
		\end{itemize}
	\end{description}
	% Bool has_keys () method description.
	\begin{description}
		\item [+ \textcolor{red}{bool} \textcolor{blue}{has\_keys} (keys):] Détermine si un ou plusieurs
		identifiant(s) sont bel et bien définient dans le gestionnaire de langue.
		\begin{itemize}
			\item [>> \textbf{\textcolor{darkgreen}{String | PoolStringArray} keys}:] Contient tous le(s)
			identifiant(s) à cherché(s).\\
		\end{itemize}
	\end{description}
	% PoolStringArray get_language_list () method description.
	\begin{description}
		\item [+ \textcolor{darkgreen}{PoolStringArray} \textcolor{blue}{get\_language\_list} ():] Renvoie
		les noms de toutes les langues prises en charge dans le module.\\
	\end{description}
	% Void reload_language () method description.
	\begin{description}
		\item [+ \textcolor{red}{void} \textcolor{blue}{reload\_language} (delay = 0.0):] Recharge dans le
		gestionnaire des langues du jeu, les \\données du language actif.
		\begin{itemize}
			\item[>> \textbf{\textcolor{red}{float} delay}:] Quel est le temps mort avant le rechargement ?
		\end{itemize}
	\end{description}

	% Languages events definition.
	\section{Les événements disponibles}
	% language_changed () signal description.
	\begin{description}
		\item [+ \textcolor{blue}{language\_changed} (data):] Signal déclenché lorsque les données de langue
		ont changées.
		\begin{itemize}
			\item [>> \textbf{\textcolor{darkgreen}{Dictionary} data}:] Contient les données du langage
			actuellement chargé en mémoire.\\
		\end{itemize}
	\end{description}
	% lang_file_cant_open () signal description.
	\begin{description}
		\item [+ \textcolor{blue}{lang\_file\_cant\_open} (data):] Signal déclenché lorsqu'on a du mal à
		ouvrir un fichier de langue ou que son accès à été refusé. Cet événement renvoie un dictionaire
		contenant les clés suivantes:
		\begin{itemize}
			\item [>> \textbf{\textcolor{darkgreen}{String} path}:] Contient le chemin pointant vers le
			fichier de langue.
			\item [>> \textbf{\textcolor{red}{int} type}:] Contient le type de l'erreur déclenché.
			\item [>> \textbf{\textcolor{darkgreen}{String} message}:] Contient le méssage renvoyé par 
			l'erreur déclenché.\\
		\end{itemize}
	\end{description}
	% lang_file_not_found () signal description.
	\begin{description}
		\item [+ \textcolor{blue}{lang\_file\_not\_found} (data):] Signal déclenché lorsqu'un fichier de
		langue n'est pas définit. Cet événement renvoie un dictionaire contenant les clés suivantes:
		\begin{itemize}
			\item [>> \textbf{\textcolor{darkgreen}{String} path}:] Contient le chemin pointant vers le
			fichier de langue.
			\item [>> \textbf{\textcolor{red}{int} type}:] Contient le type de l'erreur déclenché.
			\item [>> \textbf{\textcolor{darkgreen}{String} message}:] Contient le méssage renvoyé par 
			l'erreur déclenché.\\
		\end{itemize}
	\end{description}
	% lang_file_corrupted () signal description.
	\begin{description}
		\item [+ \textcolor{blue}{lang\_file\_corrupted} (data):] Signal déclenché lorsqu'un fichier de
		langue a été corromput de \\l'extérieur. Cet événement renvoie un dictionaire contenant les clés
		suivantes:
		\begin{itemize}
			\item [>> \textbf{\textcolor{darkgreen}{String} path}:] Contient le chemin pointant vers le
			fichier de langue.
			\item [>> \textbf{\textcolor{red}{int} type}:] Contient le type de l'erreur déclenché.
			\item [>> \textbf{\textcolor{darkgreen}{String} message}:] Contient le méssage renvoyé par 
			l'erreur déclenché.\\
		\end{itemize}
	\end{description}
	% lang_file_loading event description.
	\begin{description}
		\item [+ \textcolor{blue}{lang\_file\_loading} (data):] Signal déclenché pendant qu'un fichier de
		langue est en cours de \\chargement. Cet événement renvoie un dictionaire contenant les clés
		suivantes:
		\begin{itemize}
			\item [>> \textbf{\textcolor{darkgreen}{String} path}:] Contient le chemin pointant vers le
			fichier de langue.
			\item [>> \textbf{\textcolor{red}{bool} is\_over}:] Est-ce que toutes les données du fichier de
			langue ont-elles été \\complètement chargées ?
			\item [>> \textbf{\textcolor{red}{int} progress}:] Contient le nombre total de donnée(s) déjà
			chargée(s) en mémoire.\\
		\end{itemize}
	\end{description}
\end{document}