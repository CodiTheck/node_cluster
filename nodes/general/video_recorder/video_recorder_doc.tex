% Document type and package imports.
\documentclass[a4paper, 11pt]{article}
\usepackage[utf8]{inputenc}
\usepackage[T1]{fontenc}
\usepackage[french]{babel}
\usepackage{charter}
\usepackage[top = 2cm, bottom = 2cm, left = 1cm, right = 1cm]{geometry}
\usepackage{setspace}
\usepackage{color}
\usepackage{xcolor}
\usepackage{hyperref}
\usepackage{tocloft}

% Preanblue.
\onehalfspacing
\definecolor{gray}{rgb}{0.4, 0.4, 0.4}
\definecolor{silver}{rgb}{0.95, 0.95, 0.95}
\renewcommand{\thesection}{\Roman{section} --}
\definecolor{darkgreen}{HTML}{1E8C15}
\cftsetindents{section}{1em}{2.5em}
\hypersetup {colorlinks=true, linkcolor=blue, urlcolor=blue, pdftitle={VideoRecorder module doc}}

% The start of the article.
\begin{document}
	% Change document background to silver color.
	\pagecolor{silver}
	% VideoRecorder module description.
	\huge{\hspace{13cm}\textit{\textbf{\textcolor{darkgreen}{VideoRecorderFx}}}}\large{} \tableofcontents 
	\newpage
	% VideoRecorder module definition.
	\section{Définition}
	\textcolor{darkgreen}{\textbf{VideoRecorderFx}} est un module conçut pour la gestion des enregistrements 
	des captures d'écran dans un jeu vidéo.\\
	\textcolor{red}{\textbf{NB}:} Ce module est de nature indestructible, est compatible à un jeu 2D, 3D et
	est sauvegardable.

	% VideoRecorder properties definition.
	\section{Les propriétés disponibles}
	% Mode property.
	\textbf{+ \textcolor{red}{int} Mode = \textcolor{blue}{0}:} Contient les différents modes possibles que
	supporte ce module. Les valeurs possibles sont:
	\begin{itemize}
		\item[-> \textbf{\textcolor{gray}{VideoRecorderFx.Model.RECORDER} ou \textcolor{blue}{0}}:] Mode
		diasporama.
		\item[-> \textbf{\textcolor{gray}{VideoRecorderFx.Model.READER} ou \textcolor{blue}{1}}:] Mode
		lecture de séquence.\\
	\end{itemize}
	% BufferOptimization property.
	\textbf{+ \textcolor{red}{bool} BufferOptimization = \textcolor{red}{false}:} Souhaitez-vous supprimer 
	les données de toutes les séquences précédements lues avant la lecture d'une nouvelle séquence ?
	N'utilisez cette propriété que si le module est utilisé en tant que lecteur de séquence(s).\\\\
	% Sync property.
	\textbf{+ \textcolor{red}{bool} Sync = \textcolor{red}{false}:} La lecture des séquences s'éffectuera
	t-elle de façon synchrône ? N'utilisez cette propriété que si le module est utilisé comme un lecteur.\\
	\\
	% Count property.
	\textbf{+ \textcolor{red}{int} Count = \textcolor{blue}{1}:} Combien de fois la liste des séquences sera 
	lue ? N'utilisez cette propriété que si le module est utilisé comme un lecteur.\\\\
	% Direction property.
	\textbf{+ \textcolor{red}{int} \hypertarget{direction}{Direction} = \textcolor{blue}{0}:} Contient les 
	différents sens possibles de lecture ou d'écriture que supporte ce module. Les valeurs possibles sont:
	\begin{itemize}
		\item[-> \textbf{\textcolor{gray}{MegaAssets.Orientation.NORMAL} ou \textcolor{blue}{0}}:] Sens
		normal de lecture ou d'enregistrement des \\séquences.
		\item[-> \textbf{\textcolor{gray}{MegaAssets.Orientation.REVERSED} ou \textcolor{blue}{1}}:] Sens
		inverse de lecture ou d'enregistrement des \\séquences.
		\item[-> \textbf{\textcolor{gray}{MegaAssets.Orientation.RANDOM} ou \textcolor{blue}{2}}:] Choix
		aléatoire de lecture ou d'enregistrement.\\
	\end{itemize}
	% Action property.
	\textbf{+ \textcolor{red}{int} Action = \textcolor{blue}{0}:} Contient les différentes actions possibles
	qu'on peut éffectuées sur les séquences d'images. Les valeurs possibles sont:
	\begin{itemize}
		\item[-> \textbf{\textcolor{gray}{MegaAssets.MediaState.NONE} ou \textcolor{blue}{0}}:] Aucune
		action ne sera éffectuée.
		\item[-> \textbf{\textcolor{gray}{MegaAssets.MediaState.PLAY} ou \textcolor{blue}{1}}:] Joue toutes
		les séquences disponibles sur le module.
		\item[-> \textbf{\textcolor{gray}{MegaAssets.MediaState.PAUSE} ou \textcolor{blue}{2}}:] Suspend
		toutes les séquences en cours de lecture.
		\item[-> \textbf{\textcolor{gray}{MegaAssets.MediaState.STOP} ou \textcolor{blue}{3}}:] Stop toutes
		les séquences en cours de lecture.\\
	\end{itemize}
	% Sequences property.
	\textbf{+ \textcolor{darkgreen}{Array} Sequences}: Tableau de dictionnaires contenant les différentes
	configurations de chaque \\séquence d'images prises en charge par le développeur. Les clés que
	supportent les dictionnaires sont:
	\begin{itemize}
		\item[>> \textbf{\textcolor{red}{int} path = \textcolor{blue}{0}}:] Contient les différents chemins 
		que prend en charge ce module. Ces chemins \\représentent les endroits possibles où l'on peut déposé 
		le(s) séquence(s) d'images du jeu. Les valeurs possibles sont:
		\begin{itemize}
			\item[-> \textbf{\textcolor{gray}{MegaAssets.Path.GAME\_LOCATION} ou \textcolor{blue}{0}}:] 
			Cible le dossier racine du jeu.
			\item[-> \textbf{\textcolor{gray}{MegaAssets.Path.OS\_ROOT} ou \textcolor{blue}{1}}:] Cible le 
			dossier racine du système d'exploitation installé.
			\item[-> \textbf{\textcolor{gray}{MegaAssets.Path.USER\_DATA} ou \textcolor{blue}{2}}:] Cible le 
			dossier racine des données de l'utilisateur.
			\item[-> \textbf{\textcolor{gray}{MegaAssets.Path.USER\_ROOT} ou \textcolor{blue}{3}}:] Cible le 
			dossier racine de l'utilisateur.
			\item[-> \textbf{\textcolor{gray}{MegaAssets.Path.USER\_DESKTOP} ou \textcolor{blue}{4}}:] Cible 
			le bureau du système d'exploitation.
			\item[-> \textbf{\textcolor{gray}{MegaAssets.Path.USER\_PICTURES} ou \textcolor{blue}{5}}:] 
			Cible le dossier \textcolor{gray}{\textit{Images}} du système d'exploitation.
			\item[-> \textbf{\textcolor{gray}{MegaAssets.Path.USER\_MUSIC} ou \textcolor{blue}{6}}:] Cible 
			le dossier \textcolor{gray}{\textit{Musiques}} du système d'exploitation.
			\item[-> \textbf{\textcolor{gray}{MegaAssets.Path.USER\_VIDEOS} ou \textcolor{blue}{7}}:] Cible 
			le dossier \textcolor{gray}{\textit{Vidéos}} du système d'exploitation.
			\item[-> \textbf{\textcolor{gray}{MegaAssets.Path.USER\_DOCUMENTS} ou \textcolor{blue}{8}}:] 
			Cible le dossier \textcolor{gray}{\textit{Documents}} du système \\d'exploitation.
			\item[-> \textbf{\textcolor{gray}{MegaAssets.Path.USER\_DOWNLOADS} ou \textcolor{blue}{9}}:] 
			Cible le dossier \textcolor{gray}{\textit{Téléchargements}} du système \\d'exploitation.\\
		\end{itemize}
		\item[>> \textbf{\textcolor{darkgreen}{String} source}:] Contient le chemin pointant vers un fichier 
		ou un dossier se trouvant sur le disque dure. Evitez les répétitions, car cela n'est pas tolérées.
		Notez que si vous cryptez le(s) séquence(s), vous devez préciser le(s) fichier(s) qui contiendra/ont 
		les différentes données enregistrez au cours des séances de capture d'écran. Dans le cas contraire,
		on considéra que la source référée pointe toujours vers un dossier. L'utilisation de cette clé est
		obligatoire.\\
		\item[>> \textbf{\textcolor{red}{int} timeout = \textcolor{blue}{0.0}}:] Quel est le temps mort
		avant le démarrage de la séquence en question ?\\
		\item[>> \textbf{\textcolor{darkgreen}{Vector2} resolution = \textcolor{darkgreen}{Vector2}
		(\textcolor{blue}{-1}, \textcolor{blue}{-1})}:] Contrôle la taille des différentes images de la
		séquence en question.\\
		\item[>> \textbf{\textcolor{red}{int} quality = \textcolor{blue}{2}}:] Contrôle la qualité des 
		différentes images de la séquence en question. Les valeurs possibles sont celles définient au sein
		de la classe \href{https://docs.godotengine.org/en/stable/classes/class_image.html}
		{\textit{\textcolor{darkgreen}{Image}}} de Godot.\\
		\item[>> \textbf{\textcolor{red}{bool} encrypted = \textcolor{red}{false}}:] Devons nous crypter la
		séquence ? N'utilisez cette option que si vous \\voulez éffectuer un enregistrement.\\
		\item[>> \textbf{\textcolor{red}{bool} audio = \textcolor{red}{false}}:] Devons nous enregister tous 
		les sons à notre portés. Dans ce cas, la séquence sera enregistrer avec son propre fichier audio.
		Assurez-vous de la présence du module \textit{\textcolor{darkgreen}{\\AudioRecorderFx}} avant 
		d'activer cette option. Notez que si le module est utilisé comme un lecteur, cette option aura pour
		éffet d'activer et de désactiver tout simplement le son en cours de lecture.\\
		\item[>> \textbf{\textcolor{red}{int | float} fps = 60}:] Combien d'images seront enregistrées ou 
		lue en une second ?\\
	 	\item[>> \textbf{\textcolor{red}{int} repeat = 1}:] Combien de fois la même séquence sera lue ?
	 	N'utilisez cette clé que si le module utilisé comme un lecteur.\\
		\item[>> \textbf{\textcolor{red}{float} duration = \textcolor{blue}{0.0}}:] Quelle est la durée
		de l'enregistrement ou de la lecture de la séquence ?\\
		\newpage \item[>> \textbf{\textcolor{red}{int} save = \textcolor{blue}{0}}:] Souhaitez-vous définir
		un mode de sauvegarde au cours de l'enregistrement d'une séquence ? N'utilisez cette propriété que
		si le module est employé comme un enregistreur. Les valeurs possibles sont:
		\begin{itemize}
			\item[-> \textbf{\textcolor{gray}{VideoRecorderFx.SaveMode.NONE} ou \textcolor{blue}{0}}:]
			Aucune sauvegarde ne sera éffectuer (ni en \\mémoire, ni dans un fichier).
			\item[-> \textbf{\textcolor{gray}{VideoRecorderFx.SaveMode.MEMORY} ou \textcolor{blue}{1}}:] La 
			séquence sera sauvegarder en mémoire \\uniquement.
			\item[-> \textbf{\textcolor{gray}{VideoRecorderFx.SaveMode.FILE} ou \textcolor{blue}{2}}:] La 
			séquence sera sauvegarder dans un fichier ou \\ensemble de fichiers.\\
		\end{itemize}
		\item[>> \textbf{\textcolor{red}{int} load = \textcolor{blue}{0}}:] Souhaitez-vous définir un mode
		de sauvegarde au cours de l'enregistrement d'une séquence ? N'utilisez cette propriété que si le
		module est employé comme un lecteur. Les valeurs possibles sont:
		\begin{itemize}
			\item[-> \textbf{\textcolor{gray}{VideoRecorderFx.SaveMode.NONE} ou \textcolor{blue}{0}}:]
			Aucun chargement ne sera éffectuer (ni dans la mémoire, ni à partir d'un fichier).
			\item[-> \textbf{\textcolor{gray}{VideoRecorderFx.SaveMode.MEMORY} ou \textcolor{blue}{1}}:] La 
			séquence sera charger en mémoire avant d'être lue.
			\item[-> \textbf{\textcolor{gray}{VideoRecorderFx.SaveMode.FILE} ou \textcolor{blue}{2}}:] La 
			séquence sera directement lue à partir d'un \\fichier ou ensemble de fichiers.\\
		\end{itemize}
		\item[>> \textbf{\textcolor{darkgreen}{Array | Dictionary} started}:] Signal déclenché à chaque fois 
		qu'on démarre immédiatement la \\séquence. Cette clé exécute les différentes actions données à son 
		déclenchement. Pour soumettre les actions à exécutées référez vous à la méthode utilisée au niveau
		de la clé \textit{\textcolor{gray}{actions}} de la propriété \textit{\textcolor{gray}
		{EventsBindings}} dans les bases du framework.\\
		\item[>> \textbf{\textcolor{darkgreen}{Array | Dictionary} finished}:] Signal déclenché à chaque 
		fois que la séquence a terminée son \\exécution. Cette clé exécute les différentes actions données à 
		son déclenchement. Pour soumettre les actions à exécutées référez vous à la méthode utilisée au
		niveau de la clé \textit{\textcolor{gray}{actions}} de la propriété \textit{\textcolor{gray}
		{EventsBindings}} dans les bases du framework.\\
		\item[>> \textbf{\textcolor{darkgreen}{Array | Dictionary} playing}:] Signal déclenché pendant que
		la séquence est en cours d'exécution. Cette clé exécute les différentes actions données à son
		déclenchement. Pour soumettre les actions à exécutées référez vous à la méthode utilisée au niveau
		de la clé \textit{\textcolor{gray}{actions}} de la propriété \textit{\textcolor{gray}
		{\\EventsBindings}} dans les bases du framework.\\
	\end{itemize}
	\textcolor{red}{\textbf{NB}:} Gardez à l'esprit que ce module se sert d'un curseur pour sélectionner les
	séquences définient par le développeur. Ce curseur n'est rien d'autre que l'index de position de chaque 
	configuration de séquence. Par défaut, sa valeur est \textbf{\textcolor{blue}{0}} lorsqu'il y a une ou
	plusieurs séquence(s) configurée(s) sur le module et \textbf{\textcolor{blue}{-1}} si aucune séquence 
	n'est disponible sur ce dernier.

	% VideoRecorder methods definition.
	\section{Les méthodes disponibles}
	% ImageTexture get_current_frame () method description.
	\begin{description}
		\item [+ \textcolor{darkgreen}{ImageTexture} \textcolor{blue}{get\_current\_frame} (id = null):]
		Renvoie l'image de la frame actuelle de la \\séquence en cours d'enregistrement ou de lecture. Si
		aucun identifiant n'a été référé, la valeur du curseur sera utilisée pour éffectuer le traitement
		demandé.
		\begin{itemize}
			\item [>> \textbf{\textcolor{red}{int} | \textcolor{darkgreen}{String} id}:] Quel est
			l'identifiant de la séquence à ciblée ?\\
		\end{itemize}
	\end{description}
	% Int get_current_frame_index () method description.
	\begin{description}
		\item [+ \textcolor{red}{int} \textcolor{blue}{get\_current\_frame\_index} (id = null):] Renvoie la
		position de la frame actuelle de la \\séquence en cours d'enregistrement ou de lecture. Si aucun
		identifiant n'a été référé, la valeur du curseur sera utiliser pour éffectuer le traitement demandé.
		\begin{itemize}
			\item [>> \textbf{\textcolor{red}{int} | \textcolor{darkgreen}{String} id}:] Quel est
			l'identifiant de la séquence à ciblée ?\\
		\end{itemize}
	\end{description}
	% Int get_sequence_progress () method description.
	\begin{description}
		\item [+ \textcolor{red}{int} \textcolor{blue}{get\_sequence\_progress} (id = null):] Renvoie la
		progression actuelle d'une séquence (lecture ou enregistrement). Si aucun identifiant n'a été
		référé, celle de l'ensemble des séquences sera renvoyer. La valeur de la progression est dans 
		l'intervalle [\textcolor{blue}{0}; \textcolor{blue}{100}].
		\begin{itemize}
			\item [>> \textbf{\textcolor{red}{int} | \textcolor{darkgreen}{String} id}:] Quel est
			l'identifiant de la séquence à ciblée ?\\
		\end{itemize}
	\end{description}
	% Float get_normalized_sequence () method description.
	\begin{description}
		\item [+ \textcolor{red}{int} \textcolor{blue}{get\_normalized\_sequence} (id = null):] Renvoie la
		version normalisée de la progression \\actuelle d'une séquence (lecture ou enregistrement). Si aucun 
		identifiant n'a été référé, celle de l'ensemble des séquences sera renvoyer. La valeur de la
		progression est dans l'intervalle [\textcolor{blue}{0.0}; \textcolor{blue}{1.0}].
		\begin{itemize}
			\item [>> \textbf{\textcolor{red}{int} | \textcolor{darkgreen}{String} id}:] Quel est
			l'identifiant de la séquence à ciblée ?\\
		\end{itemize}
	\end{description}
	% Void play () method description.
	\begin{description}
		\item [+ \textcolor{red}{void} \textcolor{blue}{play} (id = null, config = \{\}, interval = 0.0):]
		Exécute les configurations d'une ou de \\plusieurs séquence(s). Si aucun identifiant n'a été référé,
		la valeur du curseur sera utiliser pour éffectuer le traitement demandé.
		\begin{itemize}
			\item [>> \textbf{\textcolor{red}{int} | \textcolor{darkgreen}{PoolIntArray | String |
			PoolStringArray} id}:] Quel(s) est/sont le(s) identifiant(s) de(s) séquence(s) à enregistrée(s) 
			ou à lire ?
			\item [>> \textbf{\textcolor{darkgreen}{Dictionary | Array} config}:] Voulez-vous changer la 
			valeur de certaines clés de(s) sequence(s) avant son/leur exécution ? Si vous donnez un tableau, 
			alors il ne devra que contenir des \\dictionaires.
			\item [>> \textbf{\textcolor{red}{float} interval}:] Quel est le temps mort avant l'exécution de 
			chaque séquence ?\\
		\end{itemize}
	\end{description}
	% Void pause () method description.
	\begin{description}
		\item [+ \textcolor{red}{void} \textcolor{blue}{pause} (id = null, interval = 0.0):] Suspend
		l'exécution d'une ou de plusieurs séquence(s). Si aucun identifiant n'a été référé, la valeur du
		curseur sera utiliser pour éffectuer le traitement demandé.
		\begin{itemize}
			\item [>> \textbf{\textcolor{red}{int} | \textcolor{darkgreen}{PoolIntArray | String |
			PoolStringArray} id}:] Quel(s) est/sont le(s) identifiant(s) de(s) séquence(s) à suspendre ?
			\item [>> \textbf{\textcolor{red}{float} interval}:] Quel est le temps mort avant la suspension 
			de chaque séquence ?\\
		\end{itemize}
	\end{description}
	% Void stop () method description.
	\newpage \begin{description}
		\item [+ \textcolor{red}{void} \textcolor{blue}{stop} (id = null, interval = 0.0):] Arrète
		l'exécution d'une ou de plusieurs séquence(s). Si aucun identifiant n'a été référé, la valeur du
		curseur sera utiliser pour cibler la séquence à arrètée.
		\begin{itemize}
			\item [>> \textbf{\textcolor{red}{int} | \textcolor{darkgreen}{PoolIntArray | String |
			PoolStringArray} id}:] Quel(s) est/sont le(s) identifiant(s) de(s) séquence(s) à arrètée(s) ?
			\item [>> \textbf{\textcolor{red}{float} interval}:] Quel est le temps mort avant l'arrèt de 
			chaque séquence ?\\
		\end{itemize}
	\end{description}
	% Float get_eleapsed_time () method description.
	\begin{description}
		\item [+ \textcolor{red}{float} \textcolor{blue}{get\_eleapsed\_time} (id = null):] Retourne le 
		temps écoulé depuis l'enregistrement ou la lecture d'une séquence. Si aucun identifiant n'a été 
		précisé, celui de l'ensemble des séquences sera renvoyer.
		\begin{itemize}
			\item [>> \textbf{\textcolor{red}{int} | \textcolor{darkgreen}{String} id}:] Quel est
			l'identifiant de l'animation à ciblée ?\\
		\end{itemize}
	\end{description}
	% Int get_state () method description.
	\begin{description}
		\item [+ \textcolor{red}{int} \textcolor{blue}{get\_state} (id = null):] Renvoie l'état 
		d'utilisation d'une séquence donnée. Si aucun identifiant n'a été précisé, le traitement sera 
		éffectué sur l'ensemble des séquences du module. Les valeurs possibles de retour sont:
		\begin{itemize}
			\item[-> \textbf{\textcolor{gray}{MegaAssets.MediaState.NONE} ou \textcolor{blue}{0}}:] Aucun
			traitement ou séquence arrètée.
			\item[-> \textbf{\textcolor{gray}{MegaAssets.MediaState.PLAY} ou \textcolor{blue}{1}}:] Séquence 
			en cours d'exécution.
			\item[-> \textbf{\textcolor{gray}{MegaAssets.MediaState.PAUSE} ou \textcolor{blue}{2}}:]
			Séquence suspendu.
			\item[-> \textbf{\textcolor{gray}{MegaAssets.MediaState.LOOP} ou \textcolor{blue}{4}}:] Séquence 
			en exécution infinie.
		\end{itemize}
		\begin{itemize}
			\item [>> \textbf{\textcolor{red}{int} | \textcolor{darkgreen}{String} id}:] Quel est
			l'identifiant de la séquence à ciblée ?\\
		\end{itemize}
	\end{description}
	% Int get_current_repeatition () method description.
	\begin{description}
		\item [+ \textcolor{red}{int} \textcolor{blue}{get\_current\_repeatition} (id = null):] Renvoie le 
		nombre actuelle de répétitions éffectuées \\auprès d'une séquence donnée. Si aucun identifiant n'a 
		été renseigné, celui de l'ensemble des séquences sera renvoyer.
		\begin{itemize}
			\item [>> \textbf{\textcolor{red}{int} id}:] Quel est l'identifiant de la séquence à ciblée ?\\
		\end{itemize}
	\end{description}
	% Int get_cursor () method description.
	\begin{description}
		\item [+ \textcolor{red}{int} \textcolor{blue}{get\_cursor} ():] Renvoie la valeur actuelle du 
		curseur du module.\\
	\end{description}
	% Void set_cursor () method description.
	\begin{description}
		\item [+ \textcolor{red}{void} \textcolor{blue}{set\_cursor} (new\_value):] Change la valeur 
		actuelle du curseur du module.
		\begin{itemize}
			\item [>> \textbf{\textcolor{red}{int} new\_value}:] Quel est la nouvelle valeur du curseur ?\\
		\end{itemize}
	\end{description}
	% Int | PoolIntArray get_active_seq_index () method description.
	\begin{description}
		\item [+ \textcolor{red}{int} | \textcolor{darkgreen}{PoolIntArray} \textcolor{blue}
		{get\_active\_seq\_index} ():] Renvoie le(s) position(s) de la ou des séquence(s) en cours 
		d'exécution.\\
	\end{description}
	% Int get_prev_seq_index () method description.
	\begin{description}
		\item [+ \textcolor{red}{int} \textcolor{blue}{get\_prev\_seq\_index} ():] Renvoie la position de 
		la séquence précédement exécutée ou \\générée.\\
	\end{description}
	% Int get_next_seq_index () method description.
	\begin{description}
		\item [+ \textcolor{red}{int} \textcolor{blue}{get\_next\_seq\_index} ():] Renvoie la position du
		future séquence à exécutée. N'appelée cette méthode que si la valeur du champ
		\textit{\hyperlink{direction}{Direction}} est sur \textit{\textcolor{gray}{RANDOM}}.\\
	\end{description}
	% Int get_big_seq_index () method description.
	\begin{description}
		\item [+ \textcolor{red}{int} \textcolor{blue}{get\_big\_seq\_index} ():] Renvoie la position de
		la séquence ayant la plus grande durée \\parmit celles définient par le développeur.\\
	\end{description}
	% Int get_small_seq_index () method description.
	\begin{description}
		\item [+ \textcolor{red}{int} \textcolor{blue}{get\_small\_seq\_index} ():] Renvoie la position de 
		la séquence ayant la plus petite durée \\parmit celles définient par le développeur.\\
	\end{description}
	% Float get_total_duration () method description.
	\begin{description}
		\item [+ \textcolor{red}{float} \textcolor{blue}{get\_total\_duration} ():] Renvoie le temps total
		des séquences disponibles sur le module (sans les délai).\\
	\end{description}
	% Float get_total_timeout () method description.
	\begin{description}
		\item [+ \textcolor{red}{float} \textcolor{blue}{get\_total\_timeout} ():] Renvoie la somme des 
		délai définient sur les séquences du module.\\
	\end{description}
	% Float get_total_time () method description.
	\begin{description}
		\item [+ \textcolor{red}{float} \textcolor{blue}{get\_total\_time} ():] Renvoie le temps total
		des séquences disponibles sur le module (avec les délai).\\
	\end{description}
	% Void save_sequences () method description.
	\begin{description}
		\item [+ \textcolor{red}{void} \textcolor{blue}{save\_sequences} (id = null):] Sauvegarde dans un 
		fichier ou ensemble de fichiers, une ou plusieurs séquence(s) déjà enregistrée(s) en mémoire. 
		N'utilisez cette méthode que si le module est utilisé comme en enregistreur.
		\begin{itemize}
			\item [>> \textbf{\textcolor{red}{int} | \textcolor{darkgreen}{String | String | 
			PoolStringArray} id}:] Quel est l'identifiant de(s) séquence(s) à ciblée(s) ?\\
		\end{itemize}
	\end{description}
	% Void load_sequences () method description.
	\begin{description}
		\item [+ \textcolor{red}{void} \textcolor{blue}{load\_sequences} (id = null):] Charge une ou
		plusieurs séquence(s) déjà enregistrée(s) sur le disque dure. N'utilisez cette méthode que si le
		module est utilisé comme en lecteur.
		\begin{itemize}
			\item [>> \textbf{\textcolor{red}{int} | \textcolor{darkgreen}{String | String | 
			PoolStringArray} id}:] Quel est l'identifiant de(s) séquence(s) à ciblée(s) ?\\
		\end{itemize}
	\end{description}
	% Dictionary get_sequences_data () method description.
	\begin{description}
		\item [+ \textcolor{darkgreen}{Dictionary} \textcolor{blue}{get\_sequences\_data} (json = false):] 
		Renvoie toutes les données sur les séquences disponibles au sein du module.
		\begin{itemize}
			\item [>> \textbf{\textcolor{red}{bool} json}:] Voulez-vous renvoyer les données au format json 
			?
		\end{itemize}
	\end{description}

	% VideoRecorder signals definition.
	\section{Les événements disponibles}
	% sequence_changed event description.
	\begin{description}
		\item [+ \textcolor{blue}{sequence\_changed} (data):] Signal déclenché lorsque la séquence en cours 
		d'enregistrement ou de lecture a changée. Notez que cet événement ne s'appel que lorsque le mode 
		de lecture des séquences du module est synchrône. Il renvoie un dictionaire contenant les clés 
		suivantes:
		\begin{itemize}
			\item [>> \textbf{\textcolor{red}{int} preview}:] Contient la position de la séquence 
			précédement exécutée.
			\item [>> \textbf{\textcolor{red}{int} current}:] Contient l'index de position de la séquence
			actuellement en cours d'exécution.\\
		\end{itemize}
	\end{description}
	% sequence_started event description.
	\begin{description}
		\item [+ \textcolor{blue}{sequence\_started} (data):] Signal déclenché à chaque fois qu'on démarre 
		immédiatement une séquence. Cet événement renvoie un dictionaire contenant les clés suivantes:
		\begin{itemize}
			\item [>> \textbf{\textcolor{red}{int} index}:] Contient la position de la séquence en question.
			\\
		\end{itemize}
	\end{description}
	% sequence_finished event description.
	\begin{description}
		\item [+ \textcolor{blue}{sequence\_finished} (data):] Signal déclenché à chaque fois qu'une 
		séquence a terminée son \\exécution (enregistrement ou lecture). Cet événement renvoie un 
		dictionaire contenant les clés suivantes:
		\begin{itemize}
			\item [>> \textbf{\textcolor{red}{int} index}:] Contient la position de la séquence en question.
			\\
		\end{itemize}
	\end{description}
	% sequence_running event description.
	\begin{description}
		\item [+ \textcolor{blue}{sequence\_running} (data):] Signal déclenché pendant qu'une séquence est 
		en cours \\d'enregistrement ou de lecture. Cet événement renvoie un dictionaire contenant les clés 
		\\suivantes:
		\begin{itemize}
			\item [>> \textbf{\textcolor{red}{float} time}:] Contient le temps écoulé depuis l'exécution de
			la séquence.
			\item [>> \textbf{\textcolor{red}{int} progress}:] Contient la progression actuelle de la
			séquence.
			\item [>> \textbf{\textcolor{red}{float} normalized}:] Contient la progression normalisée de la
			séquence.
			\item [>> \textbf{\textcolor{red}{int} count}:] Contient le nombre total de répétitions 
			éffectuées depuis la première exécution. Cependant, si la séquence est exécutée à l'infinie, 
			vous aurez une valeur négative.
			\item [>> \textbf{\textcolor{red}{int} index}:] Contient la position de la séquence en question.
			\item [>> \textbf{\textcolor{darkgreen}{ImageTexture} frame}:] Contient la frame actuelle de la.
			séquence.
			\item [>> \textbf{\textcolor{red}{int} frame\_index}:] Contient la position de la frame de la
			séquence.\\
		\end{itemize}
	\end{description}
	% list_started event description.
	\begin{description}
		\item [+ \textcolor{blue}{list\_started} ():] Signal déclenché à chaque fois qu'une ou plusieurs 
		séquence(s) sont en cours de traitement.
	\end{description}
	% list_finished event description.
	\begin{description}
		\item [+ \textcolor{blue}{list\_finished} ():] Signal déclenché à chaque fois qu'aucune séquence 
		n'est en cours de \\traitement.
	\end{description}
	% list_running event description.
	\begin{description}
		\item [+ \textcolor{blue}{list\_running} (data):] Signal déclenché pendant qu'une ou plusieurs 
		séquence(s) sont en cours de traitement. Cet événement renvoie un dictionaire contenant les clés 
		suivantes:
		\begin{itemize}
			\item [>> \textbf{\textcolor{red}{float} time}:] Contient le temps écoulé depuis l'exécution de
			l'ensemble des séquences du module.
			\item [>> \textbf{\textcolor{red}{int} progress}:] Contient la progression actuelle de 
			l'ensemble des séquences du module.
			\item [>> \textbf{\textcolor{red}{float} normalized}:] Contient la progression normalisée de 
			l'ensemble des séquences du module.
			\item [>> \textbf{\textcolor{red}{int} count}:] Contient le nombre total de répétitions 
			éffectuées depuis la première exécution. \\Cependant, si la liste des séquences est exécutée à 
			l'infinie, vous aurez une valeur négative.\\
		\end{itemize}
	\end{description}
	% timeout event description.
	\begin{description}
		\item [+ \textcolor{blue}{timeout} (data):] Signal déclenché à chaque fois qu'un temps mort est
		requis avant exécuter une séquence. Cet événement renvoie un dictionaire contenant les clés
		suivantes:
		\begin{itemize}
			\item [>> \textbf{\textcolor{red}{int} preview}:] Contient la position de la séquence
			précédente.
			\item [>> \textbf{\textcolor{red}{int} current}:] Contient la position de la séquence future.\\
		\end{itemize}
	\end{description}
	% sequence_loading event description.
	\begin{description}
		\item [+ \textcolor{blue}{sequence\_loading} (data):] Signal déclenché pendant qu'une séquence est 
		en cours de \\chargement. Cet événement renvoie un dictionaire contenant les clés suivantes:
		\begin{itemize}
			\item [>> \textbf{\textcolor{red}{int} index}:] Contient la position de la séquence en cours de
			chargement.
			\item [>> \textbf{\textcolor{red}{bool} is\_over}:] Est-ce que toutes les données de la séquence
			ont-elles été complètement \\chargées ?
			\item [>> \textbf{\textcolor{red}{int} progress}:] Contient le nombre total actuel de donnée(s)
			chargée(s) en mémoire.\\
		\end{itemize}
	\end{description}
	% sequence_saving event description.
	\begin{description}
		\item [+ \textcolor{blue}{sequence\_saving} (data):] Signal déclenché pendant qu'une séquence est 
		en cours de sauvegarde. Cet événement renvoie un dictionaire contenant les clés suivantes:
		\begin{itemize}
			\item [>> \textbf{\textcolor{red}{int} index}:] Contient la position de la séquence en question.
			\item [>> \textbf{\textcolor{red}{int} progress}:] Contient la progression actuelle en
			pourcentage de la sauvegarde.\\
		\end{itemize}
	\end{description}
	% sequence_error () signal description.
	\begin{description}
		\item [+ \textcolor{blue}{sequence\_error} (index):] Signal déclenché lorsque la séquence à chargée
		n'est pas définit ou que son chargement à échoué.
		\begin{itemize}
			\item [>> \textbf{\textcolor{red}{int} index}:] Contient la position de la séquence endommagée.
			\\
		\end{itemize}
	\end{description}
	% sequence_corrupted () signal description.
	\begin{description}
		\item [+ \textcolor{blue}{sequence\_corrupted} (index):] Signal déclenché lorsque la séquence à
		chargée a été corromput de l'extérieur.
		\begin{itemize}
			\item [>> \textbf{\textcolor{red}{int} index}:] Contient la position de la séquence corromput.
		\end{itemize}
	\end{description}
\end{document}