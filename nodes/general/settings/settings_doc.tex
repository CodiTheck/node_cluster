% Document type and package imports.
\documentclass[a4paper, 11pt]{article}
\usepackage[utf8]{inputenc}
\usepackage[T1]{fontenc}
\usepackage[french]{babel}
\usepackage{charter}
\usepackage{hyperref}
\usepackage{tocloft}
\usepackage[top = 2cm, bottom = 2cm, left = 1cm, right = 1cm]{geometry}
\usepackage{setspace}
\usepackage{color}
\usepackage{xcolor}

% Preanblue.
\onehalfspacing
\definecolor{gray}{rgb}{0.4, 0.4, 0.4}
\definecolor{silver}{rgb}{0.95, 0.95, 0.95}
\renewcommand{\thesection}{\Roman{section} --}
\definecolor{darkgreen}{HTML}{1E8C15}
\cftsetindents{section}{1em}{2.5em}
\hypersetup {colorlinks=true, linkcolor=blue, urlcolor=blue, pdftitle={Settings module doc}}

% The start of the article.
\begin{document}
	% Change document background to silver color.
	\pagecolor{silver}
	% Settings module description.
	\huge{\hspace{15cm}\textit{\textbf{\textcolor{darkgreen}{SettingsFx}}}}\large{} \tableofcontents 
	\newpage
	% Settings module definition.
	\section{Définition}
	\textcolor{darkgreen}{\textbf{SettingsFx}} est un module conçut pour la gestion des configurations 
	globales propres à un jeu. Il rapporte en majorité les paramètres d'un jeu dans Godot. La plupard des 
	champs que possède ce module ne sont rien d'autre que les configurations de Godot. Ce qui est présent 
	sur ce module rassemble les configurations les plus utilisées.\\
	\textcolor{red}{\textbf{NB}:} Ce module est de nature indestructible, est compatible à un jeu 2D, 3D et 
	est sauvegardable.

	% Settings properties definition.
	\section{Les propriétés disponibles}
	% DisableAudio property.
	\textbf{+ \textcolor{red}{bool} DisableAudio = \textcolor{red}{false}:} Souhaitez-vous désactiver les
	éffets sonores du jeu ?\\\\
	% AudioVolume property.
	\textbf{+ \textcolor{red}{float} AudioVolume = \textcolor{blue}{10.0}:} Contrôle le volume des éffets 
	sonores du jeu. Ces valeurs sont dans \\l'intervalle de [\textcolor{blue}{0.0}; \textcolor{blue}
	{100.0}].Notez qu'il faut que les éffets sonores doivent être permit avant de pouvoir varier leur 
	volume.\\\\
	% GameQuality property.
	\textbf{+ \textcolor{red}{int} GameQuality = \textcolor{blue}{0}:} Contrôle le degré de qualité du jeu.
	Les valeurs possibles sont:
	\begin{itemize}
		\item [-> \textbf{\textcolor{gray}{MegaAssets.GameGrade.LOW} ou \textcolor{blue}{0}}:] Mauvaise 
		qualité.
		\item [-> \textbf{\textcolor{gray}{MegaAssets.GameGrade.MEDIUM} ou \textcolor{blue}{1}}:] Qualité 
		moyen.
		\item [-> \textbf{\textcolor{gray}{MegaAssets.GameGrade.HIGH} ou \textcolor{blue}{2}}:] Bonne 
		qualité.\\
	\end{itemize}
	% ScreenOrientation property.
	\textbf{+ \textcolor{red}{int} ScreenOrientation = \textcolor{blue}{0}:} Voire la documentation de Godot 
	sur la propriété
	\href{https://docs.godotengine.org/en/stable/classes/class_os.html#class-os-property-screen-orientation}
	{\textit{\textcolor{blue}{ScreenOrientation}}}.\\\\
	% Borderless property.
	\textbf{+ \textcolor{red}{bool} Borderless = \textcolor{red}{false}:} Voulez-vous supprimer le cadre 
	définissant la fenêtre du jeu ? En d'autres termes, souhaitez-vous supprimer la barre de titre ?\\\\
	% Foreground property.
	\textbf{+ \textcolor{red}{bool} Foreground = \textcolor{red}{false}:} Voulez-vous mettre en premier plan 
	la fenêtre du jeu ?\\\\
	% NoResizable property.
	\textbf{+ \textcolor{red}{bool} NoResizable = \textcolor{red}{false}:} Voulez-vous permettre le
	redimentionnement de la fenêtre du jeu ?\\\\
	% OSName property.
	\textbf{+ \textcolor{red}{int} OSName = \textcolor{blue}{0}:} Contient les différentes catégories de 
	systèmes d'exploitation dont la résolution de l'écran a été prise en charge. Cette option ne s'active 
	que lorsque vous permettez le redimentionnement de la fenêtre du jeu. Les valeurs possibles sont:
	\begin{itemize}
		\item [-> \textbf{\textcolor{gray}{SettingsFx.DeviceName.DESKTOP} ou \textcolor{blue}{0}}:] Systèmes 
		d'exploitation de bureau.
		\item [-> \textbf{\textcolor{gray}{SettingsFx.DeviceName.IPHONE\_RESOLUTION} ou \textcolor{blue}
		{1}}:] Systèmes d'exploitation iphone.
		\item [-> \textbf{\textcolor{gray}{SettingsFx.DeviceName.IPAD\_RESOLUTION} ou \textcolor{blue}{2}}:] 
		Systèmes d'exploitation ipad.
		\item [-> \textbf{\textcolor{gray}{SettingsFx.DeviceName.ANDROID\_RESOLUTION} ou \textcolor{blue}
		{3}}:] Systèmes d'exploitation android.
		\item [-> \textbf{\textcolor{gray}{SettingsFx.DeviceName.CUSTOM\_RESOLUTION} ou \textcolor{blue}
		{4}}:] Souhaitez-vous définir une résolution \\personnalisée ?\\
	\end{itemize}
	% WindowSize property.
	\newpage \textbf{+ \textcolor{darkgreen}{Vector2} \hypertarget{winsize}{WindowSize} =
	\textcolor{darkgreen}{Vector3} (\textcolor{blue}{1024}, \textcolor{blue}{600}):} Quelle est la nouvelle
	résolution de la fenêtre ? Cette option ne s'active que lorsque vous souhaitez donner une résolution
	personnalisée. Notez que cette \\propriété peut se transformer en une liste de résolution de la catégory 
	du système d'exploitation choisie. Dans ce cas, vous aurez un champ de type entier.\\\\
	% Maximize property.
	\textbf{+ \textcolor{red}{bool} Maximize = \textcolor{red}{false}:} Voulez-vous maximiser la taille de 
	la fenêtre du jeu ? Si cette \\fonctionnalité est activée, la fenêtre du jeu sera redimentionnée à la 
	résolution maximale de l'écran. Notez que cette option ne s'active que lorsque vous permettez le
	redimentionnement de la fenêtre du jeu.\\\\
	% FullScreen property.
	\textbf{+ \textcolor{red}{bool} FullScreen = \textcolor{red}{false}:} Contrôle le mode plein écran du 
	jeu.\\\\
	% MinSize property.
	\textbf{+ \textcolor{darkgreen}{Vector2} MinSize = \textcolor{darkgreen}{Vector2} (\textcolor{blue}
	{1024}, \textcolor{blue}{600}):} Quelle est la résolution minimale de la fenêtre du jeu ? Si ne serais
	que l'une des valeurs de ce vecteur est négative ou nulle, on considera que la valeur minimale de cette 
	dernière est nulle. Ne donnez pas une résolution suppérieur à celle de la fenêtre du jeu.\\\\
	% MaxSize property.
	\textbf{+ \textcolor{darkgreen}{Vector2} MaxSize = \textcolor{darkgreen}{Vector2} (\textcolor{blue}
	{0}, \textcolor{blue}{0}):} Quelle est la résolution maximale de la fenêtre du jeu ? Si ne serais que 
	l'une des valeurs de ce vecteur est inférieur à la résolution actuelle de la fenêtre du jeu, on 
	considera que la taille maximale est égale à celle actuelle au sein de la fenêtre du jeu. Ne donnez pas 
	une résolution suppérieur à ce que peut supporté l'écran de votre machine.\\\\
	% WindowPosition property.
	\textbf{+ \textcolor{darkgreen}{Vector2} WindowPosition = \textcolor{darkgreen}{Vector2}
	(\textcolor{blue}{-1}, \textcolor{blue}{-1}):} Quelle est la position de la fenêtre du jeu ? Si ne 
	serais que l'une des valeurs de ce vecteur est négative, cette dernière sera automatiquement centrée 
	sur l'écran à la valeur ne respectant pas les conditions d'une résolution à la première exécution du 
	jeu.\\\\
	% TargetControl property.
	\textbf{+ \textcolor{red}{int} TargetControl = \textcolor{blue}{0}:} Contient l'index de la disposition 
	de la manette choisie par le joueur. Prenez l'exemple de \textbf{\textcolor{gray}{Resident Evil}} au 
	niveau de la gestion de la disposition des touches de la manette. Cette propriétée peut vous servir 
	surtout si vous souhaitez implémenter cette fonctionnalitée.\\\\
	% Brightness property.
	\textbf{+ \textcolor{red}{int} Brightness = \textcolor{blue}{13}:} Contrôle la luminosité de l'écran 
	dans le jeu. Ces valeurs sont dans \\l'intervalle de [\textcolor{blue}{0}; \textcolor{blue}{100}].\\\\
	% Contrast property.
	\textbf{+ \textcolor{red}{int} Contrast = \textcolor{blue}{13}:} Contrôle l'effet subjectif d'une 
	apposition quantitative de couleurs. Ces valeurs sont dans l'intervalle de [\textcolor{blue}{0};
	\textcolor{blue}{100}].\\\\
	% Saturation property.
	\textbf{+ \textcolor{red}{int} Saturation = \textcolor{blue}{13}:} Contrôle la pureté de la couleur de 
	l'écran dans le jeu. Ces valeurs sont dans l'intervalle de [\textcolor{blue}{0}; \textcolor{blue}{100}].
	\\\\
	% JoyVibration property.
	\textbf{+ \textcolor{red}{bool} JoyVibration = \textcolor{red}{false}:} Voulez-vous activer la vibration 
	de la manette ?\\\\
	% PixelizeScreen property.
	\textbf{+ \textcolor{red}{bool} PixelizeScreen = \textcolor{red}{false}:} Une fois activée, celle-ci 
	Surveille les changements de taille de \\fenêtre et les poignées pour mettre à l'échelle de l'écran du 
	jeu avec un nombre entier exact, des \\multiples d'une résolution de base en mémoire. Utile pour les jeu 
	de pixélisation 2D.\\
	% KeepScreen property.
	\textbf{+ \textcolor{red}{bool} KeepScreen = \textcolor{red}{true}:} Désirez-vous garder 
	l'écran actif lorsque le jeu est en cours d'exécution.\\\\
	% VSync property.
	\textbf{+ \textcolor{red}{bool} VSync = \textcolor{red}{true}:} Voulez-vous activer la synchrônisation
	verticale au cours de l'exécution du jeu ?\\\\
	% VSyncCompositor property.
	\textbf{+ \textcolor{red}{bool} VSyncCompositor = \textcolor{red}{false}:} Voulez-vous activer la
	synchrônisation verticale par le biais d'un compositeur au cours de l'exécution du jeu ? Dans ce cas,
	le compositeur de fenêtre du système \\d'exploitation sera utilisé lorsque le jeu est uniquement en mode 
	fenêtré. Notez que cette option n'est activée que sur un système d'exploitation de type Windows et est 
	également expérimentale et destinée à atténuer le bégaiement ressenti par certains utilisateurs. 
	Cependant, certains utilisateurs ont subi une réduction de moitié de la fréquence d'images (par exemple, 
	de 60 FPS à 30 FPS) lors de leur utilisation.\\\\
	% GameOptimization property.
	\textbf{+ \textcolor{red}{bool} GameOptimization = \textcolor{red}{true}:} Voulez-vous optimiser 
	l'utilisation du processeur ? A ce niveau, le rafraîchissement de l'écran n'est fait que si cela est
	nécessaire afin de réduire la consommation de la batterie. Utile pour les appareils portables.\\\\
	% Source property.
	\textbf{+ \textcolor{darkgreen}{String} Source = "game\_configs.cfg":} Souhaitez-vous créer un fichier 
	de configuration contenant toutes les configurations globales effectuées sur le jeu ? A ce niveau, le 
	module mettra à jour le fichier externe des configurations du jeu à chaque sauvegarde éffectuée à son 
	égard uniquement si le \\développeur à précisé un fichier (existant ou inexistant). Notez que 
	l'extension de votre fichier doit être le (.cfg).\\\\
	\textbf{+ \textcolor{red}{int} TargetPath = \textcolor{blue}{0}:} Contient les différents chemins que
	prend en charge ce module. Ces chemins \\représentent les endroits possibles où l'on peut déposé le
	fichier externe des configurations du jeu. Les valeurs possibles sont:
	\begin{itemize}
		\item[-> \textbf{\textcolor{gray}{MegaAssets.Path.GAME\_LOCATION} ou \textcolor{blue}{0}}:] Cible le 
		dossier racine du jeu.
		\item[-> \textbf{\textcolor{gray}{MegaAssets.Path.OS\_ROOT} ou \textcolor{blue}{1}}:] Cible le 
		dossier racine du système d'exploitation installé.
		\item[-> \textbf{\textcolor{gray}{MegaAssets.Path.USER\_DATA} ou \textcolor{blue}{2}}:] Cible le 
		dossier racine des données de l'utilisateur.
		\item[-> \textbf{\textcolor{gray}{MegaAssets.Path.USER\_ROOT} ou \textcolor{blue}{3}}:] Cible le 
		dossier racine de l'utilisateur.
		\item[-> \textbf{\textcolor{gray}{MegaAssets.Path.USER\_DESKTOP} ou \textcolor{blue}{4}}:] Cible le 
		bureau du système d'exploitation.
		\item[-> \textbf{\textcolor{gray}{MegaAssets.Path.USER\_PICTURES} ou \textcolor{blue}{5}}:] Cible le 
		dossier \textcolor{gray}{\textit{Images}} du système d'exploitation.
		\item[-> \textbf{\textcolor{gray}{MegaAssets.Path.USER\_MUSIC} ou \textcolor{blue}{6}}:] Cible le 
		dossier \textcolor{gray}{\textit{Musiques}} du système d'exploitation.
		\item[-> \textbf{\textcolor{gray}{MegaAssets.Path.USER\_VIDEOS} ou \textcolor{blue}{7}}:] Cible le 
		dossier \textcolor{gray}{\textit{Vidéos}} du système d'exploitation.
		\item[-> \textbf{\textcolor{gray}{MegaAssets.Path.USER\_DOCUMENTS} ou \textcolor{blue}{8}}:] Cible 
		le dossier \textcolor{gray}{\textit{Documents}} du système \\d'exploitation.
		\item[-> \textbf{\textcolor{gray}{MegaAssets.Path.USER\_DOWNLOADS} ou \textcolor{blue}{9}}:] Cible 
		le dossier \textcolor{gray}{\textit{Téléchargements}} du système \\d'exploitation.\\
	\end{itemize}
	\textbf{\textcolor{red}{int} SecurityLevel = \textcolor{blue}{0}:} Contient les différents niveaux de 
	sécurité possibles que l'on peut utilisé au cours de la création du fichier externe des configurations 
	du jeu. Les valeurs possibles sont:
	\begin{itemize}	
		\item [-> \textbf{\textcolor{gray}{MegaAssets.SecurityLevel.SIMPLE} ou \textcolor{blue}{0}}:] Simple 
		niveau de sécurité.
		\item [-> \textbf{\textcolor{gray}{MegaAssets.SecurityLevel.NORMAL} ou \textcolor{blue}{1}}:] Niveau 
		de sécurité normal.
		\item [-> \textbf{\textcolor{gray}{MegaAssets.SecurityLevel.ADVANCED} ou \textcolor{blue}{2}}:] 
		Niveau de sécurité avancé.\\
	\end{itemize}

	% Settings methods definition.
	\section{Les méthodes disponibles}
	% Int get_game_time () method description.
	\begin{description}
		\item [+ \textcolor{red}{int} \textcolor{blue}{get\_game\_time} ():] Renvoie le nombre de secondes
		déjà écoulés depuis le démarrage du jeu.\\
	\end{description}
	% Int get_scene_time () method description.
	\begin{description}
		\item [+ \textcolor{red}{int} \textcolor{blue}{get\_scene\_time} ():] Renvoie le nombre de secondes
		déjà écoulés depuis le lencement de la scène actuelle du jeu.\\
	\end{description}
	% Vector2 get_resolution () method description.
	\begin{description}
		\item [+ \textcolor{darkgreen}{Vector2} \textcolor{blue}{get\_resolution} ():] Renvoie la résolution
		actuellement sélectionnée au niveau du champ \textit{\hyperlink{winsize}{WindowSize}}.\\
	\end{description}
	% Void apply_settings () method description.
	\begin{description}
		\item [+ \textcolor{red}{void} \textcolor{blue}{apply\_settings} (delay = 0.0):] Applique les 
		configurations éffectuées au niveau du module à l'ensemble des éléments concernés par les
		configurations du jeu.
		\begin{itemize}
			\item [>> \textbf{\textcolor{red}{float} delay}:] Quel est le temps mort avant la l'application
			des configurations ?\\
		\end{itemize}
	\end{description}
	% Void load_configs_data () method description.
	\begin{description}
		\item [+ \textcolor{red}{void} \textcolor{blue}{load\_configs\_data} (delay = 0.0):] Charge à partir 
		du fichier externe de configurations, les données des configurations du jeu. Notez qu'il ne se 
		passera rien si le fichier est corromput.
		\begin{itemize}
			\item [>> \textbf{\textcolor{red}{float} delay}:] Quel est le temps mort avant le chargement des
			des données de configurations ?\\
		\end{itemize}
	\end{description}
	% Void save_game_configs_file () method description.
	\begin{description}
		\item [+ \textcolor{red}{void} \textcolor{blue}{save\_game\_configs\_file} (delay = 0.0):] Cré ou 
		mets à jour le fichier externe des \\configurations du jeu.
		\begin{itemize}
			\item [>> \textbf{\textcolor{red}{float} delay}:] Quel est le temps mort avant la mise à jour du
			fichier de configurations ?\\
		\end{itemize}
	\end{description}

	% Settings signals definition.
	\section{Les événements disponibles}
	% before_save_configs event description.
	\begin{description}
		\item [+ \textcolor{blue}{before\_save\_configs} ():] Signal déclenché avant la mise à jour du 
		fichier externe des \\configurations du jeu.\\
	\end{description}
	% after_save_configs event description.
	\begin{description}
		\item [+ \textcolor{blue}{after\_save\_configs} ():] Signal déclenché après la mise à jour du 
		fichier externe des configurations du jeu.\\
	\end{description}
	% scene_time_changed event description.
	\newpage \begin{description}
		\item [+ \textcolor{blue}{scene\_time\_changed} (time):] Signal déclenché à chaque fois le temps 
		écoulé depuis le \\lencement de la scène du jeu évolue.
		\begin{itemize}
			\item [>> \textbf{\textcolor{red}{int} time}:] Contient le temps actuellement écoulé en seconde.
			\\
		\end{itemize}
	\end{description}
	% file_corrupted () signal description.
	\begin{description}
		\item [+ \textcolor{blue}{file\_corrupted} ():] Signal déclenché lorsque le fichier de
		configurations est corromput de \\l'extérieur.\\
	\end{description}
	% file_saving event description.
	\begin{description}
		\item [+ \textcolor{blue}{file\_saving} (progress):] Signal déclenché pendant que le fichier de
		configurations est en cours de sauvegarde.
		\begin{itemize}
			\item [>> \textbf{\textcolor{red}{int} progress}:] Contient la progression actuelle de la
			sauvegarde.\\
		\end{itemize}
	\end{description}
	% file_loading event description.
	\begin{description}
		\item [+ \textcolor{blue}{file\_loading} (progress):] Signal déclenché pendant que le fichier de
		configurations est en cours de chargement.
		\begin{itemize}
			\item [>> \textbf{\textcolor{red}{int} progress}:] Contient la progression actuelle du
			chargement.\\
		\end{itemize}
	\end{description}
\end{document}