% Document type and package imports.
\documentclass[a4paper, 11pt]{article}
\usepackage[utf8]{inputenc}
\usepackage[T1]{fontenc}
\usepackage[french]{babel}
\usepackage{charter}
\usepackage[top = 2cm, bottom = 2cm, left = 1cm, right = 1cm]{geometry}
\usepackage{setspace}
\usepackage{color}
\usepackage{xcolor}
\usepackage{hyperref}
\usepackage{tocloft}

% Preanblue.
\onehalfspacing
\definecolor{gray}{rgb}{0.4, 0.4, 0.4}
\definecolor{silver}{rgb}{0.95, 0.95, 0.95}
\renewcommand{\thesection}{\Roman{section} --}
\definecolor{darkgreen}{HTML}{1E8C15}
\cftsetindents{section}{1em}{2.5em}
\hypersetup {colorlinks=true, linkcolor=blue, urlcolor=blue, pdftitle={Permutator module doc}}

% The start of the article.
\begin{document}
	% Change document background to silver color.
	\pagecolor{silver}
	% Permutator module description.
	\huge{\hspace{14cm}\textit{\textbf{\textcolor{darkgreen}{PermutatorFx}}}}\large{} \tableofcontents 
	\newpage
	% Permutator module definition.
	\section{Définition}
	\textcolor{darkgreen}{\textbf{PermutatorFx}} est un module conçut pour permuter un noeud par un autre. 
	La permutation peut se faire en fonction de certains critères.\\
	\textcolor{red}{\textbf{NB}:} Ce module est compatible à un jeu 2D, 3D et est sauvegardable.

	% Permutator properties definition.
	\section{Les propriétés disponibles}
	% Target property.
	\textbf{+ \textcolor{darkgreen}{NodePath} \hypertarget{target}{Target}:} Contient l'instance d'un noeud 
	de type \href{https://docs.godotengine.org/en/stable/classes/class_spatial.html}
	{\textit{\textcolor{darkgreen}{Spatial}}} ou
	\href{https://docs.godotengine.org/en/stable/classes/class_node2d.html}
	{\textit{\textcolor{darkgreen}{Node2D}}}.\\\\
	% Domain property.
	\textbf{+ \textcolor{darkgreen}{NodePath} Domain:} Contient l'instance d'un noeud de type
	\href{https://docs.godotengine.org/en/stable/classes/class_area.html}
	{\textit{\textcolor{darkgreen}{Area}}} ou
	\href{https://docs.godotengine.org/en/stable/classes/class_area2d.html}
	{\textit{\textcolor{darkgreen}{Area2D}}}. Notez que si la cible est un objet appartenent à un plan,
	alors vous devriez choisir un domaine du plan. IDEM pour l'objet dans l'espace.\\\\
	% PartialDestruction property.
	\textbf{+ \textcolor{red}{bool} PartialDestruction = \textcolor{red}{false}:} Doit-on détruire la cible 
	par la méthode \textcolor{blue}{\textit{queue\_free ()}} ou par la propriété \textcolor{gray}
	{\textit{visible}}.\\\\
	% Solidity property.
	\textbf{+ \textcolor{red}{float} Solidity = \textcolor{blue}{0.0}:} Contient la solidité à appliquée à
	la cible avant sa permutation. La valeur de cette propriété est dans l'intervalle [\textcolor{blue}
	{0.0}, \textcolor{blue}{100.0}]. C'est aussi son degré de résistance.\\\\
	% Subtitutes property.
	\textbf{+ \textcolor{darkgreen}{Array} Subtitutes:} Tableau de dictionnaires ou de chaînes contenant 
	toutes les différentes \\configurations sur chaque descendant prise en charge par le développeur. Cette 
	propriété renferme tous les objets qui devront remplacer la cible de ce module. Les dictionnaires issus 
	de ce tableau supportent les clés suivantes:
	\begin{itemize}
		\item[>> \textbf{\textcolor{darkgreen}{String} path}:] Quel est le chemin pointant vers le 
		remplaçant de la dite cible ? L'utilisation de cette clé est obligatoire.\\
		\item[>> \textbf{\textcolor{darkgreen}{Vector3 | Vector2 | NodePath} position}:] A quelle position 
		sera importer le remplaçant de la cible du module ? Si vous donnez le chemin d'un noeud de la scène, 
		veillez à ce que l'objet soit (dans le plan si la cible donnée dans le champ 
		\textit{\hyperlink{target}{Target}} est un noeud de type
		\href{https://docs.godotengine.org/en/stable/classes/class_node2d.html}
		{\textit{\textcolor{darkgreen}{Node2D}}}, dans l'espace \\sinon). 
		Notez que la non utilisation de cette clé provoquera une importation qui ciblera la position 
		qu'avait son pré-décésseur avant sa destruction.\\
		\item[>> \textbf{\textcolor{red}{bool} rotated = \textcolor{red}{true}}:] Doit-on changer la 
		rotation du remplaçant à celle de son pré-décésseur ?\\
		\item[>> \textbf{\textcolor{red}{bool} background = \textcolor{red}{true}}:] Contrôle le moyen 
		utilisé pour charger un objet. A \textit{\textcolor{gray}{true}}, le chargement de l'objet 
		s'éffectue en arrière plan sans bloqué le jeu.\\
		\item[>> \textbf{\textcolor{red}{bool} import = \textcolor{red}{true}}:] Doit-on importer 
		automatiquement le remplaçant ? Cette propriété n'est \\solicité qu'après le chargement des données
		du gestionnaire des données du jeu.\\
		\item[>> \textbf{\textcolor{red}{bool} physic = \textcolor{red}{false}}:] Doit-on projeté le 
		remplaçant physiquement en fonction de puissance et de la direction de frappe ?\\
	\end{itemize}
	% Triggers property.
	\textbf{+ \textcolor{darkgreen}{Array} Triggers:} Tableau de dictionnaires contenant toutes les 
	différentes configurations sur chaque objet prise en charge par le développeur. Cette propriété renferme 
	tous les objets qui devront percuter la cible de ce module. Les dictionnaires issus de ce tableau 
	supportent les clés suivantes:\\
	\begin{itemize}
		\item[>> \textbf{\textcolor{darkgreen}{String} id}:] Quel est l'identifiant du noeud à prendre en 
		charge ? L'utilisation de cette clé est \\obligatoire.\\
		\item[>> \textbf{\textcolor{red}{int} search = \textcolor{blue}{3}}:] Quel moyen utilisé pour 
		chercher le noeud à prendre en charge ? Notez que \\l'identifiant donné est pisté à par un programme 
		de recherche. Les valeurs possibles sont:
		\begin{itemize}
			\item [-> \textbf{\textcolor{gray}{MegaAssets.NodeProperty.NAME} ou \textcolor{blue}{0}}:] 
			Trouve un noeud en utilisant son nom.
			\item [-> \textbf{\textcolor{gray}{MegaAssets.NodeProperty.GROUP} ou \textcolor{blue}{1}}:] 
			Trouve un noeud en utilisant le nom de son groupe.
			\item [-> \textbf{\textcolor{gray}{MegaAssets.NodeProerty.TYPE} ou \textcolor{blue}{2}}:] Trouve 
			un noeud en utilisant le nom de sa classe.
			\item [-> \textbf{\textcolor{gray}{MegaAssets.NodeProerty.ANY} ou \textcolor{blue}{3}}:] Trouve 
			un noeud en utilisant l'un des trois moyens cités plus haut.\\
		\end{itemize}
		\item[>> \textbf{\textcolor{red}{bool} ignored = \textcolor{red}{false}}:] Doit-on ignoré
		l'identifiant précisé ? Dans ce cas, l'objet portant l'identifiant spécifié sera ignoré à sa 
		détection.\\
		\item[>> \textbf{\textcolor{red}{float} power = \textcolor{blue}{0.0}}:] Contient la puissance de 
		frappe de l'objet qui percutera la cible qu'écoute le module. Notez que plus la force de percution
		est forte plus vite la cible sera remplacée par ces descendants.
	\end{itemize}

	% Permutator methods definition.
	\section{Les méthodes disponibles}
	% Void is_permute () method description.
	\begin{description}
		\item [+ \textcolor{red}{void} \textcolor{blue}{is\_permute} ():] La cible actuelle a t-elle été 
		déjà permutée ?
	\end{description}

	% Permutator signals definition.
	\section{Les événements disponibles}
	% before_permute event description.
	\begin{description}
		\item [+ \textcolor{blue}{before\_permute} (node):] Signal déclenché avant la permutation d'un objet
		par un autre.
		\begin{itemize}
			\item [>> \textbf{\textcolor{darkgreen}{Node} node}:] Contient le noeud où ce signal a été émit.
			\\
		\end{itemize}
	\end{description}
	% after_permute event description.
	\begin{description}
		\item [+ \textcolor{blue}{after\_permute} (node):] Signal déclenché après la permutation d'un objet
		par un autre.
		\begin{itemize}
			\item [>> \textbf{\textcolor{darkgreen}{Node} node}:] Contient le noeud où ce signal a été émit.
			\\
		\end{itemize}
	\end{description}
	% collision event description.
	\begin{description}
		\item [+ \textcolor{blue}{collision} (data):] Signal déclenché lorsqu'un objet prise en charge par 
		le module percute sa cible. Notez que cela ne s'applique qu'aux objets non ignorés par le module. 
		Cet événement renvoie un dictionaire contenant les clés suivantes:
		\begin{itemize}
			\item [>> \textbf{\textcolor{darkgreen}{Node} node}:] Contient le noeud où ce signal a été émit.
			\item [>> \textbf{\textcolor{darkgreen}{Variant} object}:] Contient la référence de l'objet qui 
			a été détecté. L'objet renvoyé sera l'instance d'un noeud de type
			\href{https://docs.godotengine.org/en/stable/classes/class_spatial.html}
			{\textit{\textcolor{darkgreen}{Spatial}}} ou
			\href{https://docs.godotengine.org/en/stable/classes/class_node2d.html}
			{\textit{\textcolor{darkgreen}{Node2D}}}.
		\end{itemize}
	\end{description}
\end{document}