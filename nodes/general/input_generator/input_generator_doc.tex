% Document type and package imports.
\documentclass[a4paper, 11pt]{article}
\usepackage[utf8]{inputenc}
\usepackage[T1]{fontenc}
\usepackage[french]{babel}
\usepackage{charter}
\usepackage[top = 2cm, bottom = 2cm, left = 1cm, right = 1cm]{geometry}
\usepackage{setspace}
\usepackage{color}
\usepackage{xcolor}
\usepackage{hyperref}
\usepackage{tocloft}

% Preanblue.
\onehalfspacing
\definecolor{gray}{rgb}{0.4, 0.4, 0.4}
\definecolor{silver}{rgb}{0.95, 0.95, 0.95}
\renewcommand{\thesection}{\Roman{section} --}
\definecolor{darkgreen}{HTML}{1E8C15}
\cftsetindents{section}{1em}{2.5em}
\hypersetup {colorlinks=true, linkcolor=blue, urlcolor=blue, pdftitle={InputGenerator module doc}}

% The start of the article.
\begin{document}
	% Change document background to silver color.
	\pagecolor{silver}
	% InputGenerator module description.
	\huge{\hspace{12.5cm}\textit{\textbf{\textcolor{darkgreen}{InputGeneratorFx}}}}\large{} \tableofcontents 
	\newpage
	% InputGenerator module definition.
	\section{Définition}
	\textcolor{darkgreen}{\textbf{InputGeneratorFx}} est un module conçut pour les lectures de commandes 
	programmées. Son \\importance prend vie lorsqu'on souhaite établir une certaine interaction entre le 
	joueur et un \textcolor{gray}{\textit{IA}}. Ces interactions seront souvent matérialisées par des 
	animations entre le joueur et le système du jeu. Ce module génère de façon aléatoire une touche en 
	fonction du type de contrôleur détecté dont l'utilisateur devra appuyer avant un délai donné. Lorsque la 
	touche générée est appuyée, une action se déclenche caractérisant la réussite. Dans le cas contraire, 
	une autre action se déclenche caractérisant l'échec. Ainsi, l'utilisateur peut décidé de ce qui se 
	passera en cas de réussite et d'échec. Notez que se module prend également en charge les appuies répétés 
	sur une même touche (\textcolor{gray}{\textit{Combo}}).\\
	\textcolor{red}{\textbf{NB}:} Ce module est compatible à un jeu 2D, 3D et n'est pas sauvegardable. Avant 
	de continuer, le module ne supporte pas encore les écrans tactiles.

	% InputGenerator properties definition.
	\section{Les propriétés disponibles}
	% PlayerIndex property.
	\textbf{+ \textcolor{red}{int} PlayerIndex = \textcolor{blue}{0}:} Quel joueur voulez-vous écouter ? 
	Notez que cette option n'est destiné qu'aux manettes.\\\\
	% Inputs property.
	\textbf{+ \textcolor{darkgreen}{Array} Inputs:} Tableau de dictionnaires contenant toutes les 
	différentes configurations sur chaque touche prise en charge par le développeur. Les dictionnaires issus 
	de ce tableau supportent les clés suivantes:
	\begin{itemize}
		\item[>> \textbf{\textcolor{red}{int} controller = \textcolor{blue}{0}}:] Quel contrôleur voulez-
		vous écouter ? Les valeurs possibles sont:
		\begin{itemize}
			\item[-> \textbf{\textcolor{gray}{InputGeneratorFx.KEYBOARD} ou \textcolor{blue}{0}}:] Ecoute le 
			clavier de l'ordinateur.
			\item[-> \textbf{\textcolor{gray}{InputGeneratorFx.MOUSE} ou \textcolor{blue}{1}}:] Ecoute la 
			souris de l'ordinateur.
			\item[-> \textbf{\textcolor{gray}{InputGeneratorFx.JOYSTICK} ou \textcolor{blue}{2}}:] Ecoute la 
			manette connectée à l'ordinateur.
			\item[-> \textbf{\textcolor{gray}{InputGeneratorFx.INPUTMAP} ou \textcolor{blue}{3}}:] Ecoute la 
			carte des entrées définit au sein de Godot.\\
		\end{itemize}
		\item[>> \textbf{\textcolor{darkgreen}{String} key}:] Quel est le nom de la touche du contrôleur à
		mettre en écoute ? A ce niveau, il y a certaines restrictions que vous devez respecter. Si le 
		contrôleur est le clavier, vous êtes plié de mettre le nom correspondant à la touche recherchée tout 
		en respectant ce que renvoie \textcolor{gray}{\\Godot} lorsqu'on appuie sur une telle touche. Si le 
		contrôleur est la souris, en cas de bouton, il doit suivre la nomenclature \textcolor{gray}
		{\textbf{Mouse+IndexDeLaToucheVoulue}}. \\Exemple: \textcolor{gray}{\textit{Mouse2, Mouse1, 
		...MouseN}}. \\En cas d'axe, le ciblage des défilements doit se faire avec les mots clés suivants: 
		\textcolor{gray}{\textit{-ScrollX, +ScrollX, -ScrollY, +ScrollY}} et celui des mouvements avec 
		\textcolor{gray}{-MouseX, +MouseX, -MouseY, +MouseY}. Si le contrôleur est la manette, en cas de 
		bouton, il doit suivre la nomenclature \textcolor{gray}{\textbf{\\Joy+IndexDeLaToucheVoulue}}.\\
		Exemple: \textcolor{gray}{\textit{Joy0, Joy1, ...JoyN}}. \\En cas d'axe, suivre la nomenclature 
		\textcolor{gray}{\textbf{Signe+Axis+IndexDeLaxeVoulu}}. \\Exemple: \textcolor{gray}{\textit{-Axis2, 
		+Axis0, ...-/+AxisN}}. \\Si vous écoutez la carte d'une ou de plusieurs entrée(s) prédéfinit dans 
		les configurations de votre projet, vous devez renseigner l'identificateur de la carte à ciblée.\\
		\item[>> \textbf{\textcolor{red}{bool} translate = \textcolor{red}{false}}:] Voulez-vous traduire la 
		touche actuelle en terme plus claire pour les \\futures utilisateurs de votre application. 
		L'activation de cette option essaye de faire une traduction objective de la touche détectée tout en 
		s'adaptant au contrôleur détecté. Notez que cela se fera chaque fois qu'une touche sera générée et 
		cette option n'est pas activée lorsque le contrôleur \\utilise la carte des entrées définit au sein 
		de Godot.\\
	\end{itemize}
	% Challenges property.
	\textbf{+ \textcolor{darkgreen}{Array} Challenges:} Tableau de dictionnaires contenant toutes les 
	différentes configurations sur chaque défi prise en charge par le développeur. Les dictionnaires issus 
	de ce tableau supportent les clés suivantes:
	\begin{itemize}
		\item[>> \textbf{\textcolor{red}{float} delay = \textcolor{blue}{0.0}}:] Quel est le temps mort 
		avant l'exécution du défis en question ? Cette option ne s'active que lorsque sa valeur est 
		supérieur à \textcolor{blue}{\textbf{0.0}}.\\
		\item[>> \textbf{\textcolor{red}{float} time = \textcolor{blue}{0.0}}:] Combien de temps, le défis 
		durera t-il ? Cette option ne s'active que lorsque sa valeur est supérieur à \textcolor{blue}
		{\textbf{0.0}}. Dans le cas contraire, le défi attend sa validation.\\
		\item[>> \textbf{\textcolor{red}{int} repeat = \textcolor{blue}{1}}:] Combien de fois, le défis va 
		t-il se répété ? Cette option ne s'active que lorsque sa valeur est supérieur à \textcolor{blue}
		{\textbf{0}}. Dans le cas contraire, le défi n'est pas exécuté.\\
		\item[>> \textbf{\textcolor{red}{int} slow = \textcolor{blue}{0}}:] Contrôle le comportement des 
		éffets de ralentissements dans un défi. Les valeurs possibles sont:
		\begin{itemize}
			\item[-> \textbf{\textcolor{gray}{InputGeneratorFx.SlowMotion.NONE} ou \textcolor{blue}{0}}:] 
			Aucun effet de ralentissement ne se déclenchera.
			\item[-> \textbf{\textcolor{gray}{InputGeneratorFx.SlowMotion.PIG\_PONG} ou \textcolor{blue}
			{1}}:] L'effet de ralentissement se déclenchera de manière alterné.
			\item[-> \textbf{\textcolor{gray}{InputGeneratorFx.SlowMotion.RANDOM} ou \textcolor{blue}{2}}:] 
			L'effet de ralentissement se déclenchera de façon aléatoire.
			\item[-> \textbf{\textcolor{gray}{InputGeneratorFx.SlowMotion.ALL} ou \textcolor{blue}{3}}:] 
			L'effet de ralentissement se déclenchera lorsque le défi sera lancer.\\
		\end{itemize}
		\item[>> \textbf{\textcolor{red}{float} | \textcolor{darkgreen}{Vector2} rate = \textcolor{blue}
		{0.3}}:] Quel sera le taux ou le degré de l'éffet de ralentissement à son \\déclenchement ? Si un 
		réel est utilisé, l'éffet se déclenchera avec le taux donné par le \\développeur. Si un vecteur deux 
		dimention est utilisé, l'éffet se déclenchera avec un taux \\appartenant à l'intervalle précisé soit 
		[\textcolor{blue}{0.0}; \textcolor{blue}{1.0}]. Dans le cas contraire, aucun taux ne sera
		\\appliqué.\\
		\item[>> \textbf{\textcolor{red}{int} count = \textcolor{blue}{1}}:] Combien de fois la touche sera 
		détectée avant le déclenchement de(s) action(s) prévu à son égard ?\\
		\item[>> \textbf{\textcolor{red}{int} step = \textcolor{blue}{1}}:] Quel est le pas à ajouté à 
		chaque fois que la touche en question est détectée ? Cette option ne s'active que lorsque sa valeur 
		et celle de \textit{\textcolor{gray}{count}} sont supérieur à 0.\\
		\item[>> \textbf{\textcolor{red}{bool} resetlevel = \textcolor{red}{true}}:] Voulez-vous 
		rénitialiser le niveau d'appuie de la touche à chaque fois qu'il serait égale à sa valeur maximale ?
		En d'autres termes, doit-on remettre à zéro le niveau d'appuie de la touche lorsqu'elle déclenche
		l'action qui lui est due ?\\
		\item[>> \textbf{\textcolor{red}{int} decrease = \textcolor{blue}{0}}:] Quel est le pas à enlevé 
		lorsque la touche détectée est relâchée ? Cette option permet de mettre en place un système de combo 
		contre système. Elle devient utile lorsque l'on souhaite que le joueur appuie sur une certaine 
		touche donnée de façon répété pour accomplir une action. Cette fonctionnalité fera office de 
		décrémenteur pour ramener les éfforts de l'utilisateur à zéro, si ce dernier décide d'abandonner. 
		Cette option ne s'active que lorsque sa valeur et celle de la clé \textcolor{gray}
		{\textit{frequence}} sont supérieur à zéro.\\
		\item[>> \textbf{\textcolor{red}{float} frequence = \textcolor{blue}{0.03}}:] Quel est le temps mort 
		avant chaque décrémentation. Cette option ne s'active que lorsque sa valeur et celle de
		\textcolor{gray}{\textit{decrease}} sont supérieur à zéro.\\
		\item[>> \textbf{\textcolor{darkgreen}{Array | Dictionary} success}]: Que se passera t-il lorsque le 
		défi imposé a été réussit par son joueur ? L'utilisation de cette clé est déjà décrite au niveau des 
		bases du framework. Précisement le sujet portant sur l'utilisation de la propriété
		\textit{\textcolor{gray}{EventsBindings}} (la section des actions d'un événement).\\
		\item[>> \textbf{\textcolor{darkgreen}{Array | Dictionary} failed}]: Que se passera t-il lorsque le 
		défi imposé a été échoué ? Notez que cette dernière possède les mêmes propriétés que la clé  
		\textit{\textcolor{gray}{success}} et s'utilise de la même manière que lui.\\
	\end{itemize}
	\textcolor{red}{\textbf{NB}:} Les répétitions au niveau des touches ne sont pas tolérées.

	% InputGenerator methods definition.
	\section{Les méthodes disponibles}
	% Void start_challenges () method description.
	\begin{description}
		\item [+ \textcolor{red}{void} \textcolor{blue}{start\_challenges} (delay = 0.0):] Démarre 
		l'exécution des défi prises en charge par le \\module.
		\begin{itemize}
			\item [>> \textbf{\textcolor{red}{float} delay}:] Quel est le temps mort avant le démarrage des
			défi ?\\
		\end{itemize}
	\end{description}
	% Void stop_challenges () method description.
	\begin{description}
		\item [+ \textcolor{red}{void} \textcolor{blue}{stop\_challenges} (delay = 0.0):] Annule l'exécution 
		des défi prises en charge par le \\module. Cette fonction n'agit que lorsque les défi imposés sont 
		déjà en cours d'exécution.
		\begin{itemize}
			\item [>> \textbf{\textcolor{red}{float} delay}:] Quel est le temps mort avant l'annulation des
			défi ?\\
		\end{itemize}
	\end{description}
	% Void pause_challenges () method description.
	\begin{description}
		\item [+ \textcolor{red}{void} \textcolor{blue}{pause\_challenges} (delay = 0.0):] Suspend 
		l'exécution des défi prises en charge par le \\module. Cette fonction n'agit que lorsque les défi 
		imposés sont déjà en cours d'exécution.
		\begin{itemize}
			\item [>> \textbf{\textcolor{red}{float} delay}:] Quel est le temps mort avant la suspension des
			défi ?\\
		\end{itemize}
	\end{description}
	% Bool is_combo () method description.
	\begin{description}
		\item [+ \textcolor{red}{bool} \textcolor{blue}{is\_combo} ():] Détermine si la touche actuellement 
		détectée est en traint d'être appuyée de façon répétée ou pas.\\
	\end{description}
	% Int get_detected_key () method description.
	\begin{description}
		\item [+ \textcolor{red}{int} \textcolor{blue}{get\_detected\_key} ():] Renvoie l'index de position
		de la touche qui a été détectée suite à un appuie ou un relâchement.\\
	\end{description}
	% String get_detected_controller () method description.
	\begin{description}
		\item [+ \textcolor{darkgreen}{String} \textcolor{blue}{get\_detected\_controller} ():] Renvoie le 
		nom du contrôleur actuellement détecté. La \\valeur \textcolor{gray}{Unknown Controller} est
		renvoyée lorsqu'un contrôleur n'a pas pu être identifié.\\
	\end{description}
	% Int get_generated_key () method description.
	\begin{description}
		\item [+ \textcolor{red}{int} \textcolor{blue}{get\_generated\_key} ():] Renvoie l'index de position
		de la touche qui a été générée avant \\l'exécution d'un défi.\\
	\end{description}
	% Int get_preview_key () method description.
	\begin{description}
		\item [+ \textcolor{red}{int} \textcolor{blue}{get\_preview\_key} ():] Renvoie l'index de position
		de la touche qui a été utilisée précédement.\\
	\end{description}
	% Bool is_running () method description.
	\begin{description}
		\item [+ \textcolor{red}{bool} \textcolor{blue}{is\_running} ():] Les défi sont-ils en cours 
		d'exécution ?\\
	\end{description}
	% Int get_challenge_index () method description.
	\begin{description}
		\item [+ \textcolor{red}{bool} \textcolor{blue}{get\_challenge\_index} ():] Renvoie l'index de 
		position du défi actuellement en cours \\d'exécution.\\
	\end{description}
	% float get_normalized_value () method description.
	\begin{description}
		\item [+ \textcolor{red}{float} \textcolor{blue}{get\_normalized\_value} ():] Renvoie une valeur
		d'axe comprise entre \textcolor{blue}{0.0} et \textcolor{blue}{1.0} de la touche actuellement 
		détectée.\\
	\end{description}
	% Int get_keycode () method description.
	\begin{description}
		\item [+ \textcolor{red}{int} \textcolor{blue}{get\_keycode} ():] Renvoie le code de la touche 
		actuellement détectée.\\
	\end{description}
	% Int get_key_level () method description.
	\begin{description}
		\item [+ \textcolor{red}{int} \textcolor{blue}{get\_key\_level} ():] Renvoie la valeur du niveau 
		d'appuie de la touche actuellement détectée.\\
	\end{description}
	% Int get_key_pourcent () method description.
	\begin{description}
		\item [+ \textcolor{red}{int} \textcolor{blue}{get\_key\_pourcent} ():] Renvoie en pourcentage la 
		valeur du niveau d'appuie de la touche \\actuellement détectée.\\
	\end{description}
	% Bool is_a_button () method description.
	\begin{description}
		\item [+ \textcolor{red}{bool} \textcolor{blue}{is\_a\_button} ():] Détermine si la touche 
		actuellement détectée est un bouton ou pas.\\
	\end{description}
	% Bool is_an_axis () method description.
	\begin{description}
		\item [+ \textcolor{red}{bool} \textcolor{blue}{is\_an\_axis} ():] Détermine si la touche 
		actuellement détectée est un bouton d'axe ou pas.\\
	\end{description}
	% Int get_current_repeat () method description.
	\begin{description}
		\item [+ \textcolor{red}{int} \textcolor{blue}{get\_current\_repeat} ():] Renvoie le nombre actuel 
		de fois que le défi actuel à été exécuté.
	\end{description}

	% InputGenerator signals definition.
	\section{Les événements disponibles}
	% challenge_started event description.
	\begin{description}
		\item [+ \textcolor{blue}{challenge\_started} (data):] Signal déclenché juste au démarrage d'un 
		défi. Cet événement \\renvoie un dictionaire contenant les clés suivantes:
		\begin{itemize}
			\item [>> \textbf{\textcolor{darkgreen}{Node} node}:] Contient le noeud où cet signal a été 
			émit.
			\item [>> \textbf{\textcolor{red}{int} count}:] Contient le nombre de fois que le défi en 
			question a été répété.
			\item [>> \textbf{\textcolor{red}{int} challenge}:] Contient l'index de position du défi 
			actuellement en cours d'exécution.\\
		\end{itemize}
	\end{description}
	% challenge_finished event description.
	\newpage \begin{description}
		\item [+ \textcolor{blue}{challenge\_finished} (data):] Signal déclenché à la fin d'un défi. Cet 
		événement renvoie un \\dictionaire contenant les clés suivantes:
		\begin{itemize}
			\item [>> \textbf{\textcolor{darkgreen}{Node} node}:] Contient le noeud où cet signal a été 
			émit.
			\item [>> \textbf{\textcolor{red}{int} count}:] Contient le nombre de fois que le défi en 
			question a été répété.
			\item [>> \textbf{\textcolor{red}{int} challenge}:] Contient l'index de position du défi 
			actuellement en cours d'exécution.\\
		\end{itemize}
	\end{description}
	% challenge_changed event description.
	\begin{description}
		\item [+ \textcolor{blue}{challenge\_changed} (data):] Signal déclenché lorsque le défi actuel a 
		changé. Cet événement renvoie un dictionaire contenant les clés suivantes:
		\begin{itemize}
			\item [>> \textbf{\textcolor{darkgreen}{Node} node}:] Contient le noeud où cet signal a été 
			émit.
			\item [>> \textbf{\textcolor{red}{int} count}:] Contient le nombre de fois que le défi en 
			question a été répété.
			\item [>> \textbf{\textcolor{red}{int} challenge}:] Contient l'index de position du défi 
			actuellement en cours d'exécution.\\
		\end{itemize}
	\end{description}
	% key_level_changed event description.
	\begin{description}
		\item [+ \textcolor{blue}{key\_level\_changed} (data):] Signal déclenché lorsqu'on change le niveau 
		d'appuie d'une touche. Cet événement renvoie un dictionaire contenant les clés suivantes:
		\begin{itemize}
			\item [>> \textbf{\textcolor{darkgreen}{Node} node}:] Contient le noeud où cet signal a été 
			émit.
			\item [>> \textbf{\textcolor{red}{int} value}:] Contient la nouvelle valeur du niveau d'appuie.
			\item [>> \textbf{\textcolor{red}{int} index}:] Contient l'index de position de la touche qui a 
			été détectée.\\
		\end{itemize}
	\end{description}
	% keydown event description.
	\begin{description}
		\item [+ \textcolor{blue}{keydown} (data):] Signal déclenché lorsqu'on appuie sur une touche. Cet 
		événement renvoie un dictionaire contenant les clés suivantes:
		\begin{itemize}
			\item [>> \textbf{\textcolor{darkgreen}{Node} node}:] Contient le noeud où cet signal a été 
			émit.
			\item [>> \textbf{\textcolor{red}{int} index}:] Contient l'index de position de la touche qui a 
			été appuyée.\\
		\end{itemize}
	\end{description}
	% keyup event description.
	\begin{description}
		\item [+ \textcolor{blue}{keyup} (data):] Signal déclenché lorsqu'on relâche une touche. Cet 
		événement renvoie un \\dictionaire contenant les clés suivantes:
		\begin{itemize}
			\item [>> \textbf{\textcolor{darkgreen}{Node} node}:] Contient le noeud où cet signal a été 
			émit.
			\item [>> \textbf{\textcolor{red}{int} index}:] Contient l'index de position de la touche qui a 
			été relâchée.\\
		\end{itemize}
	\end{description}
	% decreaser_started event description.
	\begin{description}
		\item [+ \textcolor{blue}{decreaser\_started} (data):] Signal déclenché lorsque le décrémenteur 
		automatique se démarre. Cet événement renvoie un dictionaire contenant les clés suivantes:
		\begin{itemize}
			\item [>> \textbf{\textcolor{darkgreen}{Node} node}:] Contient le noeud où cet signal a été 
			émit.
			\item [>> \textbf{\textcolor{red}{int} challenge}:] Contient l'index de position du défi 
			actuellement en cours d'exécution.\\
		\end{itemize}
	\end{description}
	% decreaser_stoped event description.
	\begin{description}
		\item [+ \textcolor{blue}{decreaser\_stoped} (data):] Signal déclenché lorsque le décrémenteur 
		automatique s'arrète. Cet événement renvoie un dictionaire contenant les clés suivantes:
		\begin{itemize}
			\item [>> \textbf{\textcolor{darkgreen}{Node} node}:] Contient le noeud où cet signal a été 
			émit.
			\item [>> \textbf{\textcolor{red}{int} challenge}:] Contient l'index de position du défi 
			actuellement en cours d'exécution.\\
		\end{itemize}
	\end{description}
	% key event description.
	\newpage \begin{description}
		\item [+ \textcolor{blue}{key} (data):] Signal déclenché lorsqu'on appuie ou relâche une touche. Cet 
		événement renvoie un dictionaire contenant les clés suivantes:
		\begin{itemize}
			\item [>> \textbf{\textcolor{darkgreen}{Node} node}:] Contient le noeud où cet signal a été 
			émit.
			\item [>> \textbf{\textcolor{red}{int} index}:] Contient l'index de position de la touche qui a 
			été appuyée ou relâchée.\\
		\end{itemize}
	\end{description}
	% success event description.
	\begin{description}
		\item [+ \textcolor{blue}{success} (data):] Signal déclenché lorsque le joueur valide les conditions 
		exigées par un défi. Cet événement renvoie un dictionaire contenant les clés suivantes:
		\begin{itemize}
			\item [>> \textbf{\textcolor{darkgreen}{Node} node}:] Contient le noeud où cet signal a été 
			émit.
			\item [>> \textbf{\textcolor{red}{int} count}:] Contient le nombre de fois que le défi en 
			question a été répété.
			\item [>> \textbf{\textcolor{red}{int} challenge}:] Contient l'index de position du défi 
			actuellement en cours d'exécution.\\
		\end{itemize}
	\end{description}
	% failed event description.
	\begin{description}
		\item [+ \textcolor{blue}{failed} (data):] Signal déclenché lorsque le joueur n'a pas pu validé les 
		conditions exigées par un défi. Cet événement renvoie un dictionaire contenant les clés suivantes:
		\begin{itemize}
			\item [>> \textbf{\textcolor{darkgreen}{Node} node}:] Contient le noeud où cet signal a été 
			émit.
			\item [>> \textbf{\textcolor{red}{int} count}:] Contient le nombre de fois que le défi en 
			question a été répété.
			\item [>> \textbf{\textcolor{red}{int} challenge}:] Contient l'index de position du défi 
			actuellement en cours d'exécution.\\
		\end{itemize}
	\end{description}
\end{document}