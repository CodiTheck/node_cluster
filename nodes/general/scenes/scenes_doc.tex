% Document type and package imports.
\documentclass[a4paper, 11pt]{article}
\usepackage[utf8]{inputenc}
\usepackage[T1]{fontenc}
\usepackage[french]{babel}
\usepackage{charter}
\usepackage[top = 2cm, bottom = 2cm, left = 1cm, right = 1cm]{geometry}
\usepackage{setspace}
\usepackage{color}
\usepackage{xcolor}
\usepackage{hyperref}
\usepackage{tocloft}

% Preanblue.
\onehalfspacing
\definecolor{gray}{rgb}{0.4, 0.4, 0.4}
\definecolor{silver}{rgb}{0.95, 0.95, 0.95}
\renewcommand{\thesection}{\Roman{section} --}
\definecolor{darkgreen}{HTML}{1E8C15}
\cftsetindents{section}{1em}{2.5em}
\hypersetup {colorlinks=true, linkcolor=blue, urlcolor=blue, pdftitle={Scenes module doc}}

% The start of the article.
\begin{document}
	% Change document background to silver color.
	\pagecolor{silver}
	% Scenes module description.
	\huge{\hspace{15.5cm}\textit{\textbf{\textcolor{darkgreen}{ScenesFx}}}}\large{} \tableofcontents
	\newpage
	% Scenes module definition.
	\section{Définition}
	\textcolor{darkgreen}{\textbf{ScenesFx}} est un module conçut pour la gestion des chargements de scènes 
	dans un jeu. C'est un \\module très puissant qui intègre un système de chargement dynamique de niveau 
	ainsi qu'une \\optimisation au cours des chargements.\\
	\textcolor{red}{\textbf{NB}:} Ce module est de nature indestructible, est compatible à un jeu 2D, 3D et 
	est sauvegardable.

	% Scenes properties definition.
	\section{Les propriétés disponibles}
	% Optimization property.
	\textbf{+ \textcolor{red}{int} Optimization = \textcolor{blue}{0}:} Contient le mode d'optimisation 
	utilisée pour la gestion des chargements automatiques des scènes d'un niveau donné. Les valeurs 
	possibles sont:
	\begin{itemize}
		\item [-> \textbf{\textcolor{gray}{ScenesFx.Optimisation.NONE} ou \textcolor{blue}{0}}:] Aucune 
		optimisation ne sera appliquer.
		\item [-> \textbf{\textcolor{gray}{ScenesFx.Optimisation.VISIBILITY} ou \textcolor{blue}{1}}:] 
		L'optimisation ciblera la visibilité des scènes.
		\item [-> \textbf{\textcolor{gray}{ScenesFx.Optimisation.DESTRUCTION} ou \textcolor{blue}{2}}:] 
		L'optimisation ciblera la destruction des scènes.\\
	\end{itemize}
	\textbf{+ \textcolor{red}{bool} SyncLoading = \textcolor{red}{true}:} Devons-nous attendre qu'une scène 
	en cours de chargement, se charge entièrement avant de lancer le chargement de la prochaîne scène ?\\\\
	% Levels property.
	\textbf{+ \textcolor{darkgreen}{Dictionary} Levels:} Contient tous les niveaux du jeu. Dans ce 
	dictionaire, le développeur doit suivre la nomenclature \textit{\textcolor{gray}{NomDuNiveau:\{\}}}. Les 
	alcolades font référence à un dictionaire représentant toutes les scènes du niveau en question. Dans ce
	dictionaire, on suivera la nomenclature: \textit{\textcolor{gray}{NomDeLaScene:\\Dictionaire | String}}.
	Lorsqu'on précise le nom de la scène à prendre en charge, on a le choix entre deux type de valeurs: Une
	chaîne de caractères contenant le chemin pointant vers la scène en question ou un dictionaire supportant
	les clés suivantes:\\
	\begin{itemize}
		\item[>> \textbf{\textcolor{darkgreen}{String} path}:] Contient le chemin pointant vers la scène à 
		prendre en charge. De préférence, \\l'extension de la scène doit être le (.tscn) ou (.scn).
		L'utilisation de cette clé est obligatoire.\\
		\item[>> \textbf{\textcolor{darkgreen}{PoolStringArray} borders}:] Quelles sont les noms des scènes 
		voisines de la scène qui a été prise en charge ? Cette section est très important car elle qui est 
		solicité dans les chargements \\automatiques des scènes d'un niveau. Notez que toutes les scènes que 
		vous configurez ont \\forcément des voisines. Dans la réalisation d'un jeu de haut niveau, 
		l'optimisation des scènes est une étape incontournable car elle consiste à déterminé les scènes 
		qui se trouvent à la porté du joueur et celles qui sont hors de sa porté. Ainsi nous pouvont décider 
		de masquer ou détruire celles qui sont en dehors du champ de vision du joueur et d'afficher ou 
		charger celles qui se retrouvent à la porté du joueur sans qu'il ne s'en rendre compte. Cette 
		technique évite à la mémoire graphique d'être trop surchargé diminuant ainsi le taux de ramage du 
		jeu. Cette méthode de chargement ressemble à l'\textit{\textcolor{gray}{occlusion culling}}. Si vous 
		voulez mettre en place un tel système, vous devez faire appel à la clé \textit{\textcolor{gray}
		{borders}} contenant les noms des scènes immédiatement voisine de la scène en \\question. En 
		d'autres termes, ces dernières représenteront le champ de vision ou la porté de la scène en 
		question. Ainsi, si vous connaissez la scène où se trouve le joueur alors vous avez la porté de sa 
		vision et par conséquant décidé quelle scène, il faut cachée ou détruire et quelle scène, il faut 
		affichée ou chargée. Cette opération est automatique car géré par le module lui même. Notez que les 
		noms des scènes que vous préciserez dans cette clé doivent avoir été prises en charge dans le
		module.\\
		\item[>> \textbf{\textcolor{red}{int} priority = \textcolor{blue}{0}}:] Quelle est la priorité de
		chargement de la scène ciblée ? Notez que quelque soit la priorité, les scènes statiques sont 
		d'abord chargées avant celles dynamiques.\\
	\end{itemize}
	\textcolor{red}{\textbf{NB}:} Les répétitions des noms des scènes et des niveaux ne sont pas
	tolérées. Avant de poursuivre il y a deux choses que vous devez savoirs: Le module fonctionne grâce à 
	deux éléments de base qui sont la staticité des scènes ainsi que leur dynamicité. Il y a deux types de 
	scènes qui sont les scènes statiques et celles dynamiques. Les scènes statiques sont chargées quelque 
	soit la progression du joueur dans les scènes du niveau et celles dynamiques sont chargées en fonction 
	de la progression du joueur dans les scènes du niveau. La notion de scène statique est apparut à cause 
	de certains constats. Par exemple le skybox d'un niveau est toujours présent quelque soit la position du 
	joueur. Le skybox pourrai donc être mis dans une scène qui lui est propre afin d'être chargé de façon 
	statique pour qu'il soit toujours présent dans le niveau qui le solicite. La même chose pourrai être 
	aussi faite sur le joueur du jeu. Cependant pour permettre à ce qu'une scène soit chargée en mode 
	statique, il faut mettre le caractère \textit{\textcolor{gray}{"\$"}} devant son chemin d'accès au cours 
	de sa configuration. Par défaut toutes les scènes d'un niveau sont chargées. Il est possible de 
	spécifier celles qui seront uniquement chargées dans le niveau en mettant le caractère 
	\textit{\textcolor{gray}{"@"}} sur le chemin d'accès des scènes désignées comme étant celles à chargées 
	par défaut au cours de leur configuration. Il est conseillé de préciser uniquement les premières scènes 
	du niveau pour préserver l'optimisation. Gardez à l'esprit que cette précision n'est solicité que si 
	aucune progression n'a été éffectuée dans les scènes du niveau en question. \textbf{\textcolor{red}{Une 
	scène ne peut pas être à la fois dynamique et statique. Evitez également de faire des scènes trop 
	lourdes à chargées}}.

	% Scenes methods definition.
	\section{Les méthodes disponibles}
	% Void loads () method description.
	\begin{description}
		\item [+ \textcolor{red}{void} \textcolor{blue}{\hypertarget{load}{loads}} (data):] Charge une ou 
		plusieurs scène(s) ou un ou plusieurs niveau(x) du jeu.
		\begin{itemize}
			\item [>> \textbf{\textcolor{darkgreen}{Array} data}:] Tableau de dictionaire contenant toutes 
			les configurations de tou(te)s les scènes ou niveaux à chargé. Ce dictionaire suporte les clés 
			suivantes:
			\begin{itemize}
				\item[>> \textbf{\textcolor{darkgreen}{String} | \textcolor{red}{int} id}:] Contient
				l'identifiant du niveau ou de la scène à chargé(e). Cependant, pour chargé un niveau,
				l'identifiant doit contenir soit le nom du niveau en question, soit son index de position. 
				Pour charger une scène, l'identifiant devra suivre la nomenclature \\suivante: 
				\textit{\textcolor{gray}{NomDuNiveau|Position/NomDeLaScene|Position}}. Si le niveau n'a pas 
				été précisé, celui actif sera pris pour cible pour faire le traitement demandé. 
				L'utilisation de cette clé est obligatoire.\\
				\item[>> \textbf{\textcolor{red}{bool} open = \textcolor{red}{true}}:] Voulez-vous importer 
				automatiquement la scène ou le niveau après \\chargement ?\\
				\item[>> \textbf{\textcolor{red}{bool} borders = \textcolor{red}{true}}:] Voulez-vous 
				charger également les scènes voisines de la scène en question ? Cette clé ne s'utilise que 
				lorsque l'élément à chargé est une scène et non un niveau.\\
				\item[>> \textbf{\textcolor{red}{bool} reversed = \textcolor{red}{false}}:] Voulez-vous 
				renverser l'ordre de chargement des scènes ? Cette clé ne s'utilise que si les scènes 
				voisines de la scène en question sont prises en charge pour le chargement.\\
				\item[>> \textbf{\textcolor{red}{int | float} interval = \textcolor{blue}{0.0}}:] Quel est 
				l'intervalle de temps avant chaque chargement ?\\
				\item[>> \textbf{\textcolor{red}{bool} destroy = \textcolor{red}{false}}:] Souhaitez-vous
				détruire l'ancienne version de la scène ou du niveau en question avant un nouveau chargement 
				? Cette clé peut être utilisée pour éffectuer des \\rechargements de niveau(x) ou de
				scène(s). Si cette option n'est pas activée, toute scène ou tous niveau déjà chargé(e) ne
				peut être rechargé.\\
			\end{itemize}
		\end{itemize}
	\end{description}
	% Void visibilities () method description.
	\begin{description}
		\item [+ \textcolor{red}{void} \textcolor{blue}{visibilities} (data):] Contrôle la visibilité d'un 	
		ou de plusieurs scène(s).
		\begin{itemize}
			\item [>> \textbf{\textcolor{darkgreen}{Array} data}:] Tableau de dictionaire contenant toutes 
			les configurations sur la visibilté de toutes les scènes. Ce dictionaire suporte les clés 
			suivantes:
			\begin{itemize}
				\item[>> \textbf{\textcolor{darkgreen}{String} | \textcolor{red}{int} id}:] Contient
				l'identifiant d'une scène. L'utilisation de cette clé est obligatoire.
				\item[>> \textbf{\textcolor{red}{bool} visible = \textcolor{red}{true}}:] Voulez-vous 
				changer la visibilité de la scène en question ?
				\item[>> \textbf{\textcolor{red}{int | float} interval = \textcolor{blue}{0.0}}:] Quel est 
				l'intervalle de temps avant chaque changement d'état ?\\
			\end{itemize}
		\end{itemize}
	\end{description}
	% Void show_scenes_without () method description.
	\begin{description}
		\item [+ \textcolor{red}{void} \textcolor{blue}{show\_scenes\_without} (id, delay = 0.0):] Rend 
		visible toutes les scènes déjà chargées dans le niveau actif exceptées celles renseignées dans le 
		paramètre \textcolor{gray}{\textit{id}}.
		\begin{itemize}
			\item [>> \textbf{\textcolor{darkgreen}{String | Array | PoolStringArray | PoolIntArray} | 
			\textcolor{darkgreen}{int} id}:] Contient le(s) identifiant(s) de la ou des scène(s) à 
			ciblée(s).
			\item [>> \textbf{\textcolor{red}{float} delay}:] Quel est le temps mort avant les changements
			d'états ?\\
		\end{itemize}
	\end{description}
	% Void hide_scenes_without () method description.
	\begin{description}
		\item [+ \textcolor{red}{void} \textcolor{blue}{hide\_scenes\_without} (id, delay = 0.0):] Rend 
		invisible toutes les scènes déjà chargées dans le niveau actif exceptées celles renseignées dans le 
		paramètre \textcolor{gray}{\textit{id}}.
		\begin{itemize}
			\item [>> \textbf{\textcolor{darkgreen}{String | Array | PoolStringArray | PoolIntArray} | 
			\textcolor{darkgreen}{int} id}:] Contient le(s) identifiant(s) de la ou des scène(s) à 
			ciblée(s).
			\item [>> \textbf{\textcolor{red}{float} delay}:] Quel est le temps mort avant les changements
			d'états ?\\
		\end{itemize}
	\end{description}
	% Array | Node get_enabled_scenes () method description.
	\begin{description}
		\item [+ \textcolor{darkgreen}{Array | Node} \textcolor{blue}{get\_enabled\_scenes} ():] Renvoie les
		références de toutes les scènes visible.\\
	\end{description}
	% Array | Node get_disabled_scenes () method description.
	\begin{description}
		\item [+ \textcolor{darkgreen}{Array | Node} \textcolor{blue}{get\_disabled\_scenes} ():] Renvoie 
		les références de toutes les scènes invisible.\\
	\end{description}
	% Array | Node get_scenes () method description.
	\begin{description}
		\item [+ \textcolor{darkgreen}{Node | Array} \textcolor{blue}{get\_scenes} (id):] Renvoie le(s)
		référence(s) de la ou des scène(s) données à travers leur identifiant.
		\begin{itemize}
			\item [>> \textbf{\textcolor{darkgreen}{String | Array | PoolStringArray | PoolIntArray} |
			\textcolor{red}{int} id}:] Contient le(s) identifiant(s) de la ou des scène(s) à ciblée(s).\\
		\end{itemize}
	\end{description}
	% Array | Node get_loading_scenes () method description.
	\begin{description}
		\item [+ \textcolor{darkgreen}{Array | Node} \textcolor{blue}{get\_loading\_scenes} ():] Renvoie les
		références de toutes les scènes en cours de chargement sur le niveau en question.\\
	\end{description}
	% Array | Node get_loaded_scenes () method description.
	\begin{description}
		\item [+ \textcolor{darkgreen}{Array | Node} \textcolor{blue}{get\_loaded\_scenes} ():] Renvoie les
		références de toutes les scènes déjà chargées sur le niveau en question.\\
	\end{description}
	% Bool is_loading () method description.
	\begin{description}
		\item [+ \textcolor{red}{bool} \textcolor{blue}{is\_loading} (id):] Est-ce que la ou les scène(s) ou 
		le(s) niveau(x) donné(es) sont-ils/elles en cours de chargement ? Utilisez la nomenclature précisée 
		au niveau de la méthode \textit{\hyperlink{load}{loads ()}}.
		\begin{itemize}
			\item [>> \textbf{\textcolor{darkgreen}{String | Array | PoolStringArray | PoolIntArray} |
			\textcolor{red}{int} id}:] Contient le(s) identifiant(s) de la ou des scène(s) ou du/des
			niveau(x) à ciblé(es).\\
		\end{itemize}
	\end{description}
	% Bool is_loaded () method description.
	\begin{description}
		\item [+ \textcolor{red}{bool} \textcolor{blue}{is\_loaded} (id):] Est-ce que la ou les scène(s) ou 
		le(s) niveau(x) donné(es) ont-ils/elles été déjà chargé(es) ?  Utilisez la nomenclature précisée 
		au niveau de la méthode \textit{\hyperlink{load}{loads ()}}.
		\begin{itemize}
			\item [>> \textbf{\textcolor{darkgreen}{String | Array | PoolStringArray | PoolIntArray} |
			\textcolor{red}{int} id}:] Contient le(s) identifiant(s) de la ou des scène(s) ou du/des
			niveau(x) à ciblé(es).\\
		\end{itemize}
	\end{description}
	% Int get_progress () method description.
	\begin{description}
		\item [+ \textcolor{red}{int} \textcolor{blue}{get\_progress} (id):] Renvoie la valeur de 
		progression de la scène ou du niveau donné(e). \\Utilisez la nomenclature précisée au niveau de la 
		méthode \textit{\hyperlink{load}{loads ()}}.
		\begin{itemize}
			\item [>> \textbf{\textcolor{darkgreen}{String} | \textcolor{red}{int} id}:] Contient
			l'identifiant de la scène ou du niveau à ciblé(e).\\
		\end{itemize}
	\end{description}
	% Bool game_progress () method description.
	\begin{description}
		\item [+ \textcolor{red}{bool} \textcolor{blue}{game\_progress} ():] Le joueur a t-il progressé dans
		les scènes du jeu ?\\
	\end{description}
	% Node get_level () method description.
	\begin{description}
		\item [+ \textcolor{darkgreen}{Node} \textcolor{blue}{get\_level} ():] Renvoie la référence du 
		niveau actuellemet chargé.\\
	\end{description}
	% String get_active_scene_name () method description.
	\begin{description}
		\item [+ \textcolor{darkgreen}{String} \textcolor{blue}{get\_active\_scene\_name} (object\_id):] 
		Renvoie le nom de la scène active par le nom de son hôte.
		\begin{itemize}
			\item [>> \textbf{\textcolor{darkgreen}{String} object\_id}:] Contient le nom hôte qui n'est
			d'autre que celui d'un objet ou d'un noeud ou encore d'un joueur par exemple. Notez que le 
			module gère la position de plusieurs objets à la fois. Considérez l'objet comme un point 
			représentant la position de la scène active. Ainsi à partir de la position de l'objet, on peut 
			connaître la scène où il se trouve (scène active) et donc déterminer ces voisines (champ de 
			vision ou porté).\\
		\end{itemize}
	\end{description}
	% String set_active_scene_name () method description.
	\begin{description}
		\item [+ \textcolor{red}{void} \textcolor{blue}{\hypertarget{activescene}{set\_active\_scene\_name}} 
		(object\_id, name, delay = 0.0):] Prévient le module de la nouvelle position de l'hôte dans le 
		niveau en question. Notez que cette méthode joue un rôle très important dans le chargement 
		automatique des scènes d'un niveau ainsi que la progression dans les scènes du jeu.
		\begin{itemize}
			\item [>> \textbf{\textcolor{darkgreen}{String} object\_id}:] Contient le nom hôte qui n'est
			d'autre que celui d'un objet ou d'un noeud ou encore d'un joueur par exemple. Notez que le 
			module gère la position de plusieurs objets à la fois. Considérez l'objet comme un point 
			représentant la position de la scène active. Ainsi à partir de la position de l'objet, on peut 
			connaître la scène où il se trouve (scène active) et donc déterminer ces voisines (champ de 
			vision ou porté).
			\item [>> \textbf{\textcolor{darkgreen}{String} name}:] Contient le nom de la scène où se trouve
			l'hôte.
			\item [>> \textbf{\textcolor{red}{float} delay}:] Quel est le temps mort avant la mise à jour de
			la position de l'hôte ?\\
		\end{itemize}
	\end{description}
	% String | PoolStringArray get_level_names_of () method description.
	\newpage \begin{description}
		\item [+ \textcolor{darkgreen}{String | PoolStringArray} \textcolor{blue}{get\_level\_names\_of} 
		(id):] Renvoie le(s) nom(s) du/des niveau(x) \\contenant la ou les scène(s) donnée(s) grâce à leur
		identifiant.
		\begin{itemize}
			\item [>> \textbf{\textcolor{darkgreen}{String | Array | PoolStringArray | PoolIntArray} |
			\textcolor{red}{int} id}:] Contient le(s) identifiant(s) de la ou des scène(s) à ciblé(es).\\
		\end{itemize}
	\end{description}
	% Void load_last_level () method description.
	\begin{description}
		\item [+ \textcolor{red}{void} \textcolor{blue}{load\_last\_level} (open = true, border = true, 
		reversed = true, delay = 0.0):] Charge le \\niveau récupéré dans le gestionnaire des données du jeu. 
		En d'autres termes, si l'utilisateur \\sauvegarde les données issues de ce module, ce dernier pourra
		récupérer le niveau et la scène ou se trouvait le joueur avant qu'il ne quitte le jeu. Il est 
		absolument obligatoire de mettre à jour le module avec la méthode \textit{\hyperlink{activescene}
		{set\_active\_scene\_name ()}} pour permettre à ce dernier de savoir quel niveau et quelle scène 
		chargée à l'appelle de la méthode \textit{\textcolor{blue}{load\_last\_level ()}}.
		\begin{itemize}
			\item[>> \textbf{\textcolor{red}{bool} open}:] Voulez-vous importer automatiquement la scène ou 
			le niveau après chargement ?
			\item[>> \textbf{\textcolor{red}{bool} borders}:] Voulez-vous charger également les scènes 
			voisines de la scène en question ? Cette clé ne s'utilise que lorsque l'élément à chargé est une 
			scène et non un niveau.
			\item[>> \textbf{\textcolor{red}{bool} reversed}:] Voulez-vous renverser l'ordre de chargement 
			des scènes ? Cette clé ne s'utilise que si les scènes voisines de la scène en question sont 
			prises en charge pour le chargement.
			\item [>> \textbf{\textcolor{red}{float} delay}:] Quel est le temps mort avant le chargement ?\\
		\end{itemize}
	\end{description}
	% Void destroy () method description.
	\begin{description}
		\item [+ \textcolor{red}{void} \textcolor{blue}{destroy} (id = null, delay = 0.0):] Détruit le 
		niveau ou la/les scènes donné(es) grâce à leur identifiant. Par défaut, le niveau actif est pris 
		pour cible.
		\begin{itemize}
			\item [>> \textbf{\textcolor{darkgreen}{String | Array | PoolStringArray | PoolIntArray} | 
			\textcolor{darkgreen}{int} id}:] Contient le(s) identifiant(s) de la ou des scène(s) ou du 
			niveau à ciblé(es).
			\item [>> \textbf{\textcolor{red}{float} delay}:] Quel est le temps mort avant le(s) 
			destruction(s) ?\\
		\end{itemize}
	\end{description}
	% Dictionary get_level_data () method description.
	\begin{description}
		\item [+ \textcolor{darkgreen}{Dictionary} \textcolor{blue}{get\_level\_data} (json = false):] 
		Renvoie toutes les données concernant les niveaux du jeu.
		\begin{itemize}
			\item [>> \textbf{\textcolor{red}{bool} json}:] Voulez-vous renvoyer les données au format json 
			?\\
		\end{itemize}
	\end{description}
	% Bool is_defined () method description.
	\begin{description}
		\item [+ \textcolor{red}{bool} \textcolor{blue}{is\_defined} (id):] Est-ce que la ou les scène(s)  
		donnée(s) sont t-elles présentes dans le niveau actuellement chargé ?
		\begin{itemize}
			\item [>> \textbf{\textcolor{darkgreen}{String | Array | PoolStringArray | PoolIntArray} |
			\textcolor{red}{int} id}:] Contient le(s) identifiant(s) de la ou des scène(s) à ciblée(s).\\
		\end{itemize}
	\end{description}
	% Void open () method description.
	\begin{description}
		\item [+ \textcolor{red}{void} \textcolor{blue}{open} (id):] Ouvre une scène ou un niveau qui a déjà 
		été chargé(e) en mémoire. Utilisez la nomenclature précisée au niveau de la méthode 
		\textit{\hyperlink{load}{loads ()}}.
		\begin{itemize}
			\item [>> \textbf{\textcolor{darkgreen}{String} | \textcolor{red}{int} id}:] Contient
			l'identifiant de la scène ou du niveau à ouvrir.
		\end{itemize}
	\end{description}

	% Scenes signals definition.
	\newpage \section{Les événements disponibles}
	% level_loading event description.
	\begin{description}
		\item [+ \textcolor{blue}{level\_loading} (data):] Signal déclenché lorsqu'un niveau est en cours de 
		chargement. Cet \\événement renvoie un dictionaire contenant les clés suivantes:
		\begin{itemize}
			\item [>> \textbf{\textcolor{darkgreen}{String} id}:] Contient l'identifiant du niveau en cours
			de chargement.
			\item [>> \textbf{\textcolor{red}{int} progress}:] Contient la valeur de progression du niveau 
			en cours de chargement.\\
		\end{itemize}
	\end{description}
	% level_loaded event description.
	\begin{description}
		\item [+ \textcolor{blue}{level\_loaded} (data):] Signal déclenché lorsqu'un niveau à finit d'être 
		chargé. Cet événement \\renvoie un dictionaire contenant les clés suivantes:
		\begin{itemize}
			\item [>> \textbf{\textcolor{darkgreen}{String} id}:] Contient l'identifiant du niveau ayant été
			chargé.
			\item [>> \textbf{\textcolor{red}{int} progress}:] Contient la valeur de progression du niveau 
			ayant été chargé.\\
		\end{itemize}
	\end{description}
	% level_opened event description.
	\begin{description}
		\item [+ \textcolor{blue}{level\_opened} (id):] Signal déclenché lorsqu'un niveau à été importé dans 
		l'arbre des scènes.
		\begin{itemize}
			\item [>> \textbf{\textcolor{darkgreen}{String} id}:] Contient l'identifiant du niveau récement
			ouvert.\\
		\end{itemize}
	\end{description}
	% level_closed event description.
	\begin{description}
		\item [+ \textcolor{blue}{level\_closed} (id):] Signal déclenché lorsqu'un niveau à été détruit ou 
		rendu invisible.
		\begin{itemize}
			\item [>> \textbf{\textcolor{darkgreen}{String} id}:] Contient l'identifiant du niveau récement
			détruit ou masqué.\\
		\end{itemize}
	\end{description}
	% scene_loading event description.
	\begin{description}
		\item [+ \textcolor{blue}{scene\_loading} (data):] Signal déclenché lorsqu'une scène est en cours de 
		chargement. Cet \\événement renvoie un dictionaire contenant les clés suivantes:
		\begin{itemize}
			\item [>> \textbf{\textcolor{darkgreen}{String} id}:] Contient l'identifiant de la scène en 
			cours de chargement.
			\item [>> \textbf{\textcolor{red}{int} progress}:] Contient la valeur de progression de la scène 
			en cours de chargement.\\
		\end{itemize}
	\end{description}
	% scene_loaded event description.
	\begin{description}
		\item [+ \textcolor{blue}{scene\_loaded} (data):] Signal déclenché lorsqu'une scène à finit d'être 
		chargé. Cet événement renvoie un dictionaire contenant les clés suivantes:
		\begin{itemize}
			\item [>> \textbf{\textcolor{darkgreen}{String} id}:] Contient l'identifiant de la scène ayant 
			été chargée.
			\item [>> \textbf{\textcolor{red}{int} progress}:] Contient la valeur de progression de la scène 
			récement chargée.\\
		\end{itemize}
	\end{description}
	% scene_opened event description.
	\begin{description}
		\item [+ \textcolor{blue}{scene\_opened} (id):] Signal déclenché lorsqu'une scène à été importé dans 
		l'arbre des scènes.
		\begin{itemize}
			\item [>> \textbf{\textcolor{darkgreen}{String} id}:] Contient l'identifiant de la scène 
			récement ouverte.\\
		\end{itemize}
	\end{description}
	% scene_closed event description.
	\begin{description}
		\item [+ \textcolor{blue}{scene\_closed} (id):] Signal déclenché lorsqu'une scène à été détruit ou 
		rendu invisible.
		\begin{itemize}
			\item [>> \textbf{\textcolor{darkgreen}{String} id}:] Contient l'identifiant de la scène 
			récement détruite ou masquée.\\
		\end{itemize}
	\end{description}
	% after_destroy event description.
	\begin{description}
		\item [+ \textcolor{blue}{after\_destroy} ():] Signal déclenché après la destruction d'une scène ou 
		d'un niveau du jeu.\\
	\end{description}
\end{document}