\documentclass[a4paper, 11pt]{article}
\usepackage[utf8]{inputenc}
\usepackage[T1]{fontenc}
\usepackage[french]{babel}
\usepackage{charter}
\usepackage[top = 2cm, bottom = 2cm, left = 1cm, right = 1cm]{geometry}
\usepackage{setspace}
\usepackage{color}
\usepackage{xcolor}
\usepackage{hyperref}
\usepackage{tocloft}

% Preanblue.
\onehalfspacing
\definecolor{gray}{rgb}{0.4, 0.4, 0.4}
\definecolor{silver}{rgb}{0.95, 0.95, 0.95}
\renewcommand{\thesection}{\Roman{section} --}
\definecolor{darkgreen}{HTML}{1E8C15}
\cftsetindents{section}{1em}{2.5em}
\hypersetup {colorlinks=true, linkcolor=blue, urlcolor=blue, pdftitle={OcclusionCulling module doc}}

% The start of the article.
\begin{document}
	% Change document background to silver color.
	\pagecolor{silver}
	% OcclusionCulling module description.
	\huge{\hspace{12cm}\textit{\textbf{\textcolor{darkgreen}{OcclusionCullingFx}}}}\large{} \tableofcontents
	\newpage
	% OcclusionCulling module definition.
	\section{Définition}
	\textcolor{darkgreen}{\textbf{OcclusionCullingFx}} est un module conçut pour l'optimisation de la 
	mémoire graphique de \\l'ordinateur. Son rôle, est de rendre visible un objet uniquement lorsqu'il se 
	trouve sur le champ de vision de la caméra. Cette technique soulage la carte graphique dans certaines 
	tâches. Mais la mémoire de l'ordinateur reste surchargée.\\
	\textcolor{red}{\textbf{NB}:} Ce module est de nature indestructible, est uniquement compatible à un jeu 
	3D et n'est pas \\sauvegardable.

	% OcclusionCulling properties definition.
	\section{Les propriétés disponibles}
	% ScanCount property.
	\textbf{+ \textcolor{red}{int} ScanCount = \textcolor{blue}{20}:} Contient le nombre de scan à éffectué 
	par second. La valeur de cette \\propriété est dans l'intervalle [\textcolor{blue}{0}; \textcolor{blue}
	{++}].\\\\
	% Accuracy property.
	\textbf{+ \textcolor{red}{float} Accuracy = \textcolor{blue}{800.0}:} Contient le degré de précision sur 
	l'affichage et le chachement des \\objets prises en charge par le module. La valeur de cette propriété 
	doit être dans l'intervalle [\textcolor{blue}{1.0}; \textcolor{blue}{1000.0}].\\\\
	% Targets property.
	\textbf{+ \textcolor{darkgreen}{Array} Targets:} Tableau de dictionnaires contenant toutes les 
	différentes configurations sur chaque objet prise en charge par le développeur. Les dictionnaires issus 
	de ce tableau supportent les clés \\suivantes:\\
	\begin{itemize}
		\item[>> \textbf{\textcolor{darkgreen}{String} id}:] Quel est l'identifiant du noeud à prendre en 
		charge ? L'utilisation de cette clé est \\obligatoire.\\
		\item[>> \textbf{\textcolor{red}{int} search = \textcolor{blue}{2}}:] Quel moyen utilisé pour 
		chercher le noeud à prendre en charge ? Notez que \\l'identifiant donné est pisté à par un programme 
		de recherche dans l'arbre de la scène en question. Les valeurs possibles sont:
		\begin{itemize}
			\item [-> \textbf{\textcolor{gray}{MegaAssets.NodeProperty.NAME} ou \textcolor{blue}{0}}:] 
			Trouve un noeud en utilisant son nom.
			\item [-> \textbf{\textcolor{gray}{MegaAssets.NodeProperty.GROUP} ou \textcolor{blue}{1}}:] 
			Trouve un noeud en utilisant le nom de son groupe.
			\item [-> \textbf{\textcolor{gray}{MegaAssets.NodeProerty.TYPE} ou \textcolor{blue}{2}}:] Trouve 
			un noeud en utilisant le nom de sa classe.
			\item [-> \textbf{\textcolor{gray}{MegaAssets.NodeProerty.ANY} ou \textcolor{blue}{3}}:] Trouve 
			un noeud en utilisant l'un des trois moyens cités plus haut.
		\end{itemize}
	\end{itemize}
\end{document}