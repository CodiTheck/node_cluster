% Document type and package imports.
\documentclass[a4paper, 11pt]{article}
\usepackage[utf8]{inputenc}
\usepackage[T1]{fontenc}
\usepackage[french]{babel}
\usepackage{charter}
\usepackage[top = 2cm, bottom = 2cm, left = 1cm, right = 1cm]{geometry}
\usepackage{setspace}
\usepackage{color}
\usepackage{xcolor}
\usepackage{hyperref}
\usepackage{tocloft}

% Preanblue.
\onehalfspacing
\definecolor{gray}{rgb}{0.4, 0.4, 0.4}
\definecolor{silver}{rgb}{0.95, 0.95, 0.95}
\renewcommand{\thesection}{\Roman{section} --}
\definecolor{darkgreen}{HTML}{1E8C15}
\cftsetindents{section}{1em}{2.5em}
\hypersetup {colorlinks=true, linkcolor=blue, urlcolor=blue, pdftitle={CameraEffects module doc}}

% The start of the article.
\begin{document}
	% Change document background to silver color.
	\pagecolor{silver}
	% CameraEffects module description.
	\huge{\hspace{13.3cm}\textit{\textbf{\textcolor{darkgreen}{CameraEffectsFx}}}}\large{} \tableofcontents
	\newpage
	% CameraEffects module definition.
	\section{Définition}
	\textcolor{darkgreen}{\textbf{CameraEffectsFx}} est un module permettant de faire des effets de camera 
	dans un jeu.\\
	\textcolor{red}{\textbf{NB}:} Ce module est compatible à un jeu 2D, 3D et est sauvegardable.

	% CameraEffects properties definition.
	\section{Les propriétés disponibles}
	% Target property.
	\textbf{+ \textcolor{darkgreen}{NodePath} Target:} Contient une référence de l'instance d'un noeud de 
	type \href{https://docs.godotengine.org/en/stable/classes/class_sprite.html}
	{\textit{\textcolor{darkgreen}{Sprite}}} ou
	\href{https://docs.godotengine.org/en/stable/classes/class_texturerect.html}
	{\textit{\textcolor{darkgreen}{TextureRect}}}.\\\\
	% Effect property.
	\textbf{+ \textcolor{red}{int} Effect = \textcolor{blue}{0}:} Contient l'éffet à appliqué. Les valeurs 
	possibles sont:
	\begin{itemize}
		\item [-> \textbf{\textcolor{gray}{MegaAssets.CameraEffect.NONE} ou \textcolor{blue}{0}}:] Aucun 
		éffet à appliqué.
		\item [-> \textbf{\textcolor{gray}{MegaAssets.CameraEffect.SIMPLE\_BLUR} ou \textcolor{blue}{1}}:] 
		Utilisation d'un simple éffet flou.
		\item [-> \textbf{\textcolor{gray}{MegaAssets.CameraEffect.VIGNETTE} ou \textcolor{blue}{2}}:] 
		Utilisation de l'éffet vignette.
		\item [-> \textbf{\textcolor{gray}{MegaAssets.CameraEffect.PIXELIZE} ou \textcolor{blue}{3}}:] 
		Utilisation de l'éffet de pixélisation.
		\item [-> \textbf{\textcolor{gray}{MegaAssets.CameraEffect.WHIRL} ou \textcolor{blue}{4}}:] 
		Utilisation de l'éffet whirl.
		\item [-> \textbf{\textcolor{gray}{MegaAssets.CameraEffect.SEPIA} ou \textcolor{blue}{5}}:] 
		Utilisation de l'éffet sepia.
		\item [-> \textbf{\textcolor{gray}{MegaAssets.CameraEffect.NEGATIVE} ou \textcolor{blue}{6}}:] 
		Utilisation de l'éffet negatve.
		\item [-> \textbf{\textcolor{gray}{MegaAssets.CameraEffect.CONTRASTED} ou \textcolor{blue}{7}}:] 
		Utilisation de l'éffet de contraste.
		\item [-> \textbf{\textcolor{gray}{MegaAssets.CameraEffect.NORMALIZED} ou \textcolor{blue}{8}}:] 
		Utilisation de l'éffet normalized.
		\item [-> \textbf{\textcolor{gray}{MegaAssets.CameraEffect.BCS} ou \textcolor{blue}{9}}:] 
		Utilisation de l'éffet bcs.
		\item [-> \textbf{\textcolor{gray}{MegaAssets.CameraEffect.MIRAGE} ou \textcolor{blue}{10}}:] 
		Utilisation de l'éffet mirage.
		\item [-> \textbf{\textcolor{gray}{MegaAssets.CameraEffect.OLD\_FILM} ou \textcolor{blue}{11}}:] 
		Utilisation de l'éffet d'ancien film.
		\item [-> \textbf{\textcolor{gray}{MegaAssets.CameraEffect.STATIC\_CRT} ou \textcolor{blue}{12}}:] 
		Utilisation de l'éffet crt statique.
		\item [-> \textbf{\textcolor{gray}{MegaAssets.CameraEffect.MOSAIC} ou \textcolor{blue}{13}}:] 
		Utilisation de l'éffet mosaic.
		\item [-> \textbf{\textcolor{gray}{MegaAssets.CameraEffect.LCD} ou \textcolor{blue}{14}}:] 
		Utilisation de l'éffet d'écran lcd.
		\item [-> \textbf{\textcolor{gray}{MegaAssets.CameraEffect.GAMEBOY} ou \textcolor{blue}{15}}:] 
		Utilisation de l'éffet gameboy.
		\item [-> \textbf{\textcolor{gray}{MegaAssets.CameraEffect.TONE\_COMIC} ou \textcolor{blue}{16}}:] 
		Utilisation de l'éffet tone comic.
		\item [-> \textbf{\textcolor{gray}{MegaAssets.CameraEffect.INVERT} ou \textcolor{blue}{17}}:] 
		Utilisation de l'éffet d'invertion.
		\item [-> \textbf{\textcolor{gray}{MegaAssets.CameraEffect.TV} ou \textcolor{blue}{18}}:] 
		Utilisation de l'éffet de la télévision.
		\item [-> \textbf{\textcolor{gray}{MegaAssets.CameraEffect.VHS} ou \textcolor{blue}{19}}:] 
		Utilisation de l'éffet vhs.
		\item [-> \textbf{\textcolor{gray}{MegaAssets.CameraEffect.VHS\_GLITCH} ou \textcolor{blue}{20}}:] 
		Utilisation de l'éffet vhs glich.
		\item [-> \textbf{\textcolor{gray}{MegaAssets.CameraEffect.VHS\_PAUSE} ou \textcolor{blue}{21}}:] 
		Utilisation de l'éffet vhs pause.
		\item [-> \textbf{\textcolor{gray}{MegaAssets.CameraEffect.VHS\_SIMPLE\_GLITCH} ou \textcolor{blue}
		{22}}:] Utilisation de l'éffet vhs simple glich.
		\item [-> \textbf{\textcolor{gray}{MegaAssets.CameraEffect.BW} ou \textcolor{blue}{23}}:] 
		Utilisation de l'éffet bw.
		\item [-> \textbf{\textcolor{gray}{MegaAssets.CameraEffect.BETTER\_CC} ou \textcolor{blue}{24}}:] 
		Utilisation de l'éffet better cc.
		\item [-> \textbf{\textcolor{gray}{MegaAssets.CameraEffect.COLOR\_PRECISION} ou \textcolor{blue}
		{25}}:] Utilisation de l'éffet de la précision de couleur.
		\item [-> \textbf{\textcolor{gray}{MegaAssets.CameraEffect.GRAIN} ou \textcolor{blue}{26}}:] 
		Utilisation de l'éffet grain.
		\item [-> \textbf{\textcolor{gray}{MegaAssets.CameraEffect.LENS\_DISTORTION} ou \textcolor{blue}
		{27}}:] Utilisation de l'éffet lens distortion.
		\item [-> \textbf{\textcolor{gray}{MegaAssets.CameraEffect.SHARPNESS} ou \textcolor{blue}{28}}:] 
		Utilisation de l'éffet sharpness.
		\item [-> \textbf{\textcolor{gray}{MegaAssets.CameraEffect.SIMPLE\_GRAIN} ou \textcolor{blue}{29}}:] 
		Utilisation de l'éffet simple grain.
		\item [-> \textbf{\textcolor{gray}{MegaAssets.CameraEffect.RANDOM\_NOISE} ou \textcolor{blue}{30}}:] 
		Utilisation de l'éffet random noise.
		\item [-> \textbf{\textcolor{gray}{MegaAssets.CameraEffect.SCANLINES} ou \textcolor{blue}{31}}:] 
		Utilisation de l'éffet scanlines.
		\item [-> \textbf{\textcolor{gray}{MegaAssets.CameraEffect.GLITCH} ou \textcolor{blue}{32}}:] 
		Utilisation de l'éffet glitch.
		\item [-> \textbf{\textcolor{gray}{MegaAssets.CameraEffect.CRT\_SCREEN} ou \textcolor{blue}{33}}:] 
		Utilisation de l'éffet crt.
		\item [-> \textbf{\textcolor{gray}{MegaAssets.CameraEffect.SIMPLE\_CRT} ou \textcolor{blue}{34}}:] 
		Utilisation de l'éffet crt simple.
		\item [-> \textbf{\textcolor{gray}{MegaAssets.CameraEffect.SIMPLE\_GLITCH} ou \textcolor{blue}
		{35}}:] Utilisation de l'éffet glitch simple.
		\item [-> \textbf{\textcolor{gray}{MegaAssets.CameraEffect.CRT\_LOTTES} ou \textcolor{blue}{36}}:] 
		Utilisation de l'éffet crt lottes.
		\item [-> \textbf{\textcolor{gray}{MegaAssets.CameraEffect.ABERRATION} ou \textcolor{blue}{37}}:] 
		Utilisation de l'éffet d'abération.
		\item [-> \textbf{\textcolor{gray}{MegaAssets.CameraEffect.ADVANCED\_MOSIC} ou \textcolor{blue}
		{38}}:] Utilisation de l'éffet mosic avancé.
		\item [-> \textbf{\textcolor{gray}{MegaAssets.CameraEffect.ANIMATED\_NOISE} ou \textcolor{blue}
		{39}}:] Utilisation de l'éffet d'animation de bruit.
		\item [-> \textbf{\textcolor{gray}{MegaAssets.CameraEffect.AVERAGE} ou \textcolor{blue}{40}}:] 
		Utilisation de l'éffet average.
		\item [-> \textbf{\textcolor{gray}{MegaAssets.CameraEffect.BACKGROUND} ou \textcolor{blue}{41}}:] 
		Utilisation d'un éffet utilisé comme arrière-plan.
		\item [-> \textbf{\textcolor{gray}{MegaAssets.CameraEffect.BINARY\_CONVERSION} ou \textcolor{blue}
		{42}}:] Utilisation de l'éffet de conversion binaire.
		\item [-> \textbf{\textcolor{gray}{MegaAssets.CameraEffect.BINARY\_DEFAULT\_MIX} ou \textcolor{blue}
		{43}}:] Utilisation de l'éffet de conversion binaire par défaut.
		\item [-> \textbf{\textcolor{gray}{MegaAssets.CameraEffect.COLOR\_BLINDNESS} ou \textcolor{blue}
		{44}}:] Utilisation de l'éffet de liaison de \\couleur.
		\item [-> \textbf{\textcolor{gray}{MegaAssets.CameraEffect.DEFAULT} ou \textcolor{blue}{45}}:] 
		Utilisation d'un éffet fainéant.
		\item [-> \textbf{\textcolor{gray}{MegaAssets.CameraEffect.EDGE\_DEFAULT\_MIX} ou \textcolor{blue}
		{46}}:] Utilisation de l'éffet egde default mix.
		\item [-> \textbf{\textcolor{gray}{MegaAssets.CameraEffect.EDGE\_MOTION\_MIX} ou \textcolor{blue}
		{47}}:] Utilisation de l'éffet edge motion mix.
		\item [-> \textbf{\textcolor{gray}{MegaAssets.CameraEffect.EDGE\_PREWITT} ou \textcolor{blue}{48}}:]
		Utilisation de l'éffet edge prewitt.
		\item [-> \textbf{\textcolor{gray}{MegaAssets.CameraEffect.SIMPLE\_EDGE} ou \textcolor{blue}{49}}:] 
		Utilisation de l'éffet edge.
		\item [-> \textbf{\textcolor{gray}{MegaAssets.CameraEffect.EDGE\_SOBEL} ou \textcolor{blue}{50}}:] 
		Utilisation de l'éffet edge sobel.
		\item [-> \textbf{\textcolor{gray}{MegaAssets.CameraEffect.MONOCHROME} ou \textcolor{blue}{51}}:] 
		Utilisation de l'éffet monochrome.
		\item [-> \textbf{\textcolor{gray}{MegaAssets.CameraEffect.MOTION} ou \textcolor{blue}{52}}:] 
		Utilisation de l'éffet de mouvement.
		\item [-> \textbf{\textcolor{gray}{MegaAssets.CameraEffect.SIMPLE\_MOSIC} ou \textcolor{blue}{53}}:]
		Utilisation de l'éffet mosic.
		\item [-> \textbf{\textcolor{gray}{MegaAssets.CameraEffect.MOTION\_BLUR} ou \textcolor{blue}{54}}:] 
		Utilisation de l'éffet du mouvement flou.\\
	\end{itemize}
	% Privot property.
	\textbf{+ \textcolor{darkgreen}{NodePath} Privot:} Contient une référence de l'instance d'un noeud de 
	type \href{https://docs.godotengine.org/en/stable/classes/class_camera2d.html}
	{\textit{\textcolor{darkgreen}{Camera2D}}} ou
	\href{https://docs.godotengine.org/en/stable/classes/class_camera.html}
	{\textit{\textcolor{darkgreen}{Camera}}}. N'utilisez cette propriété que si l'éffet choisi est un 
	\textit{\textcolor{gray}{Motion Blur}}.\\\\
	% ListenTransform property.
	\textbf{+ \textcolor{red}{int} ListenTransform = \textcolor{blue}{2}:} Quelle transformation écoutée 
	lorsque l'éffet choisi est un \textit{\textcolor{gray}{Motion Blur}} ? Les valeurs possibles sont:
	\begin{itemize}
		\item [-> \textbf{\textcolor{gray}{MegaAssets.Transformation.NONE} ou \textcolor{blue}{0}}:] Aucune 
		transformation ne sera écouter, que cela soit dans le plan ou dans l'espace.
		\item [-> \textbf{\textcolor{gray}{MegaAssets.Transformation.LOCATION} ou \textcolor{blue}{1}}:] 
		L'éffet se déclenchera uniquement au \\changement de la position du \textit{\textcolor{gray}
		{Privot}}.
		\item [-> \textbf{\textcolor{gray}{MegaAssets.Transformation.ROTATION} ou \textcolor{blue}{2}}:] 
		L'éffet se déclenchera uniquement au \\changement de la rotation du \textit{\textcolor{gray}
		{Privot}}.
		\newpage \item [-> \textbf{\textcolor{gray}{MegaAssets.Transformation.SCALE} ou \textcolor{blue}
		{3}}:] L'éffet se déclenchera uniquement au changement de la taille du \textit{\textcolor{gray}
		{Privot}}.
		\item [-> \textbf{\textcolor{gray}{MegaAssets.Transformation.LOCROT} ou \textcolor{blue}{4}}:] 
		L'éffet se déclenchera au changement de la \\rotation et de la position du \textit{\textcolor{gray}
		{Privot}}.
		\item [-> \textbf{\textcolor{gray}{MegaAssets.Transformation.LOCSCALE} ou \textcolor{blue}{5}}:] 
		L'éffet se déclenchera au changement de la \\position et de la taille du \textit{\textcolor{gray}
		{Privot}}.
		\item [-> \textbf{\textcolor{gray}{MegaAssets.Transformation.ROTSCALE} ou \textcolor{blue}{6}}:] 
		L'éffet se déclenchera au changement de la \\rotation et de la taille du \textit{\textcolor{gray}
		{Privot}}.
		\item [-> \textbf{\textcolor{gray}{MegaAssets.ListenTransform.ALL} ou \textcolor{blue}{7}}:] L'éffet
		se déclenchera quelque soit la \\transformation qui a été affecté.\\
	\end{itemize}
	% ListenAxis property.
	\textbf{+ \textcolor{red}{int} ListenAxis = \textcolor{blue}{2}:} Quelle axe écouté lorsque l'éffet 
	choisi est un \textit{\textcolor{gray}{Motion Blur}} ? Les valeurs possibles sont:
	\begin{itemize}
		\item [-> \textbf{\textcolor{gray}{MegaAssets.Axis.NONE} ou \textcolor{blue}{0}}:] Aucun axe ne sera 
		écouter, que cela soit dans le plan ou dans l'espace.
		\item [-> \textbf{\textcolor{gray}{MegaAssets.Axis.X} ou \textcolor{blue}{1}}:] L'éffet se 
		déclenchera uniquement au changement de valeur au niveau de l'axe des abcisses du 
		\textit{\textcolor{gray}{Privot}}.
		\item [-> \textbf{\textcolor{gray}{MegaAssets.Axis.XY} ou \textcolor{blue}{7}}:] L'éffet se 
		déclenchera uniquement au changement de valeur au \\niveau de l'axe des abcisses et des ordonnés du 
		\textit{\textcolor{gray}{Privot}}.
		\item [-> \textbf{\textcolor{gray}{MegaAssets.Axis.XZ} ou \textcolor{blue}{8}}:] L'éffet se 
		déclenchera uniquement au changement de valeur au \\niveau de l'axe des abcisses et de Z du 
		\textit{\textcolor{gray}{Privot}}.
		\item [-> \textbf{\textcolor{gray}{MegaAssets.Axis.Y} ou \textcolor{blue}{2}}:] L'éffet se 
		déclenchera uniquement au changement de valeur au niveau de l'axe des ordonnés du 
		\textit{\textcolor{gray}{Privot}}.
		\item [-> \textbf{\textcolor{gray}{MegaAssets.Axis.YZ} ou \textcolor{blue}{9}}:] L'éffet se 
		déclenchera uniquement au changement de valeur au \\niveau de l'axe des ordonnés et de Z du 
		\textit{\textcolor{gray}{Privot}}.
		\item [-> \textbf{\textcolor{gray}{MegaAssets.Axis.Z} ou \textcolor{blue}{3}}:] L'éffet se 
		déclenchera uniquement au changement de valeur au niveau de l'axe Z du \textit{\textcolor{gray}
		{Privot}}.
		\item [-> \textbf{\textcolor{gray}{MegaAssets.Axis.XYZ} ou \textcolor{blue}{13}}:] L'éffet se 
		déclenchera uniquement au changement de valeur au niveau de l'axe des abcisses, des ordonnés et de Z 
		du \textit{\textcolor{gray}{Privot}}.\\
	\end{itemize}
	% FullScreen property.
	\textbf{+ \textcolor{red}{int} FullScreen = \textcolor{blue}{0}:} Redimension le sprite en question pour 
	que sa taille occupe tous l'espace \\disponible sur l'écran. Les valeurs possibles sont:
	\begin{itemize}
		\item [-> \textbf{\textcolor{gray}{MegaAssets.FullMonitor.NONE} ou \textcolor{blue}{0}}:] Aucun 
		redimensionement à appliqué.
		\item [-> \textbf{\textcolor{gray}{MegaAssets.FullMonitor.HORIZONTAL} ou \textcolor{blue}{1}}:] 
		Redimensionement horizontal.
		\item [-> \textbf{\textcolor{gray}{MegaAssets.FullMonitor.VERTICAL} ou \textcolor{blue}{2}}:] 
		Redimensionement vertical.
		\item [-> \textbf{\textcolor{gray}{MegaAssets.FullMonitor.BOTH} ou \textcolor{blue}{3}}:] 
		Redimensionement horizontal et vertical.\\
	\end{itemize}
	% Responsive property.
	\textbf{+ \textcolor{red}{int} Responsive = \textcolor{red}{true}:} Voulez-vous adapter les positions et 
	la taille de l'éffet en fonction des \\dimensions de l'écran ? Les valeurs possibles sont:
	\begin{itemize}
		\item [-> \textbf{\textcolor{gray}{MegaAssets.FullMonitor.NONE} ou \textcolor{blue}{0}}:] Aucun 
		responsive à appliqué.
		\item [-> \textbf{\textcolor{gray}{MegaAssets.FullMonitor.HORIZONTAL} ou \textcolor{blue}{1}}:] 
		Responsive horizontal.
		\item [-> \textbf{\textcolor{gray}{MegaAssets.FullMonitor.VERTICAL} ou \textcolor{blue}{2}}:] 
		Responsive vertical.
		\item [-> \textbf{\textcolor{gray}{MegaAssets.FullMonitor.BOTH} ou \textcolor{blue}{3}}:] Responsive 
		horizontal et vertical.\\
	\end{itemize}
	% Layout property.
	\textbf{+ \textcolor{red}{int} Layout = \textcolor{blue}{0}:} Voulez-vous adapter les positions et 
	la taille du sprite en fonction des dimensions de l'écran. Les valeurs possibles sont:
	\begin{itemize}
		\item [-> \textbf{\textcolor{gray}{MegaAssets.Disposal.NONE} ou \textcolor{blue}{0}}:] Aucune 
		disposition ne sera appliquée.
		\item [-> \textbf{\textcolor{gray}{MegaAssets.Disposal.CENTER} ou \textcolor{blue}{1}}:] Centrage 
		automatique.
		\item [-> \textbf{\textcolor{gray}{MegaAssets.Disposal.TOP} ou \textcolor{blue}{2}}:] Positionement
		en haut de l'écran.
		\item [-> \textbf{\textcolor{gray}{MegaAssets.Disposal.RIGHT} ou \textcolor{blue}{3}}:]
		Positionement à droite de l'écran.
		\item [-> \textbf{\textcolor{gray}{MegaAssets.Disposal.BOTTOM} ou \textcolor{blue}{4}}:]
		Positionement en bas de l'écran.
		\item [-> \textbf{\textcolor{gray}{MegaAssets.Disposal.LEFT} ou \textcolor{blue}{5}}:] Positionement
		à gauche de l'écran.\\
	\end{itemize}
	% Transition property.
	\textbf{+ \textcolor{red}{float} Transition = \textcolor{blue}{0.0}:} Quel est le temps mort entre deux
	éffets ? Cette propriété est solicité lorsque l'on change d'éffet.\\\\
	% TransitionType property.
	\textbf{+ \textcolor{red}{int} TransitionType = \textcolor{blue}{0}:} Quel type de transition adopté ?
	Les valeurs possibles de ce champ sont celles de Godot. Cette propriété est solicité lorsque l'on change 
	d'éffet.\\\\
	% TransitionType property.
	\textbf{+ \textcolor{red}{int} TransitionEasing = \textcolor{blue}{2}:} Quel assouplissement adopté ? 
	Les valeurs possibles de ce champ sont celles de Godot. Cette propriété est solicité lorsque l'on change 
	d'éffet.\\\\
	% TransitionBound property.
	\textbf{+ \textcolor{red}{bool} TransitionBound = \textcolor{red}{false}:} Voulez-vous établir une 
	liaison entre les éffets ? Cette propriété est solicité lorsque l'on change d'éffet.

	% CameraEffects methods definition.
	\section{Les méthodes disponibles}
	% Void interpolate_visibility () method description.
	\begin{description}
		\item [+ \textcolor{red}{void} \textcolor{blue}{interpolate\_visibility} (value, min, max, invert = 
		false, delay = 0.0):] Interpolation de \\visibilité au niveau du sprite de l'éffet en question. 
		Cette méthode agit sur la propriété \textit{\textcolor{gray}{modulate}} du sprite présentant l'éffet 
		choisi.
		\begin{itemize}
			\item [>> \textbf{\textcolor{red}{float | int} value}:] Contient l'état de la visibilité.
			\item [>> \textbf{\textcolor{red}{float | int} min}:] Contient la valeur minimale de la 
			visibilité.
			\item [>> \textbf{\textcolor{red}{float | int} max}:] Contient la valeur maximale de la 
			visibilité.
			\item [>> \textbf{\textcolor{red}{bool} invert}:] La modification de la valeur de la visibilité
			doit-elle se fait dans le sens inverse de son état actuelle ?
			\item [>> \textbf{\textcolor{red}{float} delay}:] Quel est le temps mort avant le changement 
			d'état ?
		\end{itemize}
	\end{description}

	% CameraEffects signals definition.
	\section{Les événements disponibles}
	% transition_started event description.
	\begin{description}
		\item [+ \textcolor{blue}{transition\_started} (node):] Signal déclenché avant le changement d'éffet 
		avec une transition \\supérieur à 0.
		\begin{itemize}
			\item [>> \textbf{\textcolor{darkgreen}{Node} node}:] Contient le noeud où cet signal a été 
			émit.\\
		\end{itemize}
	\end{description}
	% transition_finished event description.
	\newpage \begin{description}
		\item [+ \textcolor{blue}{transition\_finished} (node):] Signal déclenché après le changement 
		d'éffet avec une transition supérieur à 0.
		\begin{itemize}
			\item [>> \textbf{\textcolor{darkgreen}{Node} node}:] Contient le noeud où cet signal a été 
			émit.\\
		\end{itemize}
	\end{description}
	% transition_running event description.
	\begin{description}
		\item [+ \textcolor{blue}{transition\_running} (node):] Signal déclenché au cours d'une transition
		entre deux éffets.
		\begin{itemize}
			\item [>> \textbf{\textcolor{darkgreen}{Node} node}:] Contient le noeud où cet signal a été 
			émit.\\
		\end{itemize}
	\end{description}
	% effect_changed event description.
	\begin{description}
		\item [+ \textcolor{blue}{effect\_changed} (node):] Signal déclenché au changement d'éffet.
		\begin{itemize}
			\item [>> \textbf{\textcolor{darkgreen}{Node} node}:] Contient le noeud où cet signal a été 
			émit.\\
		\end{itemize}
	\end{description}
\end{document}