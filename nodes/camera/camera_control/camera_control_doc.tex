% Document type and package imports.
\documentclass[a4paper, 11pt]{article}
\usepackage[utf8]{inputenc}
\usepackage[T1]{fontenc}
\usepackage[french]{babel}
\usepackage{charter}
\usepackage[top = 2cm, bottom = 2cm, left = 1cm, right = 1cm]{geometry}
\usepackage{setspace}
\usepackage{color}
\usepackage{xcolor}
\usepackage{hyperref}
\usepackage{tocloft}

% Preanblue.
\onehalfspacing
\definecolor{gray}{rgb}{0.4, 0.4, 0.4}
\definecolor{silver}{rgb}{0.95, 0.95, 0.95}
\renewcommand{\thesection}{\Roman{section} --}
\definecolor{darkgreen}{HTML}{1E8C15}
\cftsetindents{section}{1em}{2.5em}
\hypersetup {colorlinks=true, linkcolor=blue, urlcolor=blue, pdftitle={CameraControl module doc}}

% The start of the article.
\begin{document}
	% Change document background to silver color.
	\pagecolor{silver}
	% CameraControl module description.
	\huge{\hspace{12.5cm}\textit{\textbf{\textcolor{darkgreen}{CameraControlFx}}}}\large{} \tableofcontents
	\newpage
	% CameraControl module definition.
	\section{Définition}
	\textcolor{darkgreen}{\textbf{CameraControlFx}} est un module conçut pour la gestion des caméras dans un
	jeu vidéo. Le module offre également aux développeurs des fonctionnalités afin de leur permettre de 
	configurer facilement la caméra de leur joueur.\\
	\textcolor{red}{\textbf{NB}:} Ce module est compatible à un jeu 3D et n'est pas sauvegardable.

	% CameraControl properties definition.
	\section{Les propriétés disponibles}
	% TargetCamera property.
	\textbf{+ \textcolor{darkgreen}{NodePath} TargetCamera:} Contient l'instance d'un noeud de type 
	\href{https://docs.godotengine.org/en/stable/classes/class_camera.html}
	{\textit{\textcolor{darkgreen}{Camera}}}.\\\\
	% Privot property.
	\textbf{+ \textcolor{darkgreen}{NodePath} Privot:} Contient l'instance d'un noeud de type 
	\href{https://docs.godotengine.org/en/stable/classes/class_spatial.html}
	{\textit{\textcolor{darkgreen}{Spatial}}}.\\\\
	% Mode property.
	\textbf{+ \textcolor{red}{int} Mode = \textcolor{blue}{0}:} Contient le mode de la caméra que vous 
	souhaitez utiliser. Les valeurs possibles sont:
	\begin{itemize}
		\item [-> \textbf{\textcolor{gray}{CameraControlFx.Model.FPS} ou \textcolor{blue}{0}}:] Mode FPS 
		(First Person Shooter) activé.
		\item [-> \textbf{\textcolor{gray}{CameraControlFx.Model.THIRD\_PERSON} ou \textcolor{blue}{1}}:] 
		Configuration de la caméra à la troisième \\personne.\\
	\end{itemize}
	% KeepTransform property.
	\textbf{+ \textcolor{red}{int} KeepTransform = \textcolor{blue}{0}:} Contrôle la transformation que doit
	écouté la caméra afin de pourvoir bien suivre sa cible.
	\begin{itemize}
		\item [-> \textbf{\textcolor{gray}{MegaAssets.Transformation.LOCATION} ou \textcolor{blue}{1}}:] La 
		caméra suivra sa cible en écoutant sa \\position.
		\item [-> \textbf{\textcolor{gray}{MegaAssets.Transformation.LOCROT} ou \textcolor{blue}{4}}:] La 
		caméra suivra sa cible en écoutant sa position ainsi que sa rotation.\\
	\end{itemize}
	% Distance property.
	\textbf{+ \textcolor{red}{float} Distance = \textcolor{blue}{5.0}:} Quelle est la distance de la caméra 
	par rapport à sa cible ?\\\\
	% OffsetPosition property.
	\textbf{+ \textcolor{darkgreen}{Vector3} \hypertarget{ofstpos}{OffsetPosition} = \textcolor{darkgreen}
	{Vector3} (\textcolor{blue}{0.0}, \textcolor{blue}{0.0}, \textcolor{blue}{0.0}):} Contrôle l'ajustement 
	de la caméra en terme de position.\\\\
	% Zooming property.
	\textbf{+ \textcolor{darkgreen}{Vector3} Zooming = \textcolor{darkgreen}{Vector3} (\textcolor{blue}
	{0.0}, \textcolor{blue}{100.0}, \textcolor{blue}{0.0}):} Contrôle le niveau d'agrandissement de la 
	\\caméra. La première valeur est le zoom minimum; la deuxième, le zoom maximum et la troisième le taux 
	d'agrandissement actuelle de la caméra.\\\\
	% ZoomTurbulence property.
	\textbf{+ \textcolor{red}{bool} ZoomTurbulence = \textcolor{blue}{false}:} La caméra va t-elle gérée 
	automatiquement l'agrandissement en fonction de la vitesse de sa cible par rapport l'agrandissement fixé 
	?\\\\
	% Layout property.
	\textbf{+ \textcolor{red}{int} Layout = \textcolor{blue}{2}:} Contrôle l'angle de direction de la 
	caméra. Les valeurs possibles sont:
	\begin{itemize}
		\item [-> \textbf{\textcolor{gray}{MegaAssets.Disposal.TOP} ou \textcolor{blue}{2}}:] Vue de haut.
		\item [-> \textbf{\textcolor{gray}{MegaAssets.Disposal.BOTTOM} ou \textcolor{blue}{4}}:] Vue de bas.
		\item [-> \textbf{\textcolor{gray}{MegaAssets.Disposal.FORWARD} ou \textcolor{blue}{6}}:] Vue de
		face.
		\item [-> \textbf{\textcolor{gray}{MegaAssets.Disposal.BACKWARD} ou \textcolor{blue}{7}}:] Vue de
		dos.
		\item [-> \textbf{\textcolor{gray}{MegaAssets.Disposal.LEFT} ou \textcolor{blue}{5}}:] Vue de
		gauche.
		\item [-> \textbf{\textcolor{gray}{MegaAssets.Disposal.RIGHT} ou \textcolor{blue}{3}}:] Vue de
		droite.\\
	\end{itemize}
	% OffsetRotation property.
	\textbf{+ \textcolor{darkgreen}{Vector3} \hypertarget{ofstrot}{OffsetRotation} = \textcolor{darkgreen}
	{Vector3} (\textcolor{blue}{0.0}, \textcolor{blue}{0.0}, \textcolor{blue}{0.0}):} Contrôle l'ajustement 
	de la caméra en terme de rotation.\\\\
	% RotationLimit property.
	\textbf{+ \textcolor{darkgreen}{Vector2} RotationLimit = \textcolor{darkgreen}{Vector2} 
	(\textcolor{blue}{360.0}, \textcolor{blue}{360.0}):} Contient l'angle limite de la rotation de la 
	caméra.\\\\
	% Velocity property.
	\textbf{+ \textcolor{darkgreen}{Vector2} Velocity = \textcolor{darkgreen}{Vector2} (\textcolor{blue}
	{0.5}, \textcolor{blue}{0.5}):} Contrôle la sensitivité et l'adoucissement de la rotation de la caméra. 
	La première valeur est la sensitivité de la rotation et la deuxième, l'adoucissement.\\\\
	% Cinematic property.
	\textbf{+ \textcolor{darkgreen}{Vector2} Cinematic = \textcolor{darkgreen}{Vector2} (\textcolor{blue}
	{1.0}, \textcolor{blue}{0.1}):} Contrôle l'accélération et la décélération des \\mouvements de la 
	caméra. La première valeur est l'accélération et la deuxième, la décélération.\\\\
	% MaxSpeed property.
	\textbf{+ \textcolor{darkgreen}{Vector3} MaxSpeed = \textcolor{darkgreen}{Vector3} (\textcolor{blue}
	{1.0}, \textcolor{blue}{1.0}, \textcolor{blue}{1.0}):} Contrôle la vitesse des movements de la caméra.
	\\\\
	% Collision property.
	\textbf{+ \textcolor{red}{bool} Collision = \textcolor{blue}{false}:} Désirez-vous que la caméra percute 
	tous les objets dotés d'un \\collisionneur ?\\\\
	% Local property.
	\textbf{+ \textcolor{red}{bool} Local = \textcolor{blue}{true}:} Les mouvements de caméra vont-ils se 
	faire dans un monde local ?\\\\
	% RotatePrivot property.
	\textbf{+ \textcolor{red}{bool} RotatePrivot = \textcolor{blue}{false}:} Désirez-vous tourner l'objet 
	pivot que chasse la caméra ?\\\\
	% AutoReset property.
	\textbf{+ \textcolor{red}{int} AutoReset = \textcolor{blue}{0}:} Voulez-vous renitialiser les mouvements 
	de la caméra de façon automatique ? Les valeurs possibles sont:
	\begin{itemize}
		\item [-> \textbf{\textcolor{gray}{CameraControlFx.Reset.NONE} ou \textcolor{blue}{0}}:] Aucune 
		rénitialisation ne sera éffectuée.
		\item [-> \textbf{\textcolor{gray}{CameraControlFx.Reset.At\_ONCE} ou \textcolor{blue}{1}}:] 
		Rénitialisation immédiate de la caméra.
		\item [-> \textbf{\textcolor{gray}{CameraControlFx.Reset.GRADUALLY} ou \textcolor{blue}{2}}:] 
		Rénitialisation progressive de la caméra. Ce type de rénitialisation suit le déplacement de sa
		cible.\\
	\end{itemize}
	% Gap property.
	\textbf{+ \textcolor{darkgreen}{Vector2} \hypertarget{gap}{Gap} = \textcolor{darkgreen}{Vector2} 
	(\textcolor{blue}{0.0}, \textcolor{blue}{0.0}):} A quel angle ou à quelle position de la cible devrons 
	nous \\rénitialisé la caméra ? Cette option peut s'avérée très pratique dans certains cas d'utilisation. 
	La première valeur est l'angle à atteindre pour rénitialiser la caméra et la deuxième, la distance à 
	atteindre.\\\\
	% ShakeAxis property.
	\textbf{+ \textcolor{darkgreen}{Vector2} \hypertarget{shkaxis}{ShakeAxis} = \textcolor{darkgreen}
	{Vector2} (\textcolor{blue}{25.0}, \textcolor{blue}{25.0}):} Quel sont les axes qui seront affectés par 
	la \\vibration.\\\\
	% ShakeDensity property.
	\textbf{+ \textcolor{darkgreen}{Vector2} \hypertarget{shkden}{ShakeDensity} = \textcolor{darkgreen}
	{Vector2} (\textcolor{blue}{25.0}, \textcolor{blue}{1.0}):} Contient la densité de la vibration de la 
	caméra. La première valeur contient le taux de roulage et la second, l'intensité de la vibration.\\\\	
	% TargetingSystem property.
	\textbf{+ \textcolor{red}{bool} TargetingSystem = \textcolor{blue}{false}:} Désirez-vous activer le 
	sytème de pistage de la caméra ?\\\\
	% TargetingAera property.
	\textbf{+ \textcolor{darkgreen}{NodePath} TargetingAera:} Contient l'instance d'un noeud de type 
	\href{https://docs.godotengine.org/en/stable/classes/class_area.html}
	{\textit{\textcolor{darkgreen}{Area}}}.\\\\
	% Targets property.
	\textbf{+ \textcolor{darkgreen}{Array} Targets:} Tableau de dictionnaires contenant toutes les 
	différentes configurations sur chaque objet prise en charge par le développeur. Les dictionnaires issus 
	de ce tableau supportent les clés \\suivantes:\\
	\begin{itemize}
		\item[>> \textbf{\textcolor{darkgreen}{String} id}:] Quel est l'identifiant du noeud à prendre en 
		charge ? L'utilisation de cette clé est \\obligatoire.\\
		\item[>> \textbf{\textcolor{red}{int} search = \textcolor{blue}{3}}:] Quel moyen utilisé pour 
		chercher le noeud à prendre en charge ? Notez que \\l'identifiant donné est pisté à par un programme 
		de recherche. Les valeurs possibles sont:
		\begin{itemize}
			\item [-> \textbf{\textcolor{gray}{MegaAssets.NodeProperty.NAME} ou \textcolor{blue}{0}}:] 
			Trouve un noeud en utilisant son nom.
			\item [-> \textbf{\textcolor{gray}{MegaAssets.NodeProperty.GROUP} ou \textcolor{blue}{1}}:] 
			Trouve un noeud en utilisant le nom de son groupe.
			\item [-> \textbf{\textcolor{gray}{MegaAssets.NodeProerty.TYPE} ou \textcolor{blue}{2}}:] Trouve 
			un noeud en utilisant le nom de sa classe.
			\item [-> \textbf{\textcolor{gray}{MegaAssets.NodeProerty.ANY} ou \textcolor{blue}{3}}:] Trouve 
			un noeud en utilisant l'un des trois moyens cités plus haut.\\
		\end{itemize}
		\item[>> \textbf{\textcolor{red}{bool} ignored = \textcolor{red}{false}}:] Le pisteur de la caméra 
		doit-il ignoré l'identifiant précisé ?\\
		\item[>> \textbf{\textcolor{red}{float} transition = \textcolor{blue}{1.0}}:] Combien de temps prend 
		le passage d'une cible à une autre ?\\
		\item[>> \textbf{\textcolor{red}{int} type = \textcolor{blue}{0}}:] Quel type de transition adopté ?
		Les valeurs possibles de ce champ sont celles de Godot. Cette propriété est solicité au changement 
		de cible.\\
		\item[>> \textbf{\textcolor{red}{int} easing = \textcolor{blue}{2}}:] Quel assouplissement adopté ? 
		Les valeurs possibles de ce champ sont celles de Godot. Cette propriété est solicité au changement 
		de cible.\\
		\item[>> \textbf{\textcolor{darkgreen}{Array | Dictionary} entered}:] Signal déclenché lorsque
		l'objet entre dans le champ de vision du pisteur. Cette clé exécute les différentes actions données
		à son déclenchement. Pour soumettre les actions à exécutées référez vous à la méthode utilisée au
		niveau de la clé \textit{\textcolor{gray}{actions}} de la propriété \textit{\textcolor{gray}
		{EventsBindings}} dans les bases du framework.\\
		\item[>> \textbf{\textcolor{darkgreen}{Array | Dictionary} exited}:] Signal déclenché lorsqu'un
		objet sort du champ de vision du pisteur. Cette clé exécute les différentes actions données à son
		déclenchement. Pour soumettre les \\actions à exécutées référez vous à la méthode utilisée au niveau
		de la clé \textit{\textcolor{gray}{actions}} de la propriété \textit{\textcolor{gray}
		{EventsBindings}} dans les bases du framework.\\
	\end{itemize}
	% TargetingDirection property.
	\textbf{+ \textcolor{red}{int} TargetingDirection = \textcolor{blue}{2}:} Sur quelle direction les 
	pistages se feront ? Les valeurs possibles sont:
	\begin{itemize}
		\item [-> \textbf{\textcolor{gray}{MegaAssets.Axis.X} ou \textcolor{blue}{1}}:] L'axe des absisses.
		\item [-> \textbf{\textcolor{gray}{MegaAssets.Axis.Y} ou \textcolor{blue}{2}}:] L'axe des ordonnés.
		\item [-> \textbf{\textcolor{gray}{MegaAssets.Axis.Z} ou \textcolor{blue}{3}}:] L'axe des côtes.
		\item [-> \textbf{\textcolor{gray}{MegaAssets.Axis.\_X} ou \textcolor{blue}{4}}:] L'opposé de l'axe 
		des absisses.
		\item [-> \textbf{\textcolor{gray}{MegaAssets.Axis.\_Y} ou \textcolor{blue}{5}}:] L'opposé de l'axe 
		des ordonnés.
		\item [-> \textbf{\textcolor{gray}{MegaAssets.Axis.\_Z} ou \textcolor{blue}{6}}:] L'opposé de l'axe 
		des côtes.\\
	\end{itemize}
	% FroozenTargetingAxis property.
	\textbf{+ \textcolor{red}{int} FroozenTargetingAxis = \textcolor{blue}{1}:} Quel axe bloqué au cours des 
	pistages ? Les valeurs possibles sont:
	\begin{itemize}
		\item [-> \textbf{\textcolor{gray}{MegaAssets.Axis.NONE} ou \textcolor{blue}{0}}:] Aucun blockage.
		\item [-> \textbf{\textcolor{gray}{MegaAssets.Axis.X} ou \textcolor{blue}{1}}:] Blockage de l'axe 
		des absisses.
		\item [-> \textbf{\textcolor{gray}{MegaAssets.Axis.Y} ou \textcolor{blue}{2}}:] Blockage de l'axe 
		des ordonnés.
		\item [-> \textbf{\textcolor{gray}{MegaAssets.Axis.Z} ou \textcolor{blue}{3}}:] Bockage de l'axe des 
		côtes.
		\item [-> \textbf{\textcolor{gray}{MegaAssets.Axis.XY} ou \textcolor{blue}{7}}:] Bockage des axes x 
		et y.
		\item [-> \textbf{\textcolor{gray}{MegaAssets.Axis.XZ} ou \textcolor{blue}{8}}:] Bockage des axes x 
		et z.
		\item [-> \textbf{\textcolor{gray}{MegaAssets.Axis.YZ} ou \textcolor{blue}{9}}:] Bockage des axes y 
		et z.
		\item [-> \textbf{\textcolor{gray}{MegaAssets.Axis.XYZ} ou \textcolor{blue}{10}}:] Bockage des axes 
		x, y et z.\\
	\end{itemize}
	% TargetingPriority property.
	\textbf{+ \textcolor{red}{bool} TargetingPriority = \textcolor{blue}{true}:} Devront nous rendre
	prioritaire le pistage automatique devant les actions du joueur ? En d'autres termes, le joueur peut-il 
	manipuler la caméra lorsqu'elle piste \\automatique un objet ?

	% CameraControl methods definition.
	\section{Les méthodes disponibles}
	% Void reset_camera () method description.
	\begin{description}
		\item [+ \textcolor{red}{void} \textcolor{blue}{reset\_camera} (delay = 0.0):] Rénitialise la caméra
		par rapport à sa cible.
		\begin{itemize}
			\item [>> \textbf{\textcolor{red}{float} delay}:] Quel est le temps mort avant la 
			rénitialisation ?\\
		\end{itemize}
	\end{description}
	% Void change_target () method description.
	\begin{description}
		\item [+ \textcolor{red}{void} \textcolor{blue}{change\_target} (index = -1, delay = 0.0):] Force la 
		caméra à changé de cible parmit celles détectées. Par défaut, une cible est générée si l'index de la 
		nouvelle cible n'a pas été donné.
		\begin{itemize}
			\item [>> \textbf{\textcolor{red}{int} index}:] Contient l'index de la nouvelle cible à pistée.
			\item [>> \textbf{\textcolor{red}{float} delay}:] Quel est le temps mort avant le changement ?\\
		\end{itemize}
	\end{description}
	% Void zoom () method description.
	\begin{description}
		\item [+ \textcolor{red}{void} \textcolor{blue}{zoom} (depth = -1, transition = 1.0, type = 0, 
		easing = 2, delay = 0.0):] Génère un éffet d'agrandissement en avant.
		\begin{itemize}
			\item [>> \textbf{\textcolor{red}{float} depth}:] Contient le degré d'agrandissement. Par 
			défaut, celui précisé dans les \\configurations de la caméra est utilisé pour faire le 
			traitement.
			\item[>> \textbf{\textcolor{red}{float} transition}:] L'agrandissement se fera sur combien de 
			temps ?
			\item[>> \textbf{\textcolor{red}{int} type}:] Quel type de transition adopté ? Les valeurs 
			possibles de ce champ sont celles de Godot.
			\item[>> \textbf{\textcolor{red}{int} easing}:] Quel assouplissement adopté ? Les valeurs 
			possibles de ce champ sont celles de Godot.
			\item [>> \textbf{\textcolor{red}{float} delay}:] Quel est le temps mort avant l'agrandissement 
			?\\
		\end{itemize}
	\end{description}
	% Void unzoom () method description.
	\begin{description}
		\item [+ \textcolor{red}{void} \textcolor{blue}{unzoom} (depth = -1, transition = 1.0, type = 0, 
		easing = 2, delay = 0.0):] Génère un \\éffet d'agrandissement en arrière.
		\begin{itemize}
			\item [>> \textbf{\textcolor{red}{float} depth}:] Contient le degré de désagrandissement. Par 
			défaut, celui précisé dans les \\configurations de la caméra est utilisé pour faire le 
			traitement.
			\item[>> \textbf{\textcolor{red}{float} transition}:] Le désagrandissement se fera sur combien 
			de temps ?
			\item[>> \textbf{\textcolor{red}{int} type}:] Quel type de transition adopté ? Les valeurs 
			possibles de ce champ sont celles de Godot.
			\item[>> \textbf{\textcolor{red}{int} easing}:] Quel assouplissement adopté ? Les valeurs 
			possibles de ce champ sont celles de Godot.
			\item [>> \textbf{\textcolor{red}{float} delay}:] Quel est le temps mort avant le 
			désagrandissement ?\\
		\end{itemize}
	\end{description}
	% Void zooming () method description.
	\begin{description}
		\item [+ \textcolor{red}{void} \textcolor{blue}{zooming} (velocity, depth = -1):] Etablit un éffet
		d'agrandissement en fonction de la valeur contenu dans le paramètre \textit{\textcolor{gray}
		{velocity}}.
		\begin{itemize}
			\item [>> \textbf{\textcolor{red}{float} depth}:] Contient le degré de désagrandissement. Par 
			défaut, celui précisé dans les \\configurations de la caméra est utilisé pour faire le 
			traitement.
			\item [>> \textbf{\textcolor{red}{float} velocity}:] Contient la vélocité de l'agrandissement.\\
		\end{itemize}
	\end{description}
	% Void unzooming () method description.
	\begin{description}
		\item [+ \textcolor{red}{void} \textcolor{blue}{unzooming} (velocity, depth = -1):] Etablit un éffet 
		de désagrandissement en fonction de la valeur contenu dans le paramètre \textit{\textcolor{gray}
		{velocity}}.
		\begin{itemize}
			\item [>> \textbf{\textcolor{red}{float} depth}:] Contient le degré de désagrandissement. Par 
			défaut, celui précisé dans les \\configurations de la caméra est utilisé pour faire le 
			traitement.
			\item [>> \textbf{\textcolor{red}{float} velocity}:] Contient la vélocité du désagrandissement.
			\\
		\end{itemize}
	\end{description}
	% Dictionary get_targets_data () method description.
	\begin{description}
		\item [+ \textcolor{darkgreen}{Dictionary} \textcolor{blue}{get\_targets\_data} (json = false):] 
		Renvoie toutes les données concernant les cibles de la caméra.
		\begin{itemize}
			\item [>> \textbf{\textcolor{red}{bool} json}:] Voulez-vous renvoyer les données au format json 
			?\\
		\end{itemize}
	\end{description}
	% Void rot () method description.
	\begin{description}
		\item [+ \textcolor{red}{void} \textcolor{blue}{rot} (velocity):] Pivote la caméra. Cette méthode
		s'adapte en fonction du mode choisi par le développeur.
		\begin{itemize}
			\item [>> \textbf{\textcolor{darkgreen}{Vector2 | Vector3} depth}:] Contient la vélocity du 
			pivotement.\\
		\end{itemize}
	\end{description}
	% Void local_translation () method description.
	\begin{description}
		\item [+ \textcolor{red}{void} \textcolor{blue}{local\_translation} (velocity):] Déplace la caméra
		d'une position à une autre. Cette méthode agit sur la propriété \textit{\hyperlink{ofstpos}
		{OffsetPosition}} pour provoquer un mouvement de translation dans les axes.
		\begin{itemize}
			\item [>> \textbf{\textcolor{darkgreen}{Vector2 | Vector3} depth}:] Contient la vélocity du 
			déplacement.\\
		\end{itemize}
	\end{description}
	% Void local_rotation () method description.
	\begin{description}
		\item [+ \textcolor{red}{void} \textcolor{blue}{local\_rotation} (velocity):] Pivote la caméra d'un 
		ange à un autre. Cette méthode agit sur la propriété \textit{\hyperlink{ofstrot}{OffsetRotation}} 
		pour provoquer un mouvement de rotation dans les axes.
		\begin{itemize}
			\item [>> \textbf{\textcolor{darkgreen}{Vector2 | Vector3} depth}:] Contient la vélocity du 
			pivotement.\\
		\end{itemize}
	\end{description}
	% Spatial get_current_target () method description.
	\begin{description}
		\item [+ \textcolor{darkgreen}{Spatial} \textcolor{blue}{get\_current\_target} ():] Renvoie la 
		référence de l'objet ou du noeud actuellement ciblé par le sytème de pistage automatique.\\
	\end{description}
	% Spatial get_preview_target () method description.
	\begin{description}
		\item [+ \textcolor{darkgreen}{Spatial} \textcolor{blue}{get\_preview\_target} ():] Renvoie la 
		référence de l'objet ou du noeud ayant été \\précédement ciblé par le sytème de pistage automatique.
		\\
	\end{description}
	% Spatial get_next_target () method description.
	\begin{description}
		\item [+ \textcolor{darkgreen}{Spatial} \textcolor{blue}{get\_next\_target} ():] Renvoie la
		référence du future objet ou noeud qui sera ciblé par le sytème de pistage automatique.\\
	\end{description}
	% Bool is_reset () method description.
	\begin{description}
		\item [+ \textcolor{red}{bool} \textcolor{blue}{is\_reset} ():] A t-on déjà rénitialisé la rotation 
		de la caméra ?\\
	\end{description}
	% Void shaken () method description.
	\begin{description}
		\item [+ \textcolor{red}{void} \textcolor{blue}{shaken} (amount, delay = 0.0):] Vibre la caméra.
		\begin{itemize}
			\item [>> \textbf{\textcolor{red}{float} amount}:] Contient la force de vibration de la caméra.
			\item [>> \textbf{\textcolor{red}{float} delay}:] Quel est le temps mort avant la vibration ?\\
		\end{itemize}
	\end{description}
	% Void generate_shaken () method description.
	\begin{description}
		\item [+ \textcolor{red}{void} \textcolor{blue}{generate\_shaken} (amount, delay = 0.0):] Génère 
		aléatoirement une vibration de la \\caméra. Notez que cette méthode affecte les configurations 
		éffectuées sur la caméra sur ses \\vibrations. Il se peut donc qu'après une appelle de cette 
		méthode, les valeurs des champs \textit{\hyperlink{shkaxis}{\\ShakeAxis}} et 
		\textit{\hyperlink{shkden}{ShakeDensity}} est changées.
		\begin{itemize}
			\item [>> \textbf{\textcolor{red}{float} amount}:] Contient la force de vibration de la caméra.
			\item [>> \textbf{\textcolor{red}{float} delay}:] Quel est le temps mort avant la vibration ?
		\end{itemize}
	\end{description}

	% CameraControl signals definition.
	\section{Les événements disponibles}
	% reset event description.
	\begin{description}
		\item [+ \textcolor{blue}{reset} (node):] Signal déclenché lorsque la caméra a été rénitialisée.
		\begin{itemize}
			\item [>> \textbf{\textcolor{darkgreen}{Node} node}:] Contient le noeud où ce signal a été émit.
			\\
		\end{itemize}
	\end{description}
	% target_changed event description.
	\begin{description}
		\item [+ \textcolor{blue}{target\_changed} (data):] Signal déclenché lorsque le système de pistage
		automatique de la \\caméra a changé de cible. Cet événement renvoie un dictionaire contenant les 
		clés suivantes:
		\begin{itemize}
			\item [>> \textbf{\textcolor{darkgreen}{Node} node}:] Contient le noeud où ce signal a été émit.
			\item [>> \textbf{\textcolor{darkgreen}{Node} target}:] Contient la référence de la nouvelle 
			cible de la caméra.\\
		\end{itemize}
	\end{description}
	% target_entered event description.
	\begin{description}
		\item [+ \textcolor{blue}{target\_entered} (data):] Signal déclenché lorsqu'un objet entre dans le 
		champ de vision du pisteur automatique de la caméra. Cet événement renvoie un dictionaire contenant 
		les clés suivantes:
		\begin{itemize}
			\item [>> \textbf{\textcolor{darkgreen}{Node} node}:] Contient le noeud où ce signal a été émit.
			\item [>> \textbf{\textcolor{darkgreen}{Node} target}:] Contient la référence de l'objet 
			détecté.\\
		\end{itemize}
	\end{description}
	% target_exited event description.
	\begin{description}
		\item [+ \textcolor{blue}{target\_exited} (data):] Signal déclenché lorsqu'un objet sort du champ de 
		vision du pisteur \\automatique de la caméra. Cet événement renvoie un dictionaire contenant les 
		clés suivantes:
		\begin{itemize}
			\item [>> \textbf{\textcolor{darkgreen}{Node} node}:] Contient le noeud où ce signal a été émit.
			\item [>> \textbf{\textcolor{darkgreen}{Node} target}:] Contient la référence de l'objet 
			détecté.\\
		\end{itemize}
	\end{description}
	% target_generated event description.
	\begin{description}
		\item [+ \textcolor{blue}{target\_generated} (data):] Signal déclenché lorsqu'une cible a été 
		générée par le pisteur \\automatique de la caméra. Cet événement renvoie un dictionaire contenant 
		les clés suivantes:
		\begin{itemize}
			\item [>> \textbf{\textcolor{darkgreen}{Node} node}:] Contient le noeud où ce signal a été émit.
			\item [>> \textbf{\textcolor{darkgreen}{Node} target}:] Contient la référence de la future cible
			du pisteur automatique de la caméra.\\
		\end{itemize}
	\end{description}
	% targeting_enabled event description.
	\begin{description}
		\item [+ \textcolor{blue}{targeting\_enabled} (node):] Signal déclenché lorsque le système de 
		pistage automatique de la caméra s'active.
		\begin{itemize}
			\item [>> \textbf{\textcolor{darkgreen}{Node} node}:] Contient le noeud où ce signal a été émit.
			\\
		\end{itemize}
	\end{description}
	% targeting_disabled event description.
	\begin{description}
		\item [+ \textcolor{blue}{targeting\_disabled} (node):] Signal déclenché lorsque le système de 
		pistage automatique de la caméra se désactive.
		\begin{itemize}
			\item [>> \textbf{\textcolor{darkgreen}{Node} node}:] Contient le noeud où ce signal a été émit.
			\\
		\end{itemize}
	\end{description}
	% min_zoom event description.
	\begin{description}
		\item [+ \textcolor{blue}{min\_zoom} (data):] Signal déclenché lorsqu'on atteint l'agrandissement 
		minimale de la caméra. Cet événement renvoie un dictionaire contenant les clés suivantes:
		\begin{itemize}
			\item [>> \textbf{\textcolor{darkgreen}{Node} node}:] Contient le noeud où ce signal a été émit.
			\item [>> \textbf{\textcolor{red}{float} depth}:] Contient le degré actuelle de
			l'agrandissement de la caméra.\\
		\end{itemize}
	\end{description}
	% max_zoom event description.
	\begin{description}
		\item [+ \textcolor{blue}{max\_zoom} (data):] Signal déclenché lorsqu'on atteint l'agrandissement 
		maximale de la caméra. Cet événement renvoie un dictionaire contenant les clés suivantes:
		\begin{itemize}
			\item [>> \textbf{\textcolor{darkgreen}{Node} node}:] Contient le noeud où ce signal a été émit.
			\item [>> \textbf{\textcolor{red}{float} depth}:] Contient le degré actuelle de
			l'agrandissement de la caméra.\\
		\end{itemize}
	\end{description}
	% zoom_started event description.
	\begin{description}
		\item [+ \textcolor{blue}{zoom\_started} (data):] Signal déclenché lorsqu'on démarre l'opération 
		d'agrandissement de la caméra. Cet événement renvoie un dictionaire contenant les clés suivantes:
		\begin{itemize}
			\item [>> \textbf{\textcolor{darkgreen}{Node} node}:] Contient le noeud où ce signal a été émit.
			\item [>> \textbf{\textcolor{red}{float} depth}:] Contient le degré actuelle de
			l'agrandissement de la caméra.\\
		\end{itemize}
	\end{description}
	% zoom_finished event description.
	\begin{description}
		\item [+ \textcolor{blue}{zoom\_finished} (data):] Signal déclenché lorsque l'opération 
		d'agrandissement de la caméra \\s'arrête. Cet événement renvoie un dictionaire contenant les clés 
		suivantes:
		\begin{itemize}
			\item [>> \textbf{\textcolor{darkgreen}{Node} node}:] Contient le noeud où ce signal a été émit.
			\item [>> \textbf{\textcolor{red}{float} depth}:] Contient le degré actuelle de
			l'agrandissement de la caméra.\\
		\end{itemize}
	\end{description}
	% zoom event description.
	\begin{description}
		\item [+ \textcolor{blue}{zoom} (data):] Signal déclenché au changement de la valeur 
		d'agrandissement de la caméra. Cet événement renvoie un dictionaire contenant les clés suivantes:
		\begin{itemize}
			\item [>> \textbf{\textcolor{darkgreen}{Node} node}:] Contient le noeud où ce signal a été émit.
			\item [>> \textbf{\textcolor{red}{float} depth}:] Contient le degré actuelle de
			l'agrandissement de la caméra.\\
		\end{itemize}
	\end{description}
	% generate_zoom event description.
	\begin{description}
		\item [+ \textcolor{blue}{generate\_zoom} (data):] Signal déclenché lorsqu'une valeur 
		d'agrandissement a été générée suite à un calcule éffectué en fonction de la vitesse de l'objet 
		pivot de la caméra. Cet événement renvoie un dictionaire contenant les clés suivantes:
		\begin{itemize}
			\item [>> \textbf{\textcolor{darkgreen}{Node} node}:] Contient le noeud où ce signal a été émit.
			\item [>> \textbf{\textcolor{red}{float} depth}:] Contient le future degré d'agrandissement de 
			la caméra.\\
		\end{itemize}
	\end{description}
	% max_distance event description.
	\newpage \begin{description}
		\item [+ \textcolor{blue}{max\_distance} (node):] Signal déclenché lorsque la distance de l'objet 
		pivot de la caméra est supérieur ou égale à la deuxième valeur du champ \textit{\hyperlink{gap}
		{Gap}}.
		\begin{itemize}
			\item [>> \textbf{\textcolor{darkgreen}{Node} node}:] Contient le noeud où ce signal a été émit.
			\\
		\end{itemize}
	\end{description}
	% max_angle event description.
	\begin{description}
		\item [+ \textcolor{blue}{max\_angle} (node):] Signal déclenché lorsque la distance de l'objet 
		pivot de la caméra est \\supérieur ou égale à la première valeur du champ \textit{\hyperlink{gap}
		{Gap}}.
		\begin{itemize}
			\item [>> \textbf{\textcolor{darkgreen}{Node} node}:] Contient le noeud où ce signal a été émit.
			\\
		\end{itemize}
	\end{description}
	% flat_angle event description.
	\begin{description}
		\item [+ \textcolor{blue}{flat\_angle} (data):] Signal déclenché lorsque l'angle de rotation de la 
		caméra est plat. Cet \\événement renvoie un dictionaire contenant les clés suivantes:
		\begin{itemize}
			\item [>> \textbf{\textcolor{darkgreen}{Node} node}:] Contient le noeud où ce signal a été émit.
			\item [>> \textbf{\textcolor{red}{float} angle}:] Contient l'angle de rotation de la caméra. 
			Notez que cet angle peut changé de signe en fonction du sens de rotation de la caméra.
		\end{itemize}
	\end{description}
\end{document}