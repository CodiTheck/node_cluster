% Document type and package imports.
\documentclass[a4paper, 11pt]{article}
\usepackage[utf8]{inputenc}
\usepackage[T1]{fontenc}
\usepackage[french]{babel}
\usepackage{charter}
\usepackage[top = 2cm, bottom = 2cm, left = 1cm, right = 1cm]{geometry}
\usepackage{setspace}
\usepackage{color}
\usepackage{xcolor}
\usepackage{hyperref}
\usepackage{tocloft}

% Preanblue.
\onehalfspacing
\definecolor{gray}{rgb}{0.4, 0.4, 0.4}
\definecolor{silver}{rgb}{0.95, 0.95, 0.95}
\renewcommand{\thesection}{\Roman{section} --}
\definecolor{darkgreen}{HTML}{1E8C15}
\cftsetindents{section}{1em}{2.5em}
\hypersetup {colorlinks=true, linkcolor=blue, urlcolor=blue, pdftitle={AnimationController module doc}}

% The start of the article.
\begin{document}
	% Change document background to silver color.
	\pagecolor{silver}
	% AnimationController module description.
	\huge{\hspace{10.7cm}\textit{\textbf{\textcolor{darkgreen}{AnimationControllerFx}}}}\large{} 
	\tableofcontents \newpage
	% AnimationController module definition.
	\section{Définition}
	\textcolor{darkgreen}{\textbf{AnimationControllerFx}} est un module permettant de contrôler de façon 
	dynamique les différentes animations que possède un objet. Son rôle principale est de permettre à
	l'utilisateur de pourvoir interagir dynamiquement entre le module et l'objet en question.\\
	\textcolor{red}{\textbf{NB}:} Ce module est compatible à un jeu 2D, 3D et n'est pas sauvegardable. Notez
	que ce module ne supporte pas le noeud
	\href{https://docs.godotengine.org/en/stable/classes/class_animationtreeplayer.html}
	{\textit{\textcolor{darkgreen}{AnimationTreePlayer}}}.

	% AnimationController properties definition.
	\section{Les propriétés disponibles}
	% Animator property.
	\textbf{+ \textcolor{darkgreen}{NodePath} Animator:} Contient la référence d'un animateur. Les types
	supportés sont: \href{https://docs.godotengine.org/en/stable/classes/class_animationtree.html}
	{\textit{\textcolor{darkgreen}{\\AnimationTree}}} et
	\href{https://docs.godotengine.org/en/stable/classes/class_animationplayer.html}
	{\textit{\textcolor{darkgreen}{AnimationPlayer}}}.\\\\
	% Target property.
	\textbf{+ \textcolor{darkgreen}{NodePath} Target:} Contient la référence du noeud sur lequel le module
	fera ses traitements. Le type que supporte cette propriété est le
	\href{https://docs.godotengine.org/en/stable/classes/class_kinematicbody.html}
	{\textit{\textcolor{darkgreen}{KinematicBody}}}.\\\\
	% RootMotion property.
	\textbf{+ \textcolor{red}{bool} RootMotion = \textcolor{red}{false}:} Voulez-vous déplacer le noeud 
	ciblé à partir des transformations \\récupérées sur le squelette d'animation ?\\\\
	% Gravity property.
	\textbf{+ \textcolor{red}{float} Gravity = \textcolor{blue}{9.8}:} Quel est l'effet de pésenteur qui 
	sera appliquer au noeud pris pour cible ?

	% AnimationController methods definition.
	\section{Les méthodes disponibles}
	% Float get_total_state_length () method description.
	\begin{description}
		\item [+ \textcolor{red}{float} \textcolor{blue}{get\_total\_state\_length} ():] Renvoie le temps 
		total des animations prises en charge par le développeur.\\
	\end{description}
	% Float get_max_state_length () method description.
	\begin{description}
		\item [+ \textcolor{red}{float} \textcolor{blue}{get\_max\_state\_length} ():] Renvoie le temps de 
		l'animation de plus grand parmit les \\animations chargées.\\
	\end{description}
	% Float get_min_state_length () method description.
	\begin{description}
		\item [+ \textcolor{red}{float} \textcolor{blue}{get\_min\_state\_length} ():] Renvoie le temps de 
		l'animation de plus petit parmit les \\animations chargées.\\
	\end{description}
	% PoolStringArray get_not_running_states_names () method description.
	\begin{description}
		\item [+ \textcolor{darkgreen}{PoolStringArray} \textcolor{blue}{get\_not\_running\_states\_names} 
		():] Renvoie les noms de toutes les \\animations qui ne sont pas en cours de lecture.\\
	\end{description}
	% Int get_progress_pourcentage () method description.
	\begin{description}
		\item [+ \textcolor{red}{int} \textcolor{blue}{get\_progress\_pourcentage} ():] Renvoie en 
		pourcentage, la progression en temps réel de \\l'animation en cours de lecture. L'ensemble des 
		valeurs qui seront renvoyées appartiendra à \\l'intervalle [\textcolor{blue}{0}; \textcolor{blue}
		{100}]\\
	\end{description}
	% Float get_normalized_progress () method description.
	\newpage \begin{description}
		\item [+ \textcolor{red}{float} \textcolor{blue}{get\_normalized\_progress} 
		():] Renvoie une valeur normalizée de la progression en temps réel de l'animation en cours de 
		lecture. L'ensemble des valeurs qui seront renvoyées appartiendra à l'intervalle [\textcolor{blue}
		{0.0}; \textcolor{blue}{1.0}].\\
	\end{description}
	% Void play_state () method description.
	\begin{description}
		\item [+ \textcolor{red}{void} \textcolor{blue}{play\_state} (name, queued, priority, count, speed, 
		blend, seek, delay = 0.0):] Joue une \\animation donnée en fonction des configurations prévues à son 
		égard. Notez que cette méthode n'agit que si l'animateur est de type
		\href{https://docs.godotengine.org/fr/stable/classes/class_animationplayer.html}
		{\textit{\textcolor{darkgreen}{AnimationPlayer}}}.
		\begin{itemize}
			\item [>> \textbf{\textcolor{darkgreen}{String} name}:] Contient le nom de l'animation à joué.
			\item [>> \textbf{\textcolor{red}{bool} queued}:] Voulez-vous mettre l'animation à jouée dans 
			une file d'attente ? Par défaut, cet paramètre est inactif.
			\item [>> \textbf{\textcolor{red}{int} count}:] Combien de fois l'animation sera jouer ? Par 
			défaut, l'animation sera jouée une seule fois seulement. Notez qu'une valeur négative engendrera
			une lecture infinie.
			\item [>> \textbf{\textcolor{red}{bool} priority}:] Met en priorité l'animation à jouée. La
			lecture d'une animation en priorité peut affecté ou non une animation déjà en cours de lecture.
			Par défaut, aucune priorité n'est \\attribuée à l'animation en question.
			\item [>> \textbf{\textcolor{red}{float} speed}:] Contient la vitesse de lecture de l'animation 
			? Par défaut, la valeur de ce paramètre est sur \textcolor{gray}{\textit{1.0}}.
			\item [>> \textbf{\textcolor{red}{float} blend}:] Voulez-vous mettre une transition entre 
			l'animation actuellement en cours de \\lecture et la nouvelle animation à lire ? Notez que cet 
			paramètre peut faire office de transition simple lorsque la lecture de l'animation est faite en 
			utilisant le paramètre \textcolor{gray}{\textit{seek}}. Par défaut, la valeur de ce paramètre 
			est sur \textcolor{gray}{\textit{-1.0}}.
			\item [>> \textbf{\textcolor{red}{float} seek}:] Souhatez-vous, vous déplacer dans l'animation à
			travers la position du lecteur ? Par défaut, la valeur de ce paramètre est sur \textcolor{gray}
			{\textit{-1.0}}.
			\item [>> \textbf{\textcolor{red}{float} delay}:] Quel est le temps mort avant le lancement de 
			l'animation ?\\
		\end{itemize}
	\end{description}
	% Void queue_deletion () method description.
	\begin{description}
		\item [+ \textcolor{red}{void} \textcolor{blue}{queue\_deletion} (id, update = true, delay = 0.0):] 
		Supprime un ou plusieurs file(s) \\d'attente(s) de la mémoire. Notez que cette méthode n'agit que si 
		l'animateur est de type
		\href{https://docs.godotengine.org/en/stable/classes/class_animationplayer.html}
		{\textit{\textcolor{darkgreen}{\\AnimationPlayer}}}.
		\begin{itemize}
			\item [>> \textbf{\textcolor{darkgreen}{String | PoolStringArray} id}:] Contient le(s) nom(s) 
			de(s) animation(s) en file d'éttente.
			\item [>> \textbf{\textcolor{red}{bool} update}:] Voulez-vous que les modifications affectent 
			l'animation en cours de lecture ? 
			\item [>> \textbf{\textcolor{red}{float} delay}:] Quel est le temps mort avant le lancement de 
			l'animation ?\\
		\end{itemize}
	\end{description}
	% Void clear_queue () method description.
	\begin{description}
		\item [+ \textcolor{red}{void} \textcolor{blue}{clear\_queue} (update = true, delay = 0.0):] 
		Supprime toutes les files d'attentes de la \\mémoire. Notez que cette méthode n'agit que si 
		l'animateur est de type
		\href{https://docs.godotengine.org/en/stable/classes/class_animationplayer.html}
		{\textit{\textcolor{darkgreen}{AnimationPlayer}}}.
		\begin{itemize}
			\item [>> \textbf{\textcolor{red}{bool} update}:] Voulez-vous que les modifications affectent 
			l'animation en cours de lecture ? 
			\item [>> \textbf{\textcolor{red}{float} delay}:] Quel est le temps mort avant le lancement de 
			l'animation ?\\
		\end{itemize}
	\end{description}
	% Void run_state () method description.
	\newpage \begin{description}
		\item [+ \textcolor{red}{void} \textcolor{blue}{run\_state} (source, value, reversed, fade, time, 
		type, easing, delay = 0.0):] Modifie la \\valeur d'une propriété à partir de la source donnée. Cette 
		méthode n'agit que si l'animateur est de type
		\href{https://docs.godotengine.org/en/stable/classes/class_animationplayer.html}
		{\textit{\textcolor{darkgreen}{AnimationPlayer}}}.
		\begin{itemize}
			\item [>> \textbf{\textcolor{darkgreen}{String} source}:] Quel est le chemin pointant vers la
			propriété à changée. Notez que vous n'avez plus bésoin de mettre le préfix \textcolor{gray}
			{\textit{"parameters/"}} avant de préciser l'élément que vous voulez ciblé. Exemple: 
			\textcolor{gray}{\textit{"TimeScale/scale"}} au lieu de \textcolor{gray}{\textit{"parameters/
			TimeScale/scale"}}.
			\item [>> \textbf{\textcolor{darkgreen}{Variant} value}:] Quelle est la nouvelle valeur de la
			propriété pris pour cible ?
			\item [>> \textbf{\textcolor{red}{bool} reversed}:] Voulez-vous modifier les valeurs dans 
			l'ordre inverse où elles sont venue ? Par défaut, la modification des valeurs s'éffectue de 
			façon normale.
			\item [>> \textbf{\textcolor{red}{bool} fade}:] Aimeriez-vous effectuer une transition lorsque 
			la valeur de la propriété ciblé \\changera ? Par défaut, aucune transition n'est appliquée.
			\item [>> \textbf{\textcolor{red}{float} time}:] La transition de valeur s'effectuera en combien 
			de temps ? Par défaut, la valeur de ce paramètre est sur \textcolor{gray}{\textit{0.0}}.
			\item [>> \textbf{\textcolor{red}{int} type}:] Quel type de transition souhaitez-vous utiliser ? 
			Les constantes issues de ce paramètre sont celles définient aux sein de Godot. Par défaut, la 
			valeur de ce paramètre est sur \textcolor{gray}{\textit{0}}.
			\item [>> \textbf{\textcolor{red}{int} easing}:] Quel assouplissement désirez-vous utiliser pour 
			faire la transition. Les constantes issues de ce paramètre sont celles définient aux sein de 
			Godot. Par défaut, la valeur de ce paramètre est sur \textcolor{gray}{\textit{2}}.
			\item [>> \textbf{\textcolor{red}{float} delay}:] Quel est le temps mort avant le changement de 
			valeur ?\\
		\end{itemize}
	\end{description}
	% Void combo () method description.
	\begin{description}
		\item [+ \textcolor{red}{void} \textcolor{blue}{combo} (states, ref, start, level, source = '', 
		blend = -1.0, delay = 0.0):] Gère les différents appuies répétés conçut pour déclencher des actions 
		ou états.
		\begin{itemize}
			\item [>> \textbf{\textcolor{darkgreen}{String | PoolStringArray} states}:] Quels sont/est le(s) 
			état(s) qui sera/seront exécuté(s) au fure et à mesure que le joueur progressera dans les 
			appuies répétés ?
			\item [>> \textbf{\textcolor{darkgreen}{String | PoolStringArray} ref}:] De quelle(s) 
			animation(s), les appuies répétés devront \\commencés ?
			\item [>> \textbf{\textcolor{darkgreen}{String} start}:] Sur quelle animation revenir lorsque 
			tous les états auront été exécutés ou lorsque l'animation en cours d'exécution ne correspond à 
			l'état de référence ?
			\item [>> \textbf{\textcolor{red}{int} level}:] Quel est le niveau d'appuie de la touche prise
			dans une série d'appuies répétés ?
			\item [>> \textbf{\textcolor{darkgreen}{String} source}:] Quel est le chemin pointant vers la
			propriété à changée. Notez que vous n'avez plus bésoin de mettre le préfix \textcolor{gray}
			{\textit{"parameters/"}} avant de préciser l'élément que vous voulez ciblé. Exemple: 
			\textcolor{gray}{\textit{"TimeScale/scale"}} au lieu de \textcolor{gray}{\textit{"parameters/
			TimeScale/scale"}}. Le noeud doit être \\uniquement de type
			\href{https://docs.godotengine.org/fr/stable/classes/class_animationnodetransition.html}
			{\textit{\textcolor{darkgreen}{AnimationNodeTransition}}}. C'est obligatoire et important de le 
			faire ainsi. La valeur de ce paramètre est inutile lorsque l'animateur est de type
			\href{https://docs.godotengine.org/fr/stable/classes/class_animationplayer.html}
			{\textit{\textcolor{darkgreen}{AnimationPlayer}}}.
			\item [>> \textbf{\textcolor{red}{float} blend}:] Voulez-vous mettre une transition entre 
			l'animation actuellement en cours de \\lecture et la nouvelle animation à lire ? Cette option
			est à utilisée uniquement lorsque \\l'animateur est de type
			\href{https://docs.godotengine.org/fr/stable/classes/class_animationplayer.html}
			{\textit{\textcolor{darkgreen}{AnimationPlayer}}}.
			\item [>> \textbf{\textcolor{red}{float} delay}:] Quel est le temps mort avant l'exécution des 
			états prévu ?\\
		\end{itemize}
	\end{description}
	% Dictionary get_queue_data () method description.
	\begin{description}
		\item [+ \textcolor{darkgreen}{Dictionary} \textcolor{blue}{get\_queue\_data} (json = true):]
		Renvoie les données de toutes les animations misent dans une file d'attente.
		\begin{itemize}
			\item [>> \textbf{\textcolor{red}{bool} json}:] Voulez-vous renvoyer le résultat sous le format 
			json ?
		\end{itemize}
	\end{description}

	% AnimationController signals definition.
	\newpage \section{Les événements disponibles}
	% queue_changed event description.
	\begin{description}
		\item [+ \textcolor{blue}{queue\_changed} (node):] Signal déclenché lorsque des modifications sont
		éffectuées dans la file d'attente.
		\begin{itemize}
			\item [>> \textbf{\textcolor{darkgreen}{Node} node}:] Contient le noeud où cet signal a été 
			émit.\\
		\end{itemize}
	\end{description}
	% state_started event description.
	\begin{description}
		\item [+ \textcolor{blue}{state\_started} (data):] Signal déclenché au démarrage de la lecture d'un 
		état. Cet événement \\renvoie un dictionaire contenant les clés suivantes:
		\begin{itemize}
			\item [>> \textbf{\textcolor{darkgreen}{Node} node}:] Contient le noeud où cet signal a été 
			émit.
			\item [>> \textbf{\textcolor{darkgreen}{String} name}:] Contient le nom de l'animation en 
			question.\\
		\end{itemize}
	\end{description}
	% state_finished event description.
	\begin{description}
		\item [+ \textcolor{blue}{state\_finished} (data):] Signal déclenché à la fin de la lecture d'un 
		état. Cet événement renvoie un dictionaire contenant les clés suivantes:
		\begin{itemize}
			\item [>> \textbf{\textcolor{darkgreen}{Node} node}:] Contient le noeud où cet signal a été 
			émit.
			\item [>> \textbf{\textcolor{darkgreen}{String} name}:] Contient le nom de l'animation en 
			question.\\
		\end{itemize}
	\end{description}
	% state_changed event description.
	\begin{description}
		\item [+ \textcolor{blue}{state\_changed} (data):] Signal déclenché lorsqu'un état a été changé par 
		un autre. Cet événement renvoie un dictionaire contenant les clés suivantes:
		\begin{itemize}
			\item [>> \textbf{\textcolor{darkgreen}{Node} node}:] Contient le noeud où cet signal a été 
			émit.
			\item [>> \textbf{\textcolor{darkgreen}{String} old}:] Contient le nom de l'ancienne l'animation 
			.
			\item [>> \textbf{\textcolor{darkgreen}{String} new}:] Contient le nom de la nouvelle animation 
			en question.\\
		\end{itemize}
	\end{description}
\end{document}