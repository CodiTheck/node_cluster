% Document type and package imports.
\documentclass[a4paper, 11pt]{article}
\usepackage[utf8]{inputenc}
\usepackage[T1]{fontenc}
\usepackage[french]{babel}
\usepackage{hyperref}
\usepackage{tocloft}
\usepackage{charter}
\usepackage[top = 2cm, bottom = 2cm, left = 1cm, right = 1cm]{geometry}
\usepackage{setspace}
\usepackage{color}
\usepackage{xcolor}

% Preanblue.
\onehalfspacing
\definecolor{gray}{rgb}{0.4, 0.4, 0.4}
\definecolor{silver}{rgb}{0.95, 0.95, 0.95}
\renewcommand{\thesection}{\Roman{section} --}
\cftsetindents{section}{1em}{2.5em}
\definecolor{darkgreen}{HTML}{1E8C15}
\hypersetup {colorlinks=true, linkcolor=blue, urlcolor=blue, pdftitle={InverseKinematic module doc}}

% The start of the article.
\begin{document}
	% Change document background to silver color.
	\pagecolor{silver}
	% InverseKinematic module description.
	\huge{\hspace{12cm}\textit{\textbf{\textcolor{darkgreen}{InverseKinematicFx}}}}\large{} \tableofcontents 
	\newpage
	% InverseKinematic module definition.
	\section{Définition}
	\textcolor{darkgreen}{\textbf{InverseKinematicFx}} est un module conçut pour le pistage d'un objet à 
	partir d'un squellette \\d'animation. Ce module oblige la section du squellette sélectionnée pour cibler
	un objet.\\
	\textcolor{red}{\textbf{NB}:} Ce module est compatible à un jeu 3D et n'est pas sauvegardable. Il 
	possède également un pisteur permettant de pister automatiquement ces cibles en fonction de son champ de 
	vision comme pour le module \textit{\textcolor{darkgreen}{CameraControlFx}}.

	% InverseKinematic properties definition.
	\section{Les propriétés disponibles}
	% TargetSkeleton property.
	\textbf{+ \textcolor{darkgreen}{NodePath} TargetSkeleton:} Contient l'instance d'un noeud de type
	\href{https://docs.godotengine.org/fr/stable/classes/class_skeleton.html}{\textit{\textcolor{darkgreen}
	{Skeleton}}}.\\\\
	% BoneName property.
	\textbf{+ \textcolor{darkgreen}{String} BoneName:} Quelle section du squellette d'animation voulez-vous 
	ciblée ?\\\\
	% LookAtAxis property.
	\textbf{+ \textcolor{red}{int} LookAtAxis = \textcolor{blue}{2}:} Sur quelle direction les pistages se 
	feront ? Les valeurs possibles sont:
	\begin{itemize}
		\item [-> \textbf{\textcolor{gray}{MegaAssets.Axis.X} ou \textcolor{blue}{1}}:] L'axe des absisses.
		\item [-> \textbf{\textcolor{gray}{MegaAssets.Axis.Y} ou \textcolor{blue}{2}}:] L'axe des ordonnés.
		\item [-> \textbf{\textcolor{gray}{MegaAssets.Axis.Z} ou \textcolor{blue}{3}}:] L'axe des côtes.
		\item [-> \textbf{\textcolor{gray}{MegaAssets.Axis.\_X} ou \textcolor{blue}{4}}:] L'opposé de l'axe 
		des absisses.
		\item [-> \textbf{\textcolor{gray}{MegaAssets.Axis.\_Y} ou \textcolor{blue}{5}}:] L'opposé de l'axe 
		des ordonnés.
		\item [-> \textbf{\textcolor{gray}{MegaAssets.Axis.\_Z} ou \textcolor{blue}{6}}:] L'opposé de l'axe 
		des côtes.\\
	\end{itemize}
	% FroozeAxis property.
	\textbf{+ \textcolor{red}{int} FroozeAxis = \textcolor{blue}{0}:} Quel axe bloqué au cours des pistages 
	? Les valeurs possibles sont:
	\begin{itemize}
		\item [-> \textbf{\textcolor{gray}{MegaAssets.Axis.NONE} ou \textcolor{blue}{0}}:] Aucun blockage.
		\item [-> \textbf{\textcolor{gray}{MegaAssets.Axis.X} ou \textcolor{blue}{1}}:] Blockage de l'axe 
		des absisses.
		\item [-> \textbf{\textcolor{gray}{MegaAssets.Axis.Y} ou \textcolor{blue}{2}}:] Blockage de l'axe 
		des ordonnés.
		\item [-> \textbf{\textcolor{gray}{MegaAssets.Axis.Z} ou \textcolor{blue}{3}}:] Bockage de l'axe des 
		côtes.
		\item [-> \textbf{\textcolor{gray}{MegaAssets.Axis.\_X} ou \textcolor{blue}{4}}:] Blockage de 
		l'opposé de l'axe des absisses.
		\item [-> \textbf{\textcolor{gray}{MegaAssets.Axis.\_Y} ou \textcolor{blue}{5}}:] Blockage de 
		l'opposé de l'axe des ordonnés.
		\item [-> \textbf{\textcolor{gray}{MegaAssets.Axis.\_Z} ou \textcolor{blue}{6}}:] Blockage de 
		l'opposé de l'axe des côtes.
		\item [-> \textbf{\textcolor{gray}{MegaAssets.Axis.XY} ou \textcolor{blue}{7}}:] Bockage des axes x 
		et y.
		\item [-> \textbf{\textcolor{gray}{MegaAssets.Axis.XZ} ou \textcolor{blue}{8}}:] Bockage des axes x 
		et z.
		\item [-> \textbf{\textcolor{gray}{MegaAssets.Axis.YZ} ou \textcolor{blue}{9}}:] Bockage des axes y 
		et z.
		\item [-> \textbf{\textcolor{gray}{MegaAssets.Axis.\_XY} ou \textcolor{blue}{10}}:] Bockage de 
		l'opposé des axes x et y.
		\item [-> \textbf{\textcolor{gray}{MegaAssets.Axis.\_XZ} ou \textcolor{blue}{11}}:] Bockage de 
		l'opposé des axes x et z.
		\item [-> \textbf{\textcolor{gray}{MegaAssets.Axis.\_YZ} ou \textcolor{blue}{12}}:] Bockage de 
		l'opposé des axes y et z.		
		\item [-> \textbf{\textcolor{gray}{MegaAssets.Axis.XYZ} ou \textcolor{blue}{13}}:] Bockage des axes 
		x, y et z.
		\item [-> \textbf{\textcolor{gray}{MegaAssets.Axis.\_XYZ} ou \textcolor{blue}{14}}:] Bockage de 
		l'opposé des axes x, y et z.\\
	\end{itemize}
	% Offset property.
	\textbf{+ \textcolor{darkgreen}{Vector3} Offset = \textcolor{darkgreen}{Vector3} (\textcolor{blue}{0.0}, 
	\textcolor{blue}{0.0}, \textcolor{blue}{0.0}):} Contrôle l'ajustement de la section du squellette 
	\\d'animation en terme de rotation.\\\\
	% ListenCamera property.
	\newpage \textbf{+ \textcolor{darkgreen}{NodePath} ListenCamera:} Contient l'instance d'un noeud de type 
	\textit{\textcolor{darkgreen}{CameraControlFx}}. Si l'on \\statisfait l'entrée qu'attend cette 
	propriété, le module se synchrônisera aux données renvoyées par le module \textit{\textcolor{darkgreen}
	{CameraControlFx}} uniquement lorsque ce dernier activera son système de pistage automatique.\\\\
	% TargetingAera property.
	\textbf{+ \textcolor{darkgreen}{NodePath} TargetingAera:} Contient l'instance d'un noeud de type
	\href{https://docs.godotengine.org/fr/stable/classes/class_area.html}{\textit{\textcolor{darkgreen}
	{Area}}} représentant le champ \\d'activité du module. Si l'entrée de cette propriété n'est pas 
	satisfaite, le module cherchera toutes les \\références des objets ou des noeuds donnés par le 
	développeur à partir des configurations éffectuées dans le champ \textit{\textcolor{gray}{Targets}}.\\\\
	% Targets property.
	\textbf{+ \textcolor{darkgreen}{Array} Targets:} Tableau de dictionnaires contenant toutes les 
	différentes configurations sur chaque objet prise en charge par le développeur. Les dictionnaires issus 
	de ce tableau supportent les clés \\suivantes:
	\begin{itemize}
		\item[>> \textbf{\textcolor{darkgreen}{String} id}:] Quel est l'identifiant du noeud à prendre en 
		charge ? L'utilisation de cette clé est \\obligatoire.\\
		\item[>> \textbf{\textcolor{red}{int} search = \textcolor{blue}{3}}:] Quel moyen utilisé pour 
		chercher le noeud à prendre en charge ? Notez que \\l'identifiant donné est pisté à par un programme 
		de recherche. Les valeurs possibles sont:
		\begin{itemize}
			\item [-> \textbf{\textcolor{gray}{MegaAssets.NodeProperty.NAME} ou \textcolor{blue}{0}}:] 
				Trouve un noeud en utilisant son nom.
				\item [-> \textbf{\textcolor{gray}{MegaAssets.NodeProperty.GROUP} ou \textcolor{blue}{1}}:] 
				Trouve un noeud en utilisant le nom de son groupe.
				\item [-> \textbf{\textcolor{gray}{MegaAssets.NodeProerty.TYPE} ou \textcolor{blue}{2}}:] 
				Trouve un noeud en utilisant le nom de sa classe.
				\item [-> \textbf{\textcolor{gray}{MegaAssets.NodeProerty.ANY} ou \textcolor{blue}{3}}:] 
				Trouve un noeud en utilisant l'un des trois moyens cités plus haut.\\
		\end{itemize}
		\item[>> \textbf{\textcolor{red}{bool} ignored = \textcolor{red}{false}}:] Le pisteur de la caméra 
		doit-il ignoré l'identifiant précisé ?\\
		\item[>> \textbf{\textcolor{red}{float} transition = \textcolor{blue}{1.0}}:] Combien de temps prend 
		le passage d'une cible à une autre ?\\
		\item[>> \textbf{\textcolor{red}{int} type = \textcolor{blue}{0}}:] Quel type de transition adopté ?
		Les valeurs possibles de ce champ sont celles de Godot. Cette propriété est solicité au changement 
		de cible.\\
		\item[>> \textbf{\textcolor{red}{int} easing = \textcolor{blue}{2}}:] Quel assouplissement adopté ? 
		Les valeurs possibles de ce champ sont celles de Godot. Cette propriété est solicité au changement 
		de cible.\\
		\item[>> \textbf{\textcolor{darkgreen}{Array | Dictionary} entered}:] Signal déclenché lorsque
		l'objet entre dans le champ de vision du pisteur. Cette clé exécute les différentes actions données
		à son déclenchement. Pour soumettre les actions à exécutées référez vous à la méthode utilisée au
		niveau de la clé \textit{\textcolor{gray}{actions}} de la propriété \textit{\textcolor{gray}
		{EventsBindings}} dans les bases du framework.\\
		\item[>> \textbf{\textcolor{darkgreen}{Array | Dictionary} exited}:] Signal déclenché lorsqu'un
		objet sort du champ de vision du pisteur. Cette clé exécute les différentes actions données à son
		déclenchement. Pour soumettre les \\actions à exécutées référez vous à la méthode utilisée au niveau
		de la clé \textit{\textcolor{gray}{actions}} de la propriété \textit{\textcolor{gray}
		{EventsBindings}} dans les bases du framework.
	\end{itemize}

	% InverseKinematic methods definition.
	\section{Les méthodes disponibles}
	% Void change_target () method description.
	\begin{description}
		\item [+ \textcolor{red}{void} \textcolor{blue}{change\_target} (index = -1, delay = 0.0):] Force la 
		section du squellette d'animation à changé de cible parmit celles détectées. Par défaut, une cible 
		est générée si l'index de la nouvelle cible n'a pas été donné.
		\begin{itemize}
			\item [>> \textbf{\textcolor{red}{int} index}:] Contient l'index de la nouvelle cible à pistée.
			\item [>> \textbf{\textcolor{red}{float} delay}:] Quel est le temps mort avant le changement ?\\
		\end{itemize}
	\end{description}
	% Dictionary get_targets_data () method description.
	\begin{description}
		\item [+ \textcolor{darkgreen}{Dictionary} \textcolor{blue}{get\_targets\_data} (json = false):] 
		Renvoie toutes les données concernant les cibles de l'inverseur kinénatique.
		\begin{itemize}
			\item [>> \textbf{\textcolor{red}{bool} json}:] Voulez-vous renvoyer les données au format json 
			?\\
		\end{itemize}
	\end{description}
	% Node get_current_target () method description.
	\begin{description}
		\item [+ \textcolor{darkgreen}{Node} \textcolor{blue}{get\_current\_target} ():] Renvoie la 
		référence de l'objet ou du noeud actuellement ciblé par le sytème de pistage automatique.\\
	\end{description}
	% Node get_preview_target () method description.
	\begin{description}
		\item [+ \textcolor{darkgreen}{Node} \textcolor{blue}{get\_preview\_target} ():] Renvoie la 
		référence de l'objet ou du noeud ayant été précédement ciblé par le sytème de pistage automatique.\\
	\end{description}
	% Node get_next_target () method description.
	\begin{description}
		\item [+ \textcolor{darkgreen}{Node} \textcolor{blue}{get\_next\_target} ():] Renvoie la référence 
		du future objet ou noeud qui sera ciblé par le sytème de pistage automatique.\\
	\end{description}

	% CameraControl signals definition.
	\section{Les événements disponibles}
	% target_changed event description.
	\begin{description}
		\item [+ \textcolor{blue}{target\_changed} (data):] Signal déclenché lorsque le pisteur a changé de 
		cible. Cet événement renvoie un dictionaire contenant les clés suivantes:
		\begin{itemize}
			\item [>> \textbf{\textcolor{darkgreen}{Node} node}:] Contient le noeud où ce signal a été émit.
			\item [>> \textbf{\textcolor{darkgreen}{Node} target}:] Contient la référence de la nouvelle 
			cible du pisteur.\\
		\end{itemize}
	\end{description}
	% target_entered event description.
	\begin{description}
		\item [+ \textcolor{blue}{target\_entered} (data):] Signal déclenché lorsqu'un objet entre dans le 
		champ de vision du \\pisteur. Cet événement renvoie un dictionaire contenant les clés suivantes:
		\begin{itemize}
			\item [>> \textbf{\textcolor{darkgreen}{Node} node}:] Contient le noeud où ce signal a été émit.
			\item [>> \textbf{\textcolor{darkgreen}{Node} target}:] Contient la référence de l'objet 
			détecté.\\
		\end{itemize}
	\end{description}
	% target_exited event description.
	\begin{description}
		\item [+ \textcolor{blue}{target\_exited} (data):] Signal déclenché lorsqu'un objet sort du champ de 
		vision du pisteur. Cet événement renvoie un dictionaire contenant les clés suivantes:
		\begin{itemize}
			\item [>> \textbf{\textcolor{darkgreen}{Node} node}:] Contient le noeud où ce signal a été émit.
			\item [>> \textbf{\textcolor{darkgreen}{Node} target}:] Contient la référence de l'objet 
			détecté.\\
		\end{itemize}
	\end{description}
	% target_generated event description.
	\newpage \begin{description}
		\item [+ \textcolor{blue}{target\_generated} (data):] Signal déclenché lorsqu'une cible a été 
		générée par le pisteur. Cet \\événement renvoie un dictionaire contenant les clés suivantes:
		\begin{itemize}
			\item [>> \textbf{\textcolor{darkgreen}{Node} node}:] Contient le noeud où ce signal a été émit.
			\item [>> \textbf{\textcolor{darkgreen}{Node} target}:] Contient la référence de la future cible
			du pisteur.\\
		\end{itemize}
	\end{description}
\end{document}