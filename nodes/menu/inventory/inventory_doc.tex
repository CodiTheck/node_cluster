% Document type and package imports.
\documentclass[a4paper, 11pt]{article}
\usepackage[utf8]{inputenc}
\usepackage[T1]{fontenc}
\usepackage[french]{babel}
\usepackage{charter}
\usepackage[top = 2cm, bottom = 2cm, left = 1cm, right = 1cm]{geometry}
\usepackage{setspace}
\usepackage{color}
\usepackage{xcolor}
\usepackage{hyperref}
\usepackage{tocloft}

% Preanblue.
\onehalfspacing
\definecolor{gray}{rgb}{0.4, 0.4, 0.4}
\definecolor{silver}{rgb}{0.95, 0.95, 0.95}
\renewcommand{\thesection}{\Roman{section} --}
\definecolor{darkgreen}{HTML}{1E8C15}
\cftsetindents{section}{1em}{2.5em}
\hypersetup {colorlinks=true, linkcolor=blue, urlcolor=blue, pdftitle={Inventory module doc}}

% The start of the article.
\begin{document}
	% Change document background to silver color.
	\pagecolor{silver}
	% Inventory module description.
	\huge{\hspace{14.5cm}\textit{\textbf{\textcolor{darkgreen}{InventoryFx}}}}\large{} \tableofcontents
	\newpage
	% Inventory module definition.
	\section{Définition}
	\textcolor{darkgreen}{\textbf{InventoryFx}} est un module conçut pour la gestion des inventaires dans un 
	jeu vidéo. Le module intègre un support brute ainsi qu'un support graphique permettant de gérer en temps 
	réel l'évolution de l'inventaire.\\
	\textcolor{red}{\textbf{NB}:} Ce module est compatible à un jeu 2D, 3D et est sauvegardable.

	% Inventory properties definition.
	\section{Les propriétés disponibles}
	% Mode property.
	\textbf{+ \textcolor{red}{int} Mode = \textcolor{blue}{0}:} Quel mode de l'inventaire voulez-vous 
	utiliser ? Les valeurs possibles sont:
	\begin{itemize}
		\item[-> \textbf{\textcolor{gray}{InventoryFx.Model.HYBRID} ou \textcolor{blue}{0}}:] Mode sans
		interface graphique (hybride ou brute).
		\item[-> \textbf{\textcolor{gray}{InventoryFx.Model.UI} ou \textcolor{blue}{1}}:] Mode avec
		interface graphique.\\
	\end{itemize}
	% ListenModule property.
	\textbf{+ \textcolor{darkgreen}{NodePath} ListenModule:} Contient l'instance d'un module du même trampe
	de ce module. \\Exemple: \textit{\textcolor{darkgreen}{AnimatorFx, MenuFx, AudioTrackFx}} etc... En
	d'autres termes, contient la référence d'un \\module possèdant également un curseur. Le but de cette
	fonctionnalité est de syncnrôniser la valeur du curseur du module référé à celle de ce module.\\\\
	% PocketCount property.
	\textbf{+ \textcolor{red}{int} PocketCount = \textcolor{blue}{0}:} Contient le nombre total de place à 
	allouée par l'inventaire en mémoire. Cette propriété ne prend son utilitée que si l'inventaire passe 
	en mode hybride. La valeur de cette \\propriété est dans l'intervalle [\textcolor{blue}{0}; 
	\textcolor{blue}{++}].\\\\
	% EmptySymbol property.
	\textbf{+ \textcolor{darkgreen}{Texture} EmptySymbol:} Quelle texture devrons nous utilisée si une poche 
	de l'inventaire se retrouve vidée ? Cette propriété ne prend son utilitée que si l'inventaire passe en 
	mode interface utilisateur.\\\\
	% AllowDragAndDrop property.
	\textbf{+ \textcolor{red}{bool} AllowDragAndDrop = \textcolor{red}{false}:} Voulez-vous permettre les 
	opérations du glisser-déposser sur l'inventaire ? Cette propriété ne prend son utilitée que si 
	l'inventaire passe en mode interface utilisateur.\\\\
	% DragDropStyle property.
	\textbf{+ \textcolor{darkgreen}{NodePath} DragDropStyle:} Contient l'instance d'un noeud de type
	\href{https://docs.godotengine.org/en/stable/classes/class_sprite.html}
	{\textit{\textcolor{darkgreen}{Sprite}}} ou
	\href{https://docs.godotengine.org/en/stable/classes/class_texturerect.html}
	{\textit{\textcolor{darkgreen}{TextureRect}}}. Ce champ utilise les configurations éffectuées sur le 
	noeud récupéré pour mettre à jour le style qu'utilisera le \\module pour éffectuer l'opération. Cette 
	propriété ne prend son utilitée que si l'inventaire passe en mode interface utilisateur.\\\\
	% Varieties property.
	\textbf{+ \textcolor{darkgreen}{Array} \hypertarget{varieties}{Varieties}}: Tableau de dictionnaires 
	contenant les différentes configurations de chaque variété d'objet prise en charge par le développeur. 
	Les clés que supportent les dictionnaires sont:\\
	\begin{itemize}
		\item[>> \textbf{\textcolor{darkgreen}{String} name}:] Contient le nom de la variété en question.
		L'utilisation de cette clé est obligatoire.\\
		\item[>> \textbf{\textcolor{darkgreen}{String} symbol}:] Contient le chemin pointant vers l'image 
		représentant la variété. L'utilisation de cette clé est obligatoire. Notez que cette clé ne 
		s'utilise que lorsque l'inventaire est en mode \\interface graphique.\\
		\newpage \item[>> \textbf{\textcolor{darkgreen}{Vector2} size = \textcolor{darkgreen}{Vector2}
		(\textcolor{blue}{50}, \textcolor{blue}{50})}:] Contrôle la résolution du symbol de la variété. 
		Notez que cette clé ne s'utilise que lorsque l'inventaire est en mode interface graphique.\\
		\item[>> \textbf{\textcolor{red}{int} quality = \textcolor{blue}{2}}:] Contrôle la qualité du symbol 
		de la variété. Notez que cette clé ne s'utilise que lorsque l'inventaire est en mode interface 
		graphique. Les valeurs possibles sont celles définient au sein de classe
		\href{https://docs.godotengine.org/en/stable/classes/class_image.html}
		{\textit{\textcolor{darkgreen}{Image}}} de Godot.\\
		\item[>> \textbf{\textcolor{red}{bool} uncastable = \textcolor{red}{false}}:] Devons nous rendre 
		injetable cette variété ?\\
		\item[>> \textbf{\textcolor{red}{bool} maskstock = \textcolor{red}{false}}:] Devons cacher le stock 
		à affiché dans l'inventaire graphique ? Notez que cette clé ne s'utilise que lorsque l'inventaire 
		est en mode interface graphique.\\
		\item[>> \textbf{\textcolor{darkgreen}{PoolStringArray} \hypertarget{combination}{combination}}:] 
		Quelles sont les noms des variétés qu'il faut combiner pour \\obtenir cette variété ? La taille de 
		cette tableau ne doit pas dépasser deux éléments. Notez que les variétés que vous donnerai doit déjà 
		être définit dans l'inventaire.\\
	 	\item[>> \textbf{\textcolor{darkgreen}{Array} levels}:] Contient les différentes valeurs que 
	 	doit vérifiée la valeur contenu dans la clé \textit{\textcolor{gray}{stock}}. Les dictionnaires 
	 	issus de ce tableau suppotent les clés suivantes:
	 	\begin{itemize}
	 		\item[>> \textbf{\textcolor{red}{int} value}:] Contient une des valeurs de la clé 
	 		\textit{\textcolor{gray}{stock}}.
	 		\item[>> \textbf{\textcolor{darkgreen}{Array} actions}]: Que se passera t-il lorsque la valeur 
	 		du stock de cette variété atteint celle \\indiquée ? Cette clé contient tous les configurations 
	 		relatives à un ou plusieurs flux \\d'exécution(s). L'utilisation de cette clé est déjà décrite 
	 		au niveau des bases du framework. Précisement le sujet portant sur l'utilisation de la propriété 
			\textit{\textcolor{gray}{EventsBindings}} (la section des \\actions d'un événement).\\
	 	\end{itemize}
		\item[>> \textbf{\textcolor{red}{int} stock = \textcolor{blue}{0}}:] Quel est le stock maximale de 
		cette variété ?\\
	\end{itemize}
	% Pockets property.
	\textbf{+ \textcolor{darkgreen}{Array} Pockets}: Tableau de dictionnaires contenant les différentes 
	configurations de chaque poche disponible sur l'inventaire. Cette propriété ne prend son utilitée que si 
	l'inventaire passe en mode \\interface utilisateur. Les clés que supportent les dictionnaires sont:\\
	\begin{itemize}
		\item[>> \textbf{\textcolor{darkgreen}{String | NodePath} pocket}:] Contient l'instance d'un noeud 
		de type \href{https://docs.godotengine.org/en/stable/classes/class_sprite.html}
		{\textit{\textcolor{darkgreen}{Sprite}}} ou
		\href{https://docs.godotengine.org/en/stable/classes/class_texturerect.html}
		{\textit{\textcolor{darkgreen}{TextureRect}}}. Cette clé permettra au développeur de 
		pouvoir spécifier les poches graphiques de son inventaire. Dans ce cas, l'inventaire est une 
		interface utilisateur préconstruite par le développeur. L'utilisation de cette clé est obligatoire.
		\\
		\item[>> \textbf{\textcolor{darkgreen}{String | NodePath} label}:] Contient l'instance d'un noeud 
		de type \href{https://docs.godotengine.org/en/stable/classes/class_label.html}
		{\textit{\textcolor{darkgreen}{Label}}}. Cette clé permettra au \\développeur de pouvoir spécifier 
		les textes graphiques de son inventaire. Dans ce cas, l'inventaire est une interface utilisateur 
		préconstruite par le développeur. L'utilisation de cette clé est \\obligatoire si le libelé spécifié
		n'est pas un descendant directe de l'instance de la poche renseignée.\\
		\item[>> \textbf{\textcolor{red}{bool} disabled = \textcolor{red}{false}}:] Devons nous considérer 
		cette poche comme étant une espace disponible au sein de l'inventaire ? Notez que les poches 
		désactivées ne seront jamais prises en par le module.\\
	\end{itemize}
	\textcolor{red}{\textbf{NB}:} Les répétitions au niveau des noms des variétés ne seront pas tolérées.
	Gardez à l'esprit que ce \\module se sert d'un curseur pour sélectionner les poches définient par le
	développeur. Ce curseur n'est rien d'autre que l'index de position de chaque configuration d'élément.
	Par défaut, sa valeur est \textbf{\textcolor{blue}{0}} \\lorsqu'il y a un ou plusieurs élément(s)
	configuré(s) sur le module et \textbf{\textcolor{blue}{-1}} si aucun(s) élément n'est \\disponible sur
	le module.

	% Inventory methods definition.
	\section{Les méthodes disponibles}
	% Bool is_empty () method description.
	\begin{description}
		\item [+ \textcolor{red}{bool} \textcolor{blue}{is\_empty} ():] L'inventaire est-il vide ?\\
	\end{description}
	% Bool is_filled () method description.
	\begin{description}
		\item [+ \textcolor{red}{bool} \textcolor{blue}{is\_filled} ():] L'inventaire est-il remplit ?\\
	\end{description}
	% Void add_objects () method description.
	\begin{description}
		\item [+ \textcolor{red}{void} \textcolor{blue}{add\_objects} (data, reversed = false, delay = 
		0.0):] Ajoute des objets dans l'inventaire.
		\begin{itemize}
			\item [>> \textbf{\textcolor{darkgreen}{Array} data}:] Tableau de dictionaire contenant les 
			différentes variétés ainsi que leur stock. Les dictionnaires supportent les clés suivantes:
			\begin{itemize}
				\item[>> \textbf{\textcolor{darkgreen}{String} | \textcolor{red}{int} id}:] Contient le 
				nom ou l'index de position de la variété à ciblée.
				\item[>> \textbf{\textcolor{red}{int} stock = \textcolor{blue}{1}}:] Quelle est la valeur à 
				ajoutée au stock de cette variété ?
			\end{itemize}
			\item [>> \textbf{\textcolor{red}{bool} reversed}:] Doit-on renversé l'ordre d'ajout des objets
			dans l'inventaire ?
			\item [>> \textbf{\textcolor{red}{float} delay}:] Quel est le temps mort avant le(s) ajout(s) ?
			\\
		\end{itemize}
	\end{description}
	% Void remove_objects () method description.
	\begin{description}
		\item [+ \textcolor{red}{void} \textcolor{blue}{remove\_objects} (data, reversed = false, delay = 
		0.0):] Supprime des objets de l'inventaire.
		\begin{itemize}
			\item [>> \textbf{\textcolor{darkgreen}{Array} data}:] Tableau de dictionaire contenant les 
			différentes variétés ainsi que leur stock. Les dictionnaires supportent les clés suivantes:
			\begin{itemize}
				\item[>> \textbf{\textcolor{darkgreen}{String} | \textcolor{red}{int} id}:] Contient le 
				nom ou l'index de position de la variété à ciblée.
				\item[>> \textbf{\textcolor{red}{int} stock = \textcolor{blue}{1}}:] Quelle est la valeur à 
				enlevée au stock de cette variété ?
			\end{itemize}
			\item [>> \textbf{\textcolor{red}{bool} reversed}:] Doit-on renversé l'ordre de suppression des 
			objets dans l'inventaire ?
			\item [>> \textbf{\textcolor{red}{float} delay}:] Quel est le temps mort avant le(s) 
			suppression(s) ?\\
		\end{itemize}
	\end{description}
	% Void clear () method description.
	\begin{description}
		\item [+ \textcolor{red}{void} \textcolor{blue}{clear} (id = null, delay = 0.0):] Détruit 
		complètement une variété de l'inventaire. Si aucune variété n'a été précisé, L'inventaire complet 
		sera nétoyé.
		\begin{itemize}
			\item[>> \textbf{\textcolor{darkgreen}{String | PoolStringArray | PoolIntArray} | 
			\textcolor{red}{int} id}:] Quelle(s) variété(s) voulez-vous détruire ?
			\item [>> \textbf{\textcolor{red}{float} delay}:] Quel est le temps mort avant le(s) 
			destruction(s) ?\\
		\end{itemize}
	\end{description}
	% Int get_current_stock () method description.
	\begin{description}
		\item [+ \textcolor{red}{int} \textcolor{blue}{get\_current\_stock} (id = null, pocket\_index = 
		-1, on\_config = false):] Renvoie le stock \\actuel d'une variété. Notez que si aucune paramètre 
		n'est précisée, le stock total de l'inventaire sera renvoyé. Si la variété a été précisée, le stock 
		total de cette dernière sera renvoyé. Dans le cas contraire, le stock contenu dans la poche précisée 
		sera renvoyé.
		\begin{itemize}
			\item[>> \textbf{\textcolor{darkgreen}{String} | \textcolor{red}{int} id}:] Contient le nom ou 
			l'index de position de la variété en question.
			\item[>> \textbf{\textcolor{darkgreen}{String} | \textcolor{red}{int} pocket\_index}:] Contient 
			l'index de position d'une poche quelconque.
			\item[>> \textbf{\textcolor{red}{bool} on\_config}:] Doit-on utilisé les configurations 
			éffectuées au niveau du module pour donner le résultat ? Par défaut, le module utilise les 
			valeurs issues de la progression de l'inventaire pour donner le résultat final.\\
		\end{itemize}
	\end{description}
	% Void refresh () method description.
	\begin{description}
		\item [+ \textcolor{red}{void} \textcolor{blue}{refresh} (delay = 0.0):] Réactualise l'inventaire.
		\begin{itemize}
			\item [>> \textbf{\textcolor{red}{float} delay}:] Quel est le temps mort avant le 
			rafraîchissement ?\\
		\end{itemize}
	\end{description}
	% Void combine () method description.
	\begin{description}
		\item [+ \textcolor{red}{int} \textcolor{blue}{combine} (first, second, delay = 0.0):] Effectue la 
		combinaison de deux vériétés différentes pour donner une autre variété. Cette méthode utilise la clé
		\textit{\hyperlink{combination}{combination}} du champ \textit{\hyperlink{varieties}{Varieties}} 
		pour mener à bien son travail.
		\begin{itemize}
			\item[>> \textbf{\textcolor{darkgreen}{String} | \textcolor{red}{int} first}:] Contient le 
			nom ou l'index de position de la première variété à combinée.
			\item[>> \textbf{\textcolor{darkgreen}{String} | \textcolor{red}{int} second}:] Contient le 
			nom ou l'index de position de la deuxième variété à combinée.
			\item [>> \textbf{\textcolor{red}{float} delay}:] Quel est le temps mort avant la combinaison ?
			\\
		\end{itemize}
	\end{description}
	% Dictionary get_data () method description.
	\begin{description}
		\item [+ \textcolor{darkgreen}{Dictionary} \textcolor{blue}{get\_data} (json = true):] Renvoie les 
		données sur la gestion des \\variétés de l'inventaire.
		\begin{itemize}
			\item [>> \textbf{\textcolor{red}{bool} json}:] Voulez-vous renvoyer le résultat sous le format 
			json ?\\
		\end{itemize}
	\end{description}
	% Int | PoolIntArray get_varieties_index () method description.
	\begin{description}
		\item [+ \textcolor{red}{int} | \textcolor{darkgreen}{PoolIntArray} \textcolor{blue}
		{get\_varieties\_index} (id):] Renvoie le(s) index de position d'une ou de \\plusieurs variété(s) en 
		fonction de(s) identifiant d'une ou de plusieurs poche(s).
		\begin{itemize}
			\item[>> \textbf{\textcolor{red}{int} | \textcolor{darkgreen}{PoolStringArray | PoolIntArray | 
			String} id}:] Contient le(s) index de position d'une ou de plusieurs poche(s) quelconque(s). 
			Notez que si vous mettez le(s) nom(s) d'une ou de plusieurs variété(s) définit dans le module, 
			son/leur index de position sera renvoyé.\\
		\end{itemize}
	\end{description}
	% Void cancel_operation () method description.
	\begin{description}
		\item [+ \textcolor{red}{void} \textcolor{blue}{cancel\_operation} (delay = 0.0):] Annule une 
		opération en cours d'exécution. Notez que l'annulation écrase également les progressions éffectuées
		par l'opération en question avant son arrêt. Cette méthode ne travail que si l'inventaire passe en 
		mode interface utilisateur.
		\begin{itemize}
			\item [>> \textbf{\textcolor{red}{float} delay}:] Quel est le temps mort avant l'annulation ?\\
		\end{itemize}
	\end{description}
	% Void confirm_operation () method description.
	\begin{description}
		\item [+ \textcolor{red}{void} \textcolor{blue}{confirm\_operation} (delay = 0.0):] Valide une 
		opération en cours d'exécution. Notez que la validation savegarde également les progressions 
		éffectuées par l'opération en question avant son arrêt. Cette méthode ne travail que si l'inventaire 
		passe en mode interface utilisateur.
		\begin{itemize}
			\item [>> \textbf{\textcolor{red}{float} delay}:] Quel est le temps mort avant la validation ?\\
		\end{itemize}
	\end{description}
	% Void rotate_pocket () method description.
	\begin{description}
		\item [+ \textcolor{red}{int} \textcolor{blue}{rotate\_pocket} (index, angle = 90.0, delay = 0.0):]
		Tourne une poche graphique de \\l'inventaire. Cette méthode ne travail que si l'inventaire passe en 
		mode interface utilisateur.
		\begin{itemize}
			\item[>> \textbf{\textcolor{red}{int} index}:] Contient l'index de position de la poche à 
			ciblée.
			\item[>> \textbf{\textcolor{red}{float} angle}:] Quel est l'angle de rotation de la poche ?
			\item [>> \textbf{\textcolor{red}{float} delay}:] Quel est le temps mort avant la rotation ?\\
		\end{itemize}
	\end{description}
	% Int | PoolIntArray get_pockets_index () method description.
	\newpage \begin{description}
		\item [+ \textcolor{red}{int} | \textcolor{darkgreen}{PoolIntArray} \textcolor{blue}
		{get\_pockets\_index} (variety\_id):] Renvoie le(s) index de position d'une ou plusieurs poche(s) en 
		fonction de l'identifiant d'une variété.
		\begin{itemize}
			\item[>> \textbf{\textcolor{red}{int} | \textcolor{darkgreen}{String | PoolStringArray | 
			PoolIntArray} variety\_id}:] Contient le(s) nom(s) ou le(s) \\index de position d'une ou de 
			plusieurs variété(s) quelconque(s).\\
		\end{itemize}
	\end{description}
	% Int get_cursor_position () method description.
	\begin{description}
		\item [+ \textcolor{red}{int} \textcolor{blue}{get\_cursor\_position} ():] Renvoie la position du
		sélecteur de l'inventaire. Les valeurs des \\positions du curseur sera égale à celles des poches 
		disponibles au sein de l'inventaire. Cette \\méthode ne travail que si l'inventaire passe en mode 
		interface utilisateur.\\
	\end{description}
	% Void set_cursor_position () method description.
	\begin{description}
		\item [+ \textcolor{red}{void} \textcolor{blue}{set\_cursor\_position} (new\_value, delay = 0.0):] 
		Change la position du sélecteur de \\l'inventaire. Les valeurs doivent être dans l'intervalle 
		[\textcolor{blue}{0}; (NombreTotalDePocheDisponible - \textcolor{blue}{1})]. Cette méthode ne 
		travail que si l'inventaire passe en mode interface utilisateur.
		\begin{itemize}
			\item[>> \textbf{\textcolor{red}{int} | \textcolor{darkgreen}{String} new\_value}:] Contient la 
			nouvelle valeur du sélecteur.
			\item [>> \textbf{\textcolor{red}{float} delay}:] Quel est le temps mort avant le changement de 
			valeur ?\\
		\end{itemize}
	\end{description}
	% Int | PoolIntArray get_empty_pockets_index () method description.
	\begin{description}
		\item [+ \textcolor{red}{int} | \textcolor{darkgreen}{PoolIntArray} \textcolor{blue}
		{get\_empty\_pockets\_index} (variety\_id):] Renvoie le(s) index de position d'une ou plusieurs 
		poche(s) vide(s).\\
	\end{description}
	% Int | PoolIntArray get_busy_pockets_index () method description.
	\begin{description}
		\item [+ \textcolor{red}{int} | \textcolor{darkgreen}{PoolIntArray} \textcolor{blue}
		{get\_busy\_pockets\_index} (variety\_id):] Renvoie le(s) index de position d'une ou plusieurs 
		poche(s) occupée(s).\\
	\end{description}
	% Int | PoolIntArray get_uncastable_varieties_index () method description.
	\begin{description}
		\item [+ \textcolor{red}{int} | \textcolor{darkgreen}{PoolIntArray} \textcolor{blue}
		{get\_uncastable\_varieties\_index} (on\_config = false):] Renvoie le(s) index de position de la ou 
		des variété(s) injetable(s).
		\begin{itemize}
			\item[>> \textbf{\textcolor{red}{bool} on\_config}:] Doit-on utilisé les configurations 
			éffectuées au niveau du module pour donner le résultat ? Par défaut, le module utilise les 
			valeurs issues de la progression de l'inventaire pour donner le résultat final.\\
		\end{itemize}
	\end{description}
	% Int | PoolIntArray get_combinable_varieties_index () method description.
	\begin{description}
		\item [+ \textcolor{red}{int} | \textcolor{darkgreen}{PoolIntArray} \textcolor{blue}
		{get\_combinable\_varieties\_index} (on\_config = false):] Renvoie le(s) index de position de la ou 
		des variété(s) combinable(s).
		\begin{itemize}
			\item[>> \textbf{\textcolor{red}{bool} on\_config}:] Doit-on utilisé les configurations 
			éffectuées au niveau du module pour donner le résultat ? Par défaut, le module utilise les 
			valeurs issues de la progression de l'inventaire pour donner le résultat final.\\
		\end{itemize}
	\end{description}
	% Int get_big_stock () method description.
	\begin{description}
		\item [+ \textcolor{red}{int} \textcolor{blue}{get\_big\_stock} (on\_config = false):] Renvoie la 
		valeur du stock le plus grand dans \\l'inventaire.
		\begin{itemize}
			\item[>> \textbf{\textcolor{red}{bool} on\_config}:] Doit-on utilisé les configurations 
			éffectuées au niveau du module pour donner le résultat ? Par défaut, le module utilise les 
			valeurs issues de la progression de l'inventaire pour donner le résultat final.\\
		\end{itemize}
	\end{description}
	% Int get_tiny_stock () method description.
	\newpage \begin{description}
		\item [+ \textcolor{red}{int} \textcolor{blue}{get\_tiny\_stock} (on\_config = false):] Renvoie la 
		valeur du stock le plus petit dans \\l'inventaire.
		\begin{itemize}
			\item[>> \textbf{\textcolor{red}{bool} on\_config}:] Doit-on utilisé les configurations 
			éffectuées au niveau du module pour donner le résultat ? Par défaut, le module utilise les 
			valeurs issues de la progression de l'inventaire pour donner le résultat final.\\
		\end{itemize}
	\end{description}
	% Void run_operation () method description.
	\begin{description}
		\item [+ \textcolor{red}{void} \textcolor{blue}{run\_operation} (operation, delay = 0.0):] Exécute 
		une opération de l'inventaire. Cette \\méthode ne travail que si l'inventaire passe en mode 
		interface utilisateur.
		\begin{itemize}
			\item[>> \textbf{\textcolor{red}{int} operation}:] Contient l'opération à exécutée. Les valeurs
			possibles sont:
			\begin{itemize}
				\item [-> \textbf{\textcolor{gray}{InventoryFx.Operation.IDLE} ou \textcolor{blue}{0}}:] 
				Mode normal de l'inventaire.
				\item [-> \textbf{\textcolor{gray}{InventoryFx.Operation.STATIC\_MOVING} ou \textcolor{blue}
				{1}}:] Mode déplacement statique d'une variété de l'inventaire.
				\item [-> \textbf{\textcolor{gray}{InventoryFx.Operation.DYNAMIC\_MOVING} ou 
				\textcolor{blue}{2}}:] Mode déplacement dynamique d'une variété de l'inventaire.
				\item [-> \textbf{\textcolor{gray}{InventoryFx.COMBINATION} ou \textcolor{blue}{3}}:] 
				Mode combinaison de variétés.
			\end{itemize}
			\item [>> \textbf{\textcolor{red}{float} delay}:] Quel est le temps mort avant l'exécution ?\\
		\end{itemize}
	\end{description}
	% Void bdi_run_operation () method description.
	\begin{description}
		\item [+ \textcolor{red}{void} \textcolor{blue}{bdi\_run\_operation} (operation, delay = 0.0):] 
		Exécute une opération de l'inventaire en \\mode bidirectionelle. Cette méthode ne travail que si 
		l'inventaire passe en mode interface \\utilisateur.
		\begin{itemize}
			\item[>> \textbf{\textcolor{red}{int} operation}:] Contient l'opération à exécutée. Les valeurs
			possibles sont:
			\begin{itemize}
				\item [-> \textbf{\textcolor{gray}{InventoryFx.Operation.IDLE} ou \textcolor{blue}{0}}:] 
				Mode normal de l'inventaire.
				\item [-> \textbf{\textcolor{gray}{InventoryFx.Operation.STATIC\_MOVING} ou \textcolor{blue}
				{1}}:] Mode déplacement statique d'une variété de l'inventaire.
				\item [-> \textbf{\textcolor{gray}{InventoryFx.Operation.DYNAMIC\_MOVING} ou 
				\textcolor{blue}{2}}:] Mode déplacement dynamique d'une variété de l'inventaire.
				\item [-> \textbf{\textcolor{gray}{InventoryFx.COMBINATION} ou \textcolor{blue}{3}}:] 
				Mode combinaison de variétés.
			\end{itemize}
			\item [>> \textbf{\textcolor{red}{float} delay}:] Quel est le temps mort avant l'exécution ?\\
		\end{itemize}
	\end{description}

	% Inventory signals definition.
	\section{Les événements disponibles}
	% inventory_empty event description.
	\begin{description}
		\item [+ \textcolor{blue}{inventory\_empty} (node):] Signal déclenché lorsque l'inventaire est vide.
		\begin{itemize}
			\item [>> \textbf{\textcolor{darkgreen}{Node} node}:] Contient le noeud où ce signal a été émit.
			\\
		\end{itemize}
	\end{description}
	% inventory_filled event description.
	\begin{description}
		\item [+ \textcolor{blue}{inventory\_filled} (node):] Signal déclenché lorsque l'inventaire est 
		remplit.
		\begin{itemize}
			\item [>> \textbf{\textcolor{darkgreen}{Node} node}:] Contient le noeud où ce signal a été émit.
			\\
		\end{itemize}
	\end{description}
	% inventory_updated event description.
	\begin{description}
		\item [+ \textcolor{blue}{inventory\_updated} (node):] Signal déclenché lorsque l'inventaire est 
		mise à jour.
		\begin{itemize}
			\item [>> \textbf{\textcolor{darkgreen}{Node} node}:] Contient le noeud où ce signal a été émit.
			\\
		\end{itemize}
	\end{description}
	% operation_started event description.
	\begin{description}
		\item [+ \textcolor{blue}{operation\_started} (data):] Signal déclenché au démarrage d'une opération 
		de l'inventaire. Ce événement renvoie un dictionaire contenant les clés suivantes:
		\begin{itemize}
			\item [>> \textbf{\textcolor{darkgreen}{Node} node}:] Contient le noeud où ce signal a été émit.
			\item [>> \textbf{\textcolor{red}{int} | \textbf{\textcolor{darkgreen}{PoolIntArray} 
			accepted}}:] Contient le(s) index de position de la ou des poche(s) vérifiant les conditions que 
			pose l'opération en question.
			\item [>> \textbf{\textcolor{red}{int} | \textbf{\textcolor{darkgreen}{PoolIntArray} denied}}:] 
			Contient le(s) index de position de la ou des poche(s) ne vérifiant pas les conditions que pose 
			l'opération en question.\\
		\end{itemize}
	\end{description}
	% operation_running event description.
	\begin{description}
		\item [+ \textcolor{blue}{operation\_running} (data):] Signal déclenché tanqu'une opération sera en 
		cours d'exécution. Ce événement renvoie un dictionaire contenant les clés suivantes:
		\begin{itemize}
			\item [>> \textbf{\textcolor{darkgreen}{Node} node}:] Contient le noeud où ce signal a été émit.
			\item [>> \textbf{\textcolor{red}{int} operation}:] Contient l'opération en cours d'exécution.\\
		\end{itemize}
	\end{description}
	% operation_stoped event description.
	\begin{description}
		\item [+ \textcolor{blue}{operation\_stoped} (data):] Signal déclenché lorsqu'on arrête l'exécution 
		d'une opération de \\l'inventaire. Ce événement renvoie un dictionaire contenant les clés suivantes:
		\begin{itemize}
			\item [>> \textbf{\textcolor{darkgreen}{Node} node}:] Contient le noeud où ce signal a été émit.
			\item [>> \textbf{\textcolor{red}{int} operation}:] Contient l'opération qui a été arrêtée.\\
		\end{itemize}
	\end{description}
	% operation_failed event description.
	\begin{description}
		\item [+ \textcolor{blue}{operation\_failed} (data):] Signal déclenché lorsque l'exécution de la 
		tâche principale de \\l'opération en question a échoué. Ce événement renvoie un dictionaire 
		contenant les clés \\suivantes:
		\begin{itemize}
			\item [>> \textbf{\textcolor{darkgreen}{Node} node}:] Contient le noeud où ce signal a été émit.
			\item [>> \textbf{\textcolor{red}{int} operation}:] Contient l'opération ayant provoquée 
			l'erreur.\\
		\end{itemize}
	\end{description}
\end{document}