% Document type and package imports.
\documentclass[a4paper, 11pt]{article}
\usepackage[utf8]{inputenc}
\usepackage[T1]{fontenc}
\usepackage[french]{babel}
\usepackage{charter}
\usepackage[top = 2cm, bottom = 2cm, left = 1cm, right = 1cm]{geometry}
\usepackage{setspace}
\usepackage{color}
\usepackage{xcolor}
\usepackage{hyperref}
\usepackage{tocloft}

% Preanblue.
\onehalfspacing
\definecolor{gray}{rgb}{0.4, 0.4, 0.4}
\definecolor{silver}{rgb}{0.95, 0.95, 0.95}
\renewcommand{\thesection}{\Roman{section} --}
\definecolor{darkgreen}{HTML}{1E8C15}
\cftsetindents{section}{1em}{2.5em}
\hypersetup {colorlinks=true, linkcolor=blue, urlcolor=blue, pdftitle={Menu module doc}}

% The start of the article.
\begin{document}
	% Change document background to silver color.
	\pagecolor{silver}
	% Menu module description.
	\huge{\hspace{15.8cm}\textit{\textbf{\textcolor{darkgreen}{MenuFx}}}}\large{} \tableofcontents \newpage
	% Menu module definition.
	\section{Définition}
	\textcolor{darkgreen}{\textbf{MenuFx}} est un module conçut pour la gestion des menu dans un jeu vidéo.
	Vous pouvez réaliser toute sorte de menu rien qu'en utilisant ce module. Le module implémente les règles
	de bases de réalisation d'un menu.\\
	\textcolor{red}{\textbf{NB}:} Ce module est compatible à un jeu 2D, 3D et est sauvegardable.

	% Menu properties definition.
	\section{Les propriétés disponibles}
	% ColumnCount property.
	\textbf{+ \textcolor{red}{int} \hypertarget{ccount}{ColumnCount} = \textcolor{blue}{0}:} Contient le 
	nombre de colonne sur l'horizontal. La valeur de cette \\propriété est dans l'intervalle 
	[\textcolor{blue}{0}; \textcolor{blue}{++}].\\\\
	% RowCount property.
	\textbf{+ \textcolor{red}{int} \hypertarget{rcount}{RowCount} = \textcolor{blue}{0}:} Contient le nombre 
	de ligne sur la vertical. La valeur de cette propriété est dans l'intervalle [\textcolor{blue}{0}; 
	\textcolor{blue}{++}].\\\\
	% InfiniteScrolling property.
	\textbf{+ \textcolor{red}{bool} InfiniteScrolling = \textcolor{red}{false}:} Voulez-vous mettre un éffet 
	de sélection infinie au cours des \\sélections ?\\\\
	% MouseCapture property.
	\textbf{+ \textcolor{red}{bool} MouseCapture = \textcolor{red}{true}:} Désirez-vous sychrôniser la 
	souris au menu ?\\\\
	% VolatilTime property.
	\textbf{+ \textcolor{red}{float} VolatilTime = \textcolor{blue}{0.0}:} Souhaitez-vous faire un menu 
	volatil ? Toute valeur inférieur ou égale à \textcolor{blue}{0.0} désactive l'activité de cette 
	propriété. Pour exploiter cette option, il faut écouté les événements \textit{\hyperlink{mbegin}
	{menu\_begin}} et \textit{\hyperlink{mend}{menu\_end}}. La valeur de cette propriété est dans 
	l'intervalle [\textcolor{blue}{0.0}; \textcolor{blue}{++}].\\\\
	% ExternalSelector property.
	\textbf{+ \textcolor{red}{bool} ExternalSelector = \textcolor{red}{false}:} Devrons nous mettre un 
	sélecteur externe au menu ? Le sélecteur externe fait référence à un objet externe utilisé pour 
	matérialiser la sélection des différentes options du menu.\\\\
	% SelectorTransition property.
	\textbf{+ \textcolor{darkgreen}{Vector3} SelectorTransition = \textcolor{darkgreen}{Vector3}
	(\textcolor{blue}{0.0}, \textcolor{blue}{0}, \textcolor{blue}{2}):} Devrons nous appliquer une 
	quelconque \\animation vectorielle à votre sélecteur externe ? La première valeur contient le temps de 
	la transition; la seconde l'assouplissement et la dernière, le type de transition à utilisé. Notez les 
	valeurs possibles des deux dernières entrées de cette propriété sont celles de Godot.\\\\
	% ListenModule property.
	\textbf{+ \textcolor{darkgreen}{NodePath} ListenModule:} Contient l'instance d'un module du même trampe
	de ce module. \\Exemple: \textit{\textcolor{darkgreen}{AnimatorFx, MenuFx, AudioTrackFx}} etc... En
	d'autres termes, contient la référence d'un \\module possèdant également un curseur. Le but de cette
	fonctionnalité est de synchrôniser la valeur du curseur du module référé à celle de ce module.\\\\
	% HozScrollBar property.
	\textbf{+ \textcolor{darkgreen}{NodePath} HozScrollBar:} Contient l'instance d'un noeud de type
	\href{https://docs.godotengine.org/fr/stable/classes/class_hscrollbar.html}
	{\textit{\textcolor{darkgreen}{HScrollBar}}}.\\\\
	% VerScrollBar property.
	\textbf{+ \textcolor{darkgreen}{NodePath} VerScrollBar:} Contient l'instance d'un noeud de type
	\href{https://docs.godotengine.org/fr/stable/classes/class_vscrollbar.html}
	{\textit{\textcolor{darkgreen}{VScrollBar}}}.\\\\
	% ScrollStep property.
	\newpage \textbf{+ \textcolor{red}{int} ScrollStep = \textcolor{blue}{0}:} Quel est le pas à ajouté pour 
	défiler le défileur du menu ? La valeur de cette propriété est dans l'intervalle [\textcolor{blue}{0}
	\textcolor{blue}{++}].\\\\
	% Options property.
	\textbf{+ \textcolor{darkgreen}{Array} Options:} Tableau de dictionnaires contenant toutes les 
	différentes configurations sur chaque option du menu que le développeur souhaite implémentée. Les 
	dictionnaires issus de ce tableau \\supportent les clés suivantes:\\
	\begin{itemize}
		\item[>> \textbf{\textcolor{darkgreen}{NodePath | String} option}:] Quel est le chemin d'accès
		l'option que vous voulez ajouter ? \\L'utilisation de cette clé est obligatoire.\\
		\item[>> \textbf{\textcolor{darkgreen}{String} name}:] Quel nom désirez-vous donner à votre option ?
		\\
		\item[>> \textbf{\textcolor{red}{bool} ignored = \textcolor{red}{false}}:] Voulez-vous que le module 
		ignore cette option à chaque fois que l'on \\mettra le focus dessus ?\\
		\item[>> \textbf{\textcolor{red}{bool} locked = \textcolor{red}{false}}:] Voulez-vous désactiver 
		le(s) action(s) que génère cette option ?\\
		\item[>> \textbf{\textcolor{red}{bool} tabcontrol = \textcolor{red}{false}}:] Souhaitez-vous 
		exécuter le(s) action(s) prévue(s) sur l'option lorsqu'on la sélectionne directement ?\\
		\item[>> \textbf{\textcolor{darkgreen}{Array | Dictionary} actions}]: Que se passera t-il lorsqu'on 
		exécutera le(s) action(s) que cache \\l'option ? Cette clé contient toutes les configurations
		relatives à un ou plusieurs flux \\d'exécution(s). L'utilisation de cette clé est déjà décrite au
		niveau des bases du framework. \\Précisement le sujet portant sur l'utilisation de la propriété 
		\textit{\textcolor{gray}{EventsBindings}} (la section des actions d'un événement).\\
	\end{itemize}
	\textcolor{red}{\textbf{NB}:} Les répétitions au niveau des noms des options ne seront pas tolérées.

	% Menu methods definition.
	\section{Les méthodes disponibles}
	% Void next () method description.
	\begin{description}
		\item [+ \textcolor{red}{void} \textcolor{blue}{next} (delay = 0.0):] Passe à l'option suivante dans
		le menu établit.
		\begin{itemize}
			\item [>> \textbf{\textcolor{red}{float} delay}:] Quel est le temps mort avant le passage.\\
		\end{itemize}
	\end{description}
	% Void h_next () method description.
	\begin{description}
		\item [+ \textcolor{red}{void} \textcolor{blue}{h\_next} (delay = 0.0):] Passe à l'option suivante 
		dans le menu vertical. N'utilisez cette \\méthode que le menu est configuré en deux dimension à 
		savoir les valeurs des champs \textit{\hyperlink{ccount}{\\ColumnCount}} et 
		\textit{\hyperlink{rcount}{RowCount}} sont supérieur à zéro. Notez que si l'une des valeurs de ces 
		deux champs est supérieur à zéro et que l'autre est nulle, le menu est configuré en une dimension. 
		Il est très important de comprendre cela.
		\begin{itemize}
			\item [>> \textbf{\textcolor{red}{float} delay}:] Quel est le temps mort avant le passage.\\
		\end{itemize}
	\end{description}
	% Void v_next () method description.
	\newpage \begin{description}
		\item [+ \textcolor{red}{void} \textcolor{blue}{v\_next} (delay = 0.0):] Passe à l'option suivante 
		dans le menu horizontal. N'utilisez cette méthode que le menu est configuré en deux dimension à 
		savoir les valeurs des champs \textit{\hyperlink{ccount}{\\ColumnCount}} et 
		\textit{\hyperlink{rcount}{RowCount}} sont supérieur à zéro. Notez que si l'une des valeurs de ces 
		deux champs est supérieur à zéro et que l'autre est nulle, le menu est configuré en une dimension. 
		Il est très important de comprendre cela.
		\begin{itemize}
			\item [>> \textbf{\textcolor{red}{float} delay}:] Quel est le temps mort avant le passage.\\
		\end{itemize}
	\end{description}
	% Void preview () method description.
	\begin{description}
		\item [+ \textcolor{red}{void} \textcolor{blue}{preview} (delay = 0.0):] Passe à l'option précédante 
		dans le menu établit.
		\begin{itemize}
			\item [>> \textbf{\textcolor{red}{float} delay}:] Quel est le temps mort avant le passage.\\
		\end{itemize}
	\end{description}
	% Void h_preview () method description.
	\begin{description}
		\item [+ \textcolor{red}{void} \textcolor{blue}{h\_preview} (delay = 0.0):] Passe à l'option 
		précédante dans le menu vertical. N'utilisez cette méthode que le menu est configuré en deux 
		dimension à savoir les valeurs des champs \textit{\hyperlink{ccount}{\\ColumnCount}} et 
		\textit{\hyperlink{rcount}{RowCount}} sont supérieur à zéro. Notez que si l'une des valeurs de ces 
		deux champs est supérieur à zéro et que l'autre est nulle, le menu est configuré en une dimension. 
		Il est très important de comprendre cela.
		\begin{itemize}
			\item [>> \textbf{\textcolor{red}{float} delay}:] Quel est le temps mort avant le passage.\\
		\end{itemize}
	\end{description}
	% Void v_preview () method description.
	\begin{description}
		\item [+ \textcolor{red}{void} \textcolor{blue}{v\_preview} (delay = 0.0):] Passe à l'option 
		précédante dans le menu horizontal. N'utilisez cette méthode que le menu est configuré en deux 
		dimension à savoir les valeurs des champs \textit{\hyperlink{ccount}{ColumnCount}} et 
		\textit{\hyperlink{rcount}{RowCount}} sont supérieur à zéro. Notez que si l'une des valeurs de ces 
		deux champs est supérieur à zéro et que l'autre est nulle, le menu est configuré en une dimension. 
		Il est très important de comprendre cela.
		\begin{itemize}
			\item [>> \textbf{\textcolor{red}{float} delay}:] Quel est le temps mort avant le passage.\\
		\end{itemize}
	\end{description}
	% Void next_to () method description.
	\begin{description}
		\item [+ \textcolor{red}{void} \textcolor{blue}{next\_to} (id, delay = 0.0):] Change d'option grâce
		à son identifiant.
		\begin{itemize}
			\item [>> \textbf{\textcolor{darkgreen}{String} | \textcolor{red}{int} id}:] Quel est 
			l'identifiant de l'option à rendre active ?
			\item [>> \textbf{\textcolor{red}{float} delay}:] Quel est le temps mort avant le passage.\\
		\end{itemize}
	\end{description}
	% Int get_active_option_index () method description.
	\begin{description}
		\item [+ \textcolor{red}{int} \textcolor{blue}{get\_active\_option\_index} ():] Renvoie l'index de
		l'option actuellement sélectionnée dans le menu établit.\\
	\end{description}
	% Int get_preview_option_index () method description.
	\begin{description}
		\item [+ \textcolor{red}{int} \textcolor{blue}{get\_preview\_option\_index} ():] Renvoie l'index de
		l'option pécédement sélectionnée dans le menu établit.\\
	\end{description}
	% Int get_next_option_index () method description.
	\begin{description}
		\item [+ \textcolor{red}{int} \textcolor{blue}{get\_next\_option\_index} ():] Renvoie l'index de
		la prochaine option à sélectionnée dans le menu établit.\\
	\end{description}
	% Void run_actions () method description.
	\begin{description}
		\item [+ \textcolor{red}{void} \textcolor{blue}{\hypertarget{runactions}{run\_actions}} (id = null, 
		delay = 0.0):] Exécute toutes les actions que cache une option \\donnée. Par défaut, l'exécution se 
		fera sur l'option sélectionnée.
		\begin{itemize}
			\item [>> \textbf{\textcolor{darkgreen}{String} | \textcolor{red}{int} id}:] Quel est 
			l'identifiant de l'option à ciblée ?
			\item [>> \textbf{\textcolor{red}{float} delay}:] Quel est le temps mort avant les éxécutions.\\
		\end{itemize}
	\end{description}

	% Menu signals definition.
	\section{Les événements disponibles}
	% menu_action event description.
	\begin{description}
		\item [+ \textcolor{blue}{menu\_action} (data):] Signal déclenché lorsqu'on exécute le(s) action(s)
		d'une option donnée. Ce événement renvoie un dictionaire contenant les clés suivantes:
		\begin{itemize}
			\item [>> \textbf{\textcolor{darkgreen}{Node} node}:] Contient le noeud où ce signal a été émit.
			\item [>> \textbf{\textcolor{red}{int} index}:] Contient l'index de position de l'option ayant 
			été frappée par la méthode \textit{\hyperlink{runactions}{run\_actions ()}}.\\
		\end{itemize}
	\end{description}
	% option_changed event description.
	\begin{description}
		\item [+ \textcolor{blue}{option\_changed} (data):] Signal déclenché lorsqu'on change d'option dans 
		le menu établit. Ce événement renvoie un dictionaire contenant les clés suivantes:
		\begin{itemize}
			\item [>> \textbf{\textcolor{darkgreen}{Node} node}:] Contient le noeud où ce signal a été émit.
			\item [>> \textbf{\textcolor{red}{int} preview}:] Contient l'index de position de l'option 
			précédement sélectionnée.
			\item [>> \textbf{\textcolor{red}{int} current}:] Contient l'index de position de l'option 
			actuellement sélectionnée.\\
		\end{itemize}
	\end{description}
	% option_ignored event description.
	\begin{description}
		\item [+ \textcolor{blue}{option\_ignored} (data):] Signal déclenché lorsqu'on sélectionne une 
		option qui doit être ignorée par le module. Ce événement renvoie un dictionaire contenant les clés 
		suivantes:
		\begin{itemize}
			\item [>> \textbf{\textcolor{darkgreen}{Node} node}:] Contient le noeud où ce signal a été émit.
			\item [>> \textbf{\textcolor{red}{int} index}:] Contient l'index de position de l'option 
			ignorée.\\
		\end{itemize}
	\end{description}
	% option_locked event description.
	\begin{description}
		\item [+ \textcolor{blue}{option\_locked} (data):] Signal déclenché lorsqu'on déclenche les actions
		d'une option bloquée. Ce événement renvoie un dictionaire contenant les clés suivantes:
		\begin{itemize}
			\item [>> \textbf{\textcolor{darkgreen}{Node} node}:] Contient le noeud où ce signal a été émit.
			\item [>> \textbf{\textcolor{red}{int} index}:] Contient l'index de position de l'option 
			bloquée.\\
		\end{itemize}
	\end{description}
	% menu_begin event description.
	\begin{description}
		\item [+ \textcolor{blue}{\hypertarget{mbegin}{menu\_begin}} (node):] Signal déclenché lorsqu'on 
		appel des fonctions en faveur des options du menu.
		\begin{itemize}
			\item [>> \textbf{\textcolor{darkgreen}{Node} node}:] Contient le noeud où ce signal a été émit.
			\\
		\end{itemize}
	\end{description}
	% menu_end event description.
	\begin{description}
		\item [+ \textcolor{blue}{\hypertarget{mend}{menu\_end}} (node):] Signal déclenché lorsque le temps 
		de volatilisation du menu est écoulé.
		\begin{itemize}
			\item [>> \textbf{\textcolor{darkgreen}{Node} node}:] Contient le noeud où ce signal a été émit.
			\\
		\end{itemize}
	\end{description}
	% mouse_enter event description.
	\begin{description}
		\item [+ \textcolor{blue}{mouse\_enter} (data):] Signal déclenché lorsque le curseur de la souris 
		entre dans le survol d'une option du menu établit. Ce événement renvoie un dictionaire contenant les 
		clés suivantes:
		\begin{itemize}
			\item [>> \textbf{\textcolor{darkgreen}{Node} node}:] Contient le noeud où ce signal a été émit.
			\item [>> \textbf{\textcolor{red}{int} index}:] Contient l'index de position de l'option en 
			question.\\
		\end{itemize}
	\end{description}
	% mouse_exit event description.
	\begin{description}
		\item [+ \textcolor{blue}{mouse\_exit} (data):] Signal déclenché lorsque le curseur de la souris 
		sort du survol d'une option du menu établit. Ce événement renvoie un dictionaire contenant les clés 
		suivantes:
		\begin{itemize}
			\item [>> \textbf{\textcolor{darkgreen}{Node} node}:] Contient le noeud où ce signal a été émit.
			\item [>> \textbf{\textcolor{red}{int} index}:] Contient l'index de position de l'option en 
			question.\\
		\end{itemize}
	\end{description}
	% mouse_over event description.
	\begin{description}
		\item [+ \textcolor{blue}{mouse\_over} (data):] Signal déclenché lorsque le curseur de la souris 
		survol une option du menu établit. Ce événement renvoie un dictionaire contenant les clés suivantes:
		\begin{itemize}
			\item [>> \textbf{\textcolor{darkgreen}{Node} node}:] Contient le noeud où ce signal a été émit.
			\item [>> \textbf{\textcolor{red}{int} index}:] Contient l'index de position de l'option en 
			question.\\
		\end{itemize}
	\end{description}
	% mouse_clicked event description.
	\begin{description}
		\item [+ \textcolor{blue}{mouse\_clicked} (data):] Signal déclenché lorsque le curseur de la souris 
		clique une option du menu établit. Ce événement renvoie un dictionaire contenant les clés suivantes:
		\begin{itemize}
			\item [>> \textbf{\textcolor{darkgreen}{Node} node}:] Contient le noeud où ce signal a été émit.
			\item [>> \textbf{\textcolor{red}{int} index}:] Contient l'index de position de l'option en 
			question.
			\item [>> \textbf{\textcolor{red}{int} button}:] Contient l'index du bouton qui a été appuyée.\\
		\end{itemize}
	\end{description}
\end{document}